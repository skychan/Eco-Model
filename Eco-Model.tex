% !Mode:: "TeX:UTF-8"

%%
%% Copyright 2007, 2008, 2009 Elsevier Ltd
%%
%% This file is part of the 'Elsarticle Bundle'.
%% ---------------------------------------------
%%
%% It may be distributed under the conditions of the LaTeX Project Public
%% License, either version 1.2 of this license or (at your option) any
%% later version.  The latest version of this license is in
%%    http://www.latex-project.org/lppl.txt
%% and version 1.2 or later is part of all distributions of LaTeX
%% version 1999/12/01 or later.
%%
%% The list of all files belonging to the 'Elsarticle Bundle' is
%% given in the file `manifest.txt'.
%%

%% Template article for Elsevier's document class `elsarticle'
%% with numbered style bibliographic references
%% SP 2008/03/01
%%
%%
%%
%% $Id: elsarticle-template-num.tex 4 2009-10-24 08:22:58Z rishi $
%%
%%
\documentclass[twocolumn,twoside,times,x11names,3p]{elsarticle}
%3p,times,procedia,twocolumn,twoside
\graphicspath{{figures/}}
\usepackage{geometry}
\geometry{twoside,
    paperwidth=210mm,
    paperheight=297mm,
    textheight=730pt,
    textwidth=43.5pc,
    lmargin=15mm,
    tmargin=68.18pt,
    headheight=10pt,
    headsep=12pt,
    footskip=12pt,
    footnotesep=24pt,
    columnsep=1.5pc
}

\usepackage[unicode,
    pdfstartview=FitH,
    bookmarksnumbered=true,
    bookmarksopen=true,
    colorlinks=true,
    citecolor=black,
    linkcolor=black,
    anchorcolor=black,
    urlcolor=black,
    %draft,
    breaklinks=true]{hyperref}
\renewcommand\figureautorefname{Fig.}
\renewcommand\tableautorefname{Tab.}
\renewcommand\sectionautorefname{Sec.}
\renewcommand\subsectionautorefname{Sec.}
\renewcommand\subsubsectionautorefname{Sec.}
\newcommand{\subfigureautorefname}{\figureautorefname}

\usepackage{tabularx}                               % 自动设置表格的列宽
\usepackage[caption=false]{subfig}
\usepackage{bm,amsmath,amssymb,amsthm}
%\usepackage[amsmath,thmmarks,hyperref]{ntheorem}    % 定理类环境宏包,其中 amsmath 选项用来兼容 AMS LaTeX 的宏包
\usepackage{cases}      % 括号宏包
\usepackage{paralist}   % 列表宏包
\usepackage{mdwlist}    % 提供suspend 暂停列表功能的宏包

\usepackage{multirow,bigstrut}      % 使用Multirow宏包,使得表格可以合并多个row格
\usepackage{booktabs}               % 表格,横的粗线;\specialrule{1pt}{0pt}{0pt}
\usepackage{supertabular}           % 超表格环境
% \usepackage{caption}
\usepackage{rotating}               % 旋转图表
\usepackage{floatrow}
\usepackage{pifont}                 % 列表符号宏包
\floatsetup[table]{style=Plaintop}


%% 乱七八糟字体编码
\usepackage[T1]{fontenc}

%%% tikz 绘图
\usepackage{tikz}
\usetikzlibrary{mindmap,shapes,arrows,chains,shadows,decorations.markings}
%\usepackage{verbatim}
%\usepackage[active,tightpage]{preview}
%\PreviewEnvironment{tikzpicture}
%\setlength\PreviewBorder{5mm}%
\colorlet{lcfree}{Green3}
\colorlet{lcnorm}{Blue3}
\colorlet{lccong}{Red3}
% -------------------------------------------------
% Set up a new layer for the debugging marks, and make sure it is on
% top
\pgfdeclarelayer{marx}
\pgfsetlayers{main,marx}
% A macro for marking coordinates (specific to the coordinate naming
% scheme used here). Swap the following 2 definitions to deactivate
% marks.
\providecommand{\cmark}[2][]{%
  \begin{pgfonlayer}{marx}
    \node [nmark] at (c#2#1) {#2};
  \end{pgfonlayer}{marx}
  } 
\providecommand{\cmark}[2][]{\relax} 
\usepackage{pgf-umlcd}

%\usepackage{rotfloat}
%\usepackage[all,pdf]{xy}
\usepackage{listings} %插入代码
%\usepackage[dvipsnames]{xcolor} %代码高亮




% \usepackage[utf8]{inputenc}
%% Use the option review to obtain double line spacing
%% \documentclass[preprint,review,12pt]{elsarticle}

%% Use the options 1p,twocolumn; 3p; 3p,twocolumn; 5p; or 5p,twocolumn
%% for a journal layout:
%% \documentclass[final,1p,times]{elsarticle}
%% \documentclass[final,1p,times,twocolumn]{elsarticle}
%% \documentclass[final,3p,times]{elsarticle}
%% \documentclass[final,3p,times,twocolumn]{elsarticle}
%% \documentclass[final,5p,times]{elsarticle}
%% \documentclass[final,5p,times,twocolumn]{elsarticle}

%% if you use PostScript figures in your article
%% use the graphics package for simple commands
%% \usepackage{graphics}
%% or use the graphicx package for more complicated commands
%% \usepackage{graphicx}
%% or use the epsfig package if you prefer to use the old commands
%% \usepackage{epsfig}

%% The amssymb package provides various useful mathematical symbols
%% \usepackage{amssymb}
%% The amsthm package provides extended theorem environments
%% \usepackage{amsthm}

%% The lineno packages adds line numbers. Start line numbering with
%% \begin{linenumbers}, end it with \end{linenumbers}. Or switch it on
%% for the whole article with \linenumbers after \end{frontmatter}.
%% \usepackage{lineno}

%% natbib.sty is loaded by default. However, natbib options can be
%% provided with \biboptions{...} command. Following options are
%% valid:

%%   round  -  round parentheses are used (default)
%%   square -  square brackets are used   [option]
%%   curly  -  curly braces are used      {option}
%%   angle  -  angle brackets are used    <option>
%%   semicolon  -  multiple citations separated by semi-colon
%%   colon  - same as semicolon, an earlier confusion
%%   comma  -  separated by comma
%%   numbers-  selects numerical citations
%%   super  -  numerical citations as superscripts
%%   sort   -  sorts multiple citations according to order in ref. list
%%   sort&compress   -  like sort, but also compresses numerical citations
%%   compress - compresses without sorting
%%
%% \biboptions{comma,round}

% \biboptions{}


% \journal{Nuclear Physics B}

\begin{document}

\begin{frontmatter}

% !TEX root = Eco-Model.tex

\title{Operation Mode Study in Cloud Manufacturing Ecosystem
%Cloud Manufacturing Ecosystem - Scheduling And Evolving
}

\author[add1]{Shengkai Chen}
\author[add2,add1]{Shuiliang Fang\corref{cor1}}
\author[add2,add1]{Tao Peng}
% \ead{me_fangsl@zju.edu.cn}
% \address[label1]{ZJU}
% \address[label1]{Address One}
\address[add2]{The State Key Laboratory of Fluid Power Transmission and Control, College of Mechanical Engineering, Zhejiang University, Hangzhou, 310027, China}
\address[add1]{Key Laboratory of Advanced Manufacturing Technology of Zhejiang Province, College of Mechanical Engineering, Zhejiang University, Hangzhou , 310027, China}
% \author[a,b,*]{Third Author}

% \address[a]{First affiliation, Address, City and Postcode, Country}
% \address[b]{Second affiliation, Address, City and Postcode, Country}
% \fntext[label4]{Small city}

\cortext[cor1]{Corresponding author. Email: me\_fangsl@zju.edu.cn}

\begin{abstract}
In cloud manufacturing, individuals engage in manufacturing business through a well-designed platform, where the operator is able to manage distributed massive manufacturing resources.
Cloud manufacturing ecosystem has more complicated relationship among individuals in it than that in normal manufacturing system, one individual can make decisions depend on the surroundings and others'. In this paper, we designed an original operation mode with 3 extensions named incubation, outsourcing and metabolism for the ecosystem. We designed an agent-based simulation model to validate the effectiveness of these modes. Experiment result shows evolution in ecosystem: with incubation mode, the job queue length is reduced and the resource idle rate is declined; with outsourcing mode, the job queue length is also reduced but not as much as that with incubation mode and more resource is required; with metabolism mode, lower resource is required than that with other modes and resource idle rate just rose up a little.

\end{abstract}

\begin{keyword}
Cloud manufacturing ecosystem \sep
Decision-making\sep
Operation mode\sep
Ecosystem evolution\sep
Agent-based simulation

\end{keyword}
% \belowfrontmatterskip0pt


\end{frontmatter}

%%
%% Start line numbering here if you want
%%
% \linenumbers

%% main text
% !TEX root = ../beamer.tex
\section{Introduction}

\subsection{Cloud Manufacturing}
\begin{frame}{Brief Introduction}{Cloud Manufacturing}
\onslide<+->{Cloud manufacturing technologies including
\begin{itemize}
\item Internet of Things
\item Cloud computing
\item Semantic Web
\item Virtualisation
\item Service-oriented technologies
\end{itemize}
}
\onslide<+->{Aims the manufacturing process to be
\begin{itemize}
\item networked
\item intelligent
\item service-oriented
\item knowledge-based
\item energy efficient
\end{itemize}}
\end{frame}

\begin{frame}{Initial Results}{Cloud Manufacturing}
\only<1>{\begin{figure}
\centering
\includegraphics[width=0.95\textwidth]{figures/mainpage.png}
\caption{\href{http://mie.zju.edu.cn/cloud/}{Portal Web}}
\end{figure}}
\only<2>{
	\begin{block}{Main Component of Cloud Manufacturing prototype system}
		\begin{itemize}
			\item Platform portal system
			\item Order/Task management system
			\item Resource/Service management system
			\item Knowledgement management system
			\item Commercial application sharing system
		\end{itemize}	
	\end{block}
}
\end{frame}

\subsection{Current Items}

\begin{frame}{Current Items}{National High-Tech. R\&D Program, China}
\textbf{Foundation Item}: \textit{Key Technologies Research on Cloud Manufacturing Platform for Group Enterprises (No. 2015AA042101)}
\begin{block}{Main Tasks}
  \begin{itemize}
  \item Business model and standard formulating in group enterprises;
  \item Multi-tenancy technology for on-demand usage;
  \item Platform development for enterprises (Cooperate);
  \item Application demonstration (Cooperate).
  \end{itemize}
\end{block}
\end{frame}
\begin{frame}{Current Items}{National High-Tech. R\&D Program, China}
	\begin{figure}
		\centering
		\includegraphics[height=0.85\textheight]{figures/techrouteorange.pdf}
		\caption{Technology Roadmap}
	\end{figure}
\end{frame}


\begin{frame}{Current Items}{National Natural Science Foundation, China}
\textbf{Foundation Item}: \textit{BOSS based Product Manufacturing-service Modeling and Cloud Platform Guided Service-network Evolving (No. 71571161)}
\begin{block}{Main Tasks}
\begin{itemize}
\item Methodology study for standard Manufacturing-service;
\item Methodology study for Manufacturing-service evalution;
\item Optimize the matching for service and task;
\item Explore the evolution of Manufacturing-service.
\end{itemize}
\end{block}
\end{frame}
\begin{frame}{Current Items}{National Natural Science Foundation, China}
	\begin{figure}
		\centering
		\includegraphics[height=0.85\textheight]{figures/nsf.pdf}
		\caption{Main technical research relationship map}	
	\end{figure}
\end{frame}
% !TEX root = Eco-Model.tex
\section{Background and literatures} % (fold)
\label{sec:literature_review}
\subsection{Background} % (fold)
\label{sub:background}
Resource consumption in manufacturing activities is inevitable, waste and idle of these resources  are pervasive in existing manufacturing systems \cite{Smith2012}. Effective utilization of resources and resource productivity driven manufacturing innovation is one key consideration to handle the environmental burden \cite{Dornfeld2014,Li2012}.


In cloud manufacturing context, provider entity publishes manufacturing resources, demander entity publishes manufacturing tasks, platform arranges these resources properly to perform the tasks.
In scope are the mode to generate manufacturing service, which is an arrangement of resources, and the mode to keep resource in high quality instead of low quality.   

% subsection background (end)

\subsection{Literatures} % (fold)
\label{sub:literatures}

Platform operator can manage manufacturing service that encapsulated distributed manufacturing resources intensively with appropriate business model \cite{Xu2012}, modular approaches and multi-layer architectures are the most common approaches to build a cloud manufacturing platform or system framework \cite{Tao2012,Valilai2013}, Lv used the list of views to depict this multi-layer architecture \cite{LvJuly312012-Aug.22012}.

Servitization is the key philosophy to operate cloud manufacturing \cite{li2010cloud}. A  service can be created statically which comes along with a provider \cite{Tao2012}, or can be created dynamically according to task pattern, such method as `Multi-Composition for Each Task' \cite{Liu2013} that combines incompetent service as a whole. A service can also be created by AI planning-based automatic composition framework \cite{OhJan.-March2008}.

Simulation approach has been widely used in manufacturing systems on operation planning and scheduling, real-time control, operating policies, performance analysis \cite{Smith2003}. In operating policies field, scheduling policies can be tested with simulation performance under given machine conditions \cite{Sabuncuoglu2003}, machine
segmentation policies can be simulated in a combined MRP and Kanban production system \cite{Felberbauer9-12Dec.2012}. Mourtzis et al. \cite{Mourtzis2015} explored a series of simulation-based solutions in industrial practices and concluded that research trends are in  Internet- and cloud-based situations.

% subsection literatures (end)
% section literature_review (end)

% !TEX root = Eco-Model.tex
\section{Problem Background} % (fold)
\label{sec:problem_background}

\subsection{Cloud Manufacturing Ecosystem} % (fold)
\label{sub:cloud_manufacturing_ecosystem}

% subsection cloud_manufacturing_ecosystem (end)

\subsection{Main Copmlexity Concepts} % (fold)
\label{sub:main_copmlexity_concepts}

\subsubsection{Self-organization}

\subsubsection{Emergence}

\subsubsection{Co-evolution}

\subsubsection{Adaption}
% subsection main_copmlexity_concepts (end)

\subsection{Ecosystem Evolvement} % (fold)
\label{sub:ecosystem_evolvement}

% subsection ecosystem_evolvement (end)

\subsection{Optimal Guidance} % (fold)
\label{sub:optimal_guidance}

% subsection optimal_guidance (end)
% section problem_background (end)
% !TEX root = Eco-Model.tex
\section{Design of the ecosystem} % (fold)
\label{sec:design_of_the_ecosystem}
\subsection{Preliminaries} % (fold)
\label{sub:preliminaries}
The manufacturing ecosystem we proposed in this paper consists of the structure of cloud manufacturing, the operation mode and the controlling rules. Before the detailed interpretation for these parts, we describe some basic settings as follows.

\subsubsection{Master plan} % (fold)
\label{ssub:master_plam}
With the cloud manufacturing platform operating, demander comes, signs up and publish orders. Match module in platform informs the task messages in the order to every matched resources and services. After the select decision makes by the demander, the selected resources and services prepare to the performance of the task. 
Providers find frequent task after a bunch of successful accomplishment via resource reviews, then another type of job, service-call generated by them to gather certain type and amount of resources registered in the platform to create service in order to perform the certain task more efficient than before. Most of resource related decision making are well controlled by platform with the review module.

A single order($o_i$) consists of a set $\mathcal{T}_i = \left\{ t_{i1},t_{i2},\dots,t_{in}\right\}$ of tasks which have to be processed. The tasks are interrelated by kinds of constraints. First, precedence constraints force task $t_{ij}$ not to be started before all its immediate predecessors, how to construct the predecessor will be explained later. Second, performing the tasks requires resources with limited capacities. Third, resources cooperation transmits and enhances these constraints.

A single resource($r_k$) belongs to one type in set $\mathcal{A} = \left\{1,2,\dots\,A\right\}$ given by the platform. While being processed via resource cooperation, task $t_{ij}$ requires $q_{k,ij}$ units amount of the resources with every type in $\mathcal{A}_{ij}\subset\mathcal{A}$ during every period of its non-preemptable duration $p_{ij}$. Each resource $r_k$ has a limited capacity of $C_k$ at any point in time.

A single service($s_l$) is a composition of all resources in set $R_l$ with their partial capacities, it can only process certain task with matched resource configuration($\mathcal{A}_{ij}=\mathcal{A}_l$,$Q_{k,ij}=Q_l$) without resource coordinating any more. While being processed via one matched service, task can be suspended then outsourced to other matched service or resource cooperation.
Resource comes along with provider when it enter the ecosystem, however, service doesn't, service is generated after a period of time, the prototype of a service is another type of job, we call it the service-call.

A single service-call($sc_l$) comes out after the finish of one frequent task, so it have the same resource configuration the frequent task. Service-call can only be performed via resource cooperation with duration $p_l$, because the identity of resource, the capacity cannot be return to the resource after the finish of service-call, so the service-call brings new constrains to the related resources. Unlike the task, the amount of one type resource required in the service-call can be pieced together from more than one resource as long as they are in the same type.

Specifically, a simple instance with configuration \autoref{tab:simplejobconfiguration}--\ref{tab:simpleresourceconfiguration} and schedule chart \autoref{fig:scheduleChart} will describe the setting more clearly. 
\begin{table}[htbp]
  \centering
  \scriptsize
  \caption{Simple job configuration}
    \begin{tabular}{cccccc}
    \toprule
    \multicolumn{1}{c}{\multirow{2}[0]{*}{ Job}} & \multicolumn{3}{c}{Need Resource Capacity} & \multicolumn{1}{c}{\multirow{2}[0]{*}{Release Time}} & \multicolumn{1}{c}{\multirow{2}[0]{*}{Duration}} \\
    \multicolumn{1}{c}{} & $Type_1$ & $Type_2$ & $Type_3$ & \multicolumn{1}{c}{} & \multicolumn{1}{c}{} \\
    \midrule
    $t_1$ & 0     & 5     & 7     & 0     & 7 \\
    $t_2$ & 7     & 5     & 7     & 4     & 5 \\
    $t_3$ & 6     & 0     & 5     & 2     & 3 \\
    $sc_1$ & 0     & 0     & 3    & 1     & 1 \\
    $sc_2$ & 2     & 7     & 6     & 3     & 1 \\
    \bottomrule
    \end{tabular}%
  \label{tab:simplejobconfiguration}%
\end{table}%
\begin{table}[htbp]
  \centering
  \scriptsize
  \caption{Simple resource configuration}
    \begin{tabular}{ccc}
    \toprule
    Resource & Type  & Capacity \\
    \midrule
    $r_1$ & 1     & 10 \\
    $r_2$ & 2     & 10 \\
    $r_3$ & 3     & 10 \\
    $r_4$ & 3     & 10 \\
    \bottomrule
    \end{tabular}%
  \label{tab:simpleresourceconfiguration}%
\end{table}%
\begin{figure}[htbp]
	\centering
	\resizebox{.8\textwidth}{!}{% !TEX root = flow_head.tex


\begin{tikzpicture}
\foreach \x in {0,6.5,13,19.5}
	\draw [->,thick] (15.7cm,\x cm) node{t} (0,\x cm) -- +(15.5cm, 0) ;
\foreach \x in {0,6.5,13,19.5}
	\draw [->,thick] (0,\x cm) -- +(0,5.2 cm);
\foreach \x in {0,6.5,13,19.5}
	\draw[yshift=5.35cm] (0,\x cm) node{Capacity};
\foreach \x in {1,...,15}
	\foreach \y in {0,6.5,13,19.5}
		\draw (\x cm,\y cm) -- +(0,2mm);
\foreach \x in {1,...,15}
	\foreach \y in {0,6.5,13,19.5}
		\draw[yshift=-2mm] (\x cm,\y cm) node{$\x$};
\foreach \y in {0,6.5,13,19.5}
	\foreach \x in {1,...,10}
		\draw[yshift=\y cm] (0,0.5*\x cm) -- +(2mm,0) (-3mm,0.5*\x cm) node{$\x$};

\draw[yshift=6.5cm] (7.5cm,-11mm) node{$r_4$, initial capacity = 10, type = 3};
\draw[yshift=13cm] (7.5cm,-11mm) node{$r_3$, initial capacity = 10, type = 3};
\draw[yshift=19.5cm] (7.5cm,-11mm) node{$r_2$, initial capacity = 10, type = 2};
\draw[yshift=26cm] (7.5cm,-11mm) node{$r_1$, initial capacity = 10, type = 1};

\fill[blue!20!,yshift=19.5cm] (6cm,4cm) rectangle (7cm,5cm);
\draw[yshift=19.5cm] (3cm,0) rectangle (6cm,3cm) (6cm,4cm) rectangle (8cm,5cm) (8cm,0) rectangle (13cm,3.5cm) (1cm,0) rectangle (3cm,4.5cm) (3cm,3cm) rectangle (6cm,3.5cm)  (4.5cm,1.6cm) node{$t_3$} (7cm,4.5cm) node{$sc_2$} (10.5cm,1.8cm) node{$t_2$} (2cm,2.25cm) node{$t_4$} (4.5cm,3.25cm) node{$t_5$};
\draw[dotted,yshift=19.5cm] (0,5cm) -- (7cm,5cm) (3cm,4cm) -- (15cm,4cm) (7cm,0) -- (7cm,5cm) (8cm,4cm) -- (8cm,3.5cm);
\draw[yshift=13cm] (0,0) rectangle (7cm,2.5cm) (7cm,4cm) rectangle (8cm,5cm) (8cm,0) rectangle (13cm,2.5cm) (1cm,2.5cm) rectangle (3cm,4.5cm)  (3.5cm,1.35cm) node{$t_1$} (7.5cm,4.5cm) node{$sc_2$} (10.5cm, 1.35cm) node{$t_2$}  (2cm,3.5cm) node{$t_4$};
\draw[dotted,yshift=13cm] (0,5cm) -- (7cm,5cm) (3cm,4cm) -- (15cm,4cm) (7cm,5cm) -- (7cm,0) (8cm,5cm) -- (8cm,0);
\draw[yshift=6.5cm] (0,0) rectangle (7cm,3.5cm) (7cm,2cm) rectangle (8cm,5cm) (3cm,3.5cm) rectangle (6cm,5cm) (3.5cm,1.85cm) node{$t_1$} (7.5cm,3.5cm) node{$sc_2$} (4.5cm,4.25cm) node{$t_5$};
\draw[dotted,yshift=6.5cm] (0,5cm) -- (7cm,5cm) (8cm,2cm) -- (15cm,2cm) (8cm,0) -- (8cm,2cm);
\draw (0,3.5cm) rectangle (1cm,5cm) (3cm,0) rectangle (6cm,2.5cm) (8cm,0) rectangle (13cm,3.5cm) (1cm,0) rectangle (3cm,3cm) (4.5cm, 1.35cm) node {$t_3$} (0.5cm,4.3cm) node{$sc_1$} (10.5cm,1.8cm) node{$t_2$} (2cm,1.5cm) node{$t_4$};
\draw[dotted] (0,3.5cm) -- (15cm,3.5cm) (1cm,0) -- (1cm,3.5cm);
\end{tikzpicture}
}
	\caption{Simple instance schedule chart}
	\label{fig:scheduleChart}
\end{figure}
This instance makes some simplification in the subscript fields in order to emphasize the resource cooperation, all the job(task and service-call) performance can only be started when the each of the related resource is ready, so the shadow in the figure is the waiting period. Horizontal dotted line constrained the available capacity of the resource for the following jobs, every performance of service-call will make the line lower and it will never get higher again unless the related service is repealed.

% subsubsection master_plan (end)

\subsubsection{Assumptions, nomenclature} % (fold)
\label{ssub:assumptions_nomenclature}
job
order
task
service-call
abstract machine
resource
service
demander
provider
platform
rank
hardness

detailed design will be explained in the following sections.
% subsubsection assumptions_nomenclature (end)
% subsection preliminaries (end)

\subsection{Operation mode} % (fold)
\label{sub:operation_mode}

% subsection operation_mode (end)

\subsection{Interactions and decisions} % (fold)
\label{sub:interactions_and_decisions}

% subsection interactions_and_decisions (end)

\subsection{Controlling rules} % (fold)
\label{sub:controlling_rules}

% subsection controlling_rules (end)

\subsection{Tricky problems} % (fold)
\label{sub:tricky_problems}
The job lock
% subsection tricky_problems (end)
% section design_of_the_ecosystem (end)
% !TEX root = Eco-Model.tex
\section{Methodological approach} % (fold)
\label{sec:methodological_approach}

This section starts with a brief overview of ABMS. Then .... Finally ... 
i.e. the swarm model of the system.

In a word, the center of the research is the design of the user agent, and the core of the agent is to make decisions. so the core of the evolution is to make decision? 

core of the paper: agent simulation, decision making.

\subsection{Framework of Agent-based modeling} % (fold)
\label{sub:framework of agent based modeling}
\begin{figure}[htbp]
\centering\small
\subfloat[Order Flow]{\resizebox{0.45\textwidth}{!}{% !TEX root = Eco-Model.tex
\begin{tikzpicture}[%
    >=triangle 60,              % Nice arrows; your taste may be different
    start chain=going below,    % General flow is top-to-bottom
    node distance=6mm and 60mm, % Global setup of box spacing
    every join/.style={norm},   % Default linetype for connecting boxes
    ]
% ------------------------------------------------- 
% A few box styles 
% <on chain> *and* <on grid> reduce the need for manual relative
% positioning of nodes
\tikzset{
  base/.style={draw, on chain, on grid, align=center, minimum height=4ex},
  proc/.style={base, rectangle, text width=8em},
  test/.style={base, diamond, aspect=2, text width=5em},
  term/.style={proc, rounded corners},
  % coord node style is used for placing corners of connecting lines
  coord/.style={coordinate, on chain, on grid, node distance=6mm and 25mm},
  sky/.style={coordinate, on chain, on grid, node distance=6mm and 60mm},
  % nmark node style is used for coordinate debugging marks
  nmark/.style={draw, cyan, circle, font={\sffamily\bfseries}},
  % -------------------------------------------------
  % Connector line styles for different parts of the diagram
  norm/.style={->, draw, lcnorm},
  free/.style={->, draw, lcfree},
  cong/.style={->, draw, lccong},
  it/.style={font={\small\itshape}}
}
% -------------------------------------------------
% Start by placing the nodes
\node [term]              {Be Submitted};
\node [proc, join]        {Constrcut into Tasks};
\node [proc, join]  (p1)  {Evaluate the Seller};
\node [test, join]  (t1)  {Transact?};
\node [test]        (t2)  {Task Remain?};
\node [test]        (t3)  {Outsourcing?};
\node [proc]        (p2)  {Process Task and Submit Result};
\node [test, join]  (t4)  {Order Finished?};
\node [term]        (m1)  {End Smothly};

\node [sky, right=of p1] (c5) {};%\cmark{5}
\node [proc, right=of t1] (p3)  {Select Next Seller};
\node [test]        (t5)  {Partly or Whole?};
\node [proc]        (p5)  {Process Un-Outsouring Part and Submit Result};
\node [test, join]  (t6)  {Process Smothly?};
\node [proc]        (p4)  {Submit Outsourcing Part};
\node [coord, yshift=2.5mm]       (c4)  {}; %\cmark{4}
\node [term, right=of m1] (m2)  {End with Error};

\node [coord, left=of t2] (c1) {}; %\cmark{1}
\node [coord, right=of t3](c2) {}; %\cmark{2}
\node [coord, right=of t4](c3) {}; %\cmark{3}
\node [coord, right=of t5, xshift=3mm](c6) {}; %\cmark{6}
\node [coord, left=of c4, xshift=-1mm](c8) {}; %\cmark{8}

\path (t1.east) to node [near start, yshift=1em] {$n$} (p3);
  \draw [o->, lcnorm] (t1.east) -- (p3);
  \draw [->, lcnorm] (p3) |- (c5);
\path (t1.south) to node [near start, xshift=1em] {$y$} (t2);
  \draw [*->, lcnorm] (t1.south) -- (t2);
\path (t2.west) to node [near start, yshift=1em]  {$n$} (p2.west);
  \draw [o->, lcnorm] (t2.west) -- (c1) |- (p2.west);
\path (t2.south) to node [near start, xshift=1em] {$y$} (t3);
  \draw [*->, lcnorm] (t2.south)  -- (t3);
\path (t3.east) to node [near start, yshift=1em] {$y$}  (c2);
  \draw [*->, lcnorm] (t3.east) -| (c2) |- (t5.west);
\path (t3.south) to node [near start, xshift=1em] {$n$} (p2);
  \draw [o->, lcnorm] (t3.south) -- (p2);
\path (t4.south) to node [near start, xshift=1em] {$y$} (m1);
  \draw [*->, lcnorm] (t4.south) -- (m1);
\path (t4.east) to node [near start, yshift=-1em] {$n$} (c3);
  \draw [o->, lcnorm] (t4.east) -- (c3) |- (c4) -- (m2);
\path (t5.east) to node [near start, yshift=1em] {whole} (c6);
\path (t5.south) to node [near start, xshift=2em] {partly} (p5);
  \draw [o->, lcnorm] (t5.east) -- (c6) |- (p1);
  \draw [*->, lcnorm] (t5.south) -- (p5);
\path (t6.south) to node [near start, xshift=1em] {$y$} (p4);
\path (t6.west) to node [near start, yshift=1em] {$n$} (c8);
  \draw [o->, lcnorm] (t6.west) -| (c8); 
  \draw [*->, lcnorm] (t6.south) -- (p4);
  \draw [->, lcnorm] (p4.east) -| (c6);
\end{tikzpicture}}\label{fig:orderflow}} \hspace{0.09\textwidth}
\subfloat[User Rank]{\resizebox{0.45\textwidth}{!}{% !TEX root = ../Eco-Model.tex
\begin{tikzpicture}[%
    >=triangle 60,              % Nice arrows; your taste may be different
    start chain=going below,    % General flow is top-to-bottom
    node distance=6mm and 60mm, % Global setup of box spacing
    every join/.style={norm},   % Default linetype for connecting boxes
    ]
% ------------------------------------------------- 
% A few box styles 
% <on chain> *and* <on grid> reduce the need for manual relative
% positioning of nodes
\tikzset{
  base/.style={draw, on chain, on grid, align=center, minimum height=4ex},
  proc/.style={base, rectangle, text width=8em},
  test/.style={base, diamond, aspect=2, text width=5em},
  term/.style={proc, rounded corners},
  % coord node style is used for placing corners of connecting lines
  coord/.style={coordinate, on chain, on grid, node distance=6mm and 25mm},
  % nmark node style is used for coordinate debugging marks
  nmark/.style={draw, cyan, circle, font={\sffamily\bfseries}},
  % -------------------------------------------------
  % Connector line styles for different parts of the diagram
  norm/.style={->, draw, lcnorm},
  free/.style={->, draw, lcfree},
  cong/.style={->, draw, lccong},
  it/.style={font={\small\itshape}}
}
% -------------------------------------------------
% Start by placing the nodes
\node [term, fill = lcfree!25]              (m1)  {Be Introduced};
\node [proc, join]        (p8)  {Get Initial Rank $R$};
\node [coord, yshift=2mm] (c1)  {}; %\cmark{1}
\node [test, yshift=2mm]  (t1)  {Transact?};
\node [proc, yshift=-2mm] (p1)  {Increase $R$};

\node [coord, yshift=0mm] (c7)  {}; %\cmark{7}

\node [test, yshift=0mm] (t2)  {All Tasks Done?};
\node [proc]              (p2)  {Process Un-Outsourceing Task};
\node [test]              (t3)  {Accepted?};
\node [proc]              (p3)  {Pass the Outsouring Task};
\node [test, join]        (t4)  {Task Done?};

\node [proc, right=of t1] (p4)  {Wait};
\node [proc]              (p5)  {Compete to Get Task};
\node [test, join]        (t5)  {Make it?};

\node [coord, yshift=2mm] (c4)  {}; %\cmark{4}

\node [test, yshift=2mm]  (t6)  {Outsourcing?};
\node [test]              (t7)  {Partly or Whole?};
\node [proc]              (p6)  {Submit Outsourcing Request};
\node [proc]              (p7)  {Decrease $R$};
\node [test, join]        (t8)  {$R < R_{base}$?};
\node [term, fill=lccong!25]              (m2)  {Eliminate};

\node [coord, left=of p1] (c2)  {}; %\cmark{2}
\node [coord, right=of t5](c3)  {}; %\cmark{3}
\node [coord, right=of p2](c5)  {}; %\cmark{5}
\node [coord, right=of t3](c6)  {}; %\cmark{6}
\node [coord, left=of c7] (c8)  {}; %\cmark{8}
\node [coord, right=of t4](c9)  {}; %\cmark{9}

\draw [->, lcnorm] (p4) |- (c1);
\draw [->, lcnorm] (p8) -- (t1);
\draw [->, lcnorm] (p1) --(c2) |- (c1);
\draw [->, lcnorm] (p2) -- (t2);
\draw [->, lcnorm] (p6) -| (c6) -- (t3);

\path (t1.east)   to node [near start, yshift=1em] {$n$} (p4);
\path (t1.south)  to node [near start, xshift=1em] {$y$} (p1);
  \draw [o->, lcnorm] (t1.east) -- (p4);
  \draw [*->, lcnorm] (t1.south) |- (p5);
\path (t2.east)   to node [near start, yshift=-1em]{$n$} ++(2mm,0);
\path (t2.north)  to node [near start, xshift=1em] {$y$} (p1);
  \draw [o->, lcnorm] (t2.east) -- (c4) ;
  \draw [*->, lcnorm] (t2.north) -- (p1);
\path (t5.east)   to node [near start, yshift=1em] {$n$} (c3);
\path (t5.south)  to node [near start, xshift=1em] {$y$} (t6);
  \draw [o->, lcnorm] (t5.east) -- (c3) |- (p4);
  \draw [*->, lcnorm] (t5.south) -- (t6);
\path (t6.west)   to node [near start, yshift=-1em]{$n$} (c5);
\path (t6.south)  to node [near start, xshift=1em] {$y$} (t7);
  \draw [*->, lcnorm] (t6.south) -- (t7);
  \draw [o->, lcnorm] (t6.west) -| (c5) -- (p2);
\path (t7.west)   to node [near start, yshift=1em] {partly} ++(-1mm,0);
\path (t7.south)  to node [near start, xshift=2em] {whole} (p6);
  \draw [*->, lcnorm] (t7.west) -| (c5);
  \draw [o->, lcnorm] (t7.south) -- (p6);
\path (t8.east)   to node [near start, yshift=1em] {$n$} ++(1mm,0);
\path (t8.south)  to node [near start, xshift=1em] {$y$} (m2);
 \draw [o->, lcnorm] (t8.east) -| (c3);
 \draw [*->, lcnorm] (t8.south) -- (m2);
\path (t3.north)  to node [near start, xshift=1em] {$n$} (p2);
\path (t3.south)  to node [near start, xshift=1em] {$y$} (p3);
  \draw [o->, lcnorm] (t3.north) -- (p2);
  \draw [*->, lcnorm] (t3.south) -- (p3);
\path (t4.west)   to node [near start, yshift=1em] {$y$} ++(-1mm,0);
\path (t4.east)   to node [near start, yshift=1em] {$n$} (c9);
  \draw [*->, lcnorm] (t4.west) -| (c8) -- (c7);
  \draw [o->, lcnorm] (t4.east) -- (c9) |- (p7);
\end{tikzpicture}}\label{fig:userrank}}
% \resizebox{0.9\textwidth}{!}{% !TEX root = ../Eco-Model.tex
\begin{tikzpicture}[%
    >=triangle 60,              % Nice arrows; your taste may be different
    start chain=going right,    % General flow is top-to-bottom
    node distance=6mm and 10mm, % Global setup of box spacing
    every join/.style={norm},   % Default linetype for connecting boxes
    ]
% ------------------------------------------------- 
% A few box styles 
% <on chain> *and* <on grid> reduce the need for manual relative
% positioning of nodes
\tikzset{
  base/.style={draw, on chain, on grid, align=center, rectangle},
  proc/.style={base, text width=6em, minimum height=8em},
  term/.style={base, text width=5em, rounded corners, minimum height = 3ex},
  dproc/.style={proc, minimum height = 20ex},
  gathe/.style={draw, on chain, on grid, align=center, circle, minimum height=4ex, text width=0.5em, node distance=10mm and 8mm},
  % coord node style is used for placing corners of connecting lines
  coord/.style={coordinate, on chain, on grid, node distance=6mm and 25mm},
  sky/.style={coordinate, on chain, on grid, node distance=6mm and 60mm},
  % nmark node style is used for coordinate debugging marks
  nmark/.style={draw, cyan, circle, font={\sffamily\bfseries}},
  % -------------------------------------------------
  % Connector line styles for different parts of the diagram
  norm/.style={->, draw, lcnorm},
  free/.style={->, draw, lcfree},
  cong/.style={->, draw, lccong},
  it/.style={font={\small\itshape}}
}
% -------------------------------------------------
% Start by placing the nodes
\node [term, fill = lcfree!25]              {Order};
\node [proc, join=by free]  (p1) {Task Hub};
\node [proc, dashed] (d) {};
\node [term, xshift = 12mm] (oc) {Order Check};
\node [term, join=by cong, fill = lccong!25]  {Submit};

\node [gathe, right=of d,yshift=5mm, xshift= 15mm] (g1) {};
\node [term, below=of d, yshift=15.5mm] (u1) {User 1};
\node [term, below=of u1] (u2) {User 2};
\node [below=-1mm of u2] (u3) {$\vdots$};
\node [term,below=of u3, yshift=-1mm] (u4) {User n};
\node [below=of u4,yshift=2.6mm] {User Cluster};
\node [gathe, below=of g1] (g2) {};
\node [coord, above=of g1, yshift=8mm] (c1) {};

\draw [->, lccong] (g2) -- (oc.west);
\draw [->, lcnorm] (p1) -- (u1.west);
\draw [->, lcnorm] (p1) -- (u2.west);
\draw [->, lcnorm] (p1) -- (u4.west);

\draw [->, lcnorm] (u1.east) -- (g1.west);
\draw [->, lccong] (u1.east) -- (g2.west);
\draw [->, lcnorm] (u2.east) -- (g1.west);
\draw [->, lccong] (u2.east) -- (g2.west);
\draw [->, lcnorm] (u4.east) -- (g1.west);
\draw [->, lccong] (u4.east) -- (g2.west);
\draw [->, lcnorm] (g1) -- (c1) -- node [black, near end, yshift=0.75em, it]
    {(Outsourcing)} ++(-3.5cm,0) -| (p1) ;

\end{tikzpicture}}
\caption{Simple Flow Chart of Order}
% \label{fig:simpleorderflow}
\end{figure}

\subsubsection{Assumptions and notation}
\begin{itemize}
    \item one provider can publish only one order at one time;
    \item service quality give the initial rank, the product quality depends on the service quality, review depends on the service quality, and the rank depends on the review;
    \item when it comes to rank value, we refer two different rank values: user rank(provider) and service rank, their relationship should be determined.
\end{itemize}


\subsection{Agents in the ecosystem model} % (fold)
\label{sub:agents_in_the_ecosystem_model}
Using UML 
% subsection agents_in_the_ecosystem_model (end)

\subsubsection{Meta agent} % (fold)
\label{ssub:meta_agent}
is a decision maker
% subsubsection meta_agent (end)

\subsubsection{Demander agent} % (fold)
\label{ssub:demander_agent}
After pricing, chose the provider and the amount considering the time limit of the project and the budget.
% subsubsection demander_agent (end)

\subsubsection{Provider agent} % (fold)
\label{ssub:provider_agent}
Pricing function specify two main parts:
unit cost and
provided amount
% subsubsection provider_agent (end)

\subsubsection{Complex user agent} % (fold)
\label{ssub:complex_user_agent}

% subsubsection complex_user_agent (end)

\subsubsection{Platform agent} % (fold)
\label{ssub:platform_agent}
matching the 
% subsubsection platform_agent (end)


\subsection{Ecosystem model} % (fold)
\label{sub:ecosystem model}

% subsection model_realization (end)

\subsubsection{Main modules design}
\label{subsub:main_modules}
inherent relationships, execution sequence/queue.

for service provider to modify their service area, i.e. breadth and depth.

Decision making involves choosing some course of action among various alternatives. For a decision maker, fulfilling the con

\subsubsection{System Model}
\begin{itemize}
    \item Co-evolution
    \item Self-organization
    \item Emergence 
    \item Adaption
\end{itemize}

Assumptions:
\begin{enumerate}
    \item substitute transport with time consume
    \item manufacturing agent randomly show up in the map
\end{enumerate}

Co-evolution consist of:
\begin{itemize}
    \item Scarcity of customers
    \item Conscious choice that enables the organizations to change
    \item Interconnectedness of the enables the organizations to change
    \item Feedback processes that carry the long-term consequences of co-evolution
\end{itemize}

% subsection modeling (end)

\subsection{Simulation design} % (fold)
\label{sub:simulation_design}

% subsection simulation_design (end)

\subsection{Agent-based modeling and simulation} % (fold)
\label{sub:agent_based_modeling_and_simulation}
\subsubsection{Agent design}

\subsubsection{Logical process}

\subsubsection{Interaction among agents}
some mathematical formulation display here to describe the whole ecosystem wide phenomenon, like Little formulation.


% subsection agent_based_model_and_simulation (end)
% section methodological_approach (end)
% !TEX root = Eco-Model.tex
\section{Verification and validation} % (fold)
\label{sec:verification_and_validation}
\subsubsection{Case Design} % (fold)
\label{ssub:case_design}
design the case example, we use the RanGen\cite{Demeulemeester2003,Vanhoucke2008} to generate the orders
% subsubsection case_design (end)

\subsection{Verificaiton} % (fold)
\label{sub:verificaiton}

% subsection verificaiton (end)
\subsection{Validation} % (fold)
\label{sub:validation}

% subsection validation (end)
% section verification_and_validation (end)

% !TEX root = Eco-Model.tex
\section{Numerical Results} % (fold)
\label{sec:numerical_results}

\subsection{effect of 1} % (fold)
\label{sub:effect_of_1}

% subsection effect_of_1 (end)

\subsection{effect of 2} % (fold)
\label{sub:effect_of_2}

% subsection effect_of_2 (end)

% xxx

% section numerical_results (end)
% !TEX root = Eco-Model.tex
\section{Application} % (fold)
\label{sec:application}



% section application (end)
% !TEX root = Eco-Model.tex
\section{Contributions, Limitations, and Future Research} % (fold)
\label{sec:contributions_limitations_and_future_research}


On the other hand, it means, that even if many indicators for the collaboration perspectives are applied it has to be regarded that they will not be able to draw a complete and objective picture of the collaboration performance.
% section contributions_limitations_and_future_research (end)

%% 引言(准备)
%% 文献回顾(包括云制造、生态系统、制造生态系统)
%% 应用背景(包括涉及的云制造模式)
%% 模型建立
%% 模型求解及结果分析(仿真设计、实例验证)
%% 应用向导
%% 展望
%% 参考文献


%% References
%%
%% Following citation commands can be used in the body text:
%% Usage of \cite is as follows:
%%   \cite{key}         ==>>  [#]
%%   \cite[chap. 2]{key} ==>> [#, chap. 2]
%%

%% References with bibTeX database:
\bibliographystyle{../templates/Elsevier/elsarticle-num}
%\bibliographystyle{../templates/Elsevier/elsarticle-harv}
%\bibliographystyle{../templates/Elsevier/elsarticle-num-names}
%\bibliographystyle{../templates/Elsevier/model1a-num-names}
%\bibliographystyle{../templates/Elsevier/model1b-num-names}
%\bibliographystyle{../templates/Elsevier/model1c-num-names}
%\bibliographystyle{../templates/Elsevier/model1-num-names}
%\bibliographystyle{../templates/Elsevier/model2-names}
%\bibliographystyle{../templates/Elsevier/model3a-num-names}
%\bibliographystyle{../templates/Elsevier/model3-num-names}
%\bibliographystyle{../templates/Elsevier/model4-names}
%\bibliographystyle{../templates/Elsevier/model5-names}
%\bibliographystyle{../templates/Elsevier/model6-num-names}

%\nocite{*}
\bibliography{system}
\appendix
\newpage
% !TEX root = Eco-Model.tex
\section{Parameter Setting} % (fold)
\label{sec:parameter_setting}

% section parameter_setting (end)
\end{document}

%%
%% End of file `elsarticle-template-num.tex'.