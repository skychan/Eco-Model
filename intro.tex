% !TEX root = Eco-Model.tex
\section{Introduction}
Manufacturing activities consume kinds of resources (e.g. material, equipment, manpower, nature resource), which will resulting in substantial environmental issues. Properly arrange these resources to collaborate a manufacturing process is one of feasible approaches to reduce the waste during the consumption of resources, the concept of cloud manufacturing \cite{li2010cloud,Xu2012} provides a operating framework to realize the arrangement.

However, the relationship among entities in a cloud manufacturing system become more complicated than that in existing manufacturing systems, since the integration of advanced technologies makes it possible for individual to make decisions depend on enriched information.
In particular, \autoref{fig:originmode} shows the basic procedures that need three main entities to make decisions on. \textbf{Demander} is the entity that publish manufacturing task, it prefers high quality of task performance; \textbf{Provider} is the entity that processes the task with the resources it provided prefer high use rate of resource; \textbf{Platform} it the intermediary entity that promotes the manufacturing activities, it prefers a suitable amount of registered resources to satisfy the needs of the market and an optimal life cycle management of resources. Entities' preferences stimulate the emergence of good resource arrangement pattern that will lead to the emergence operation mode in the cloud manufacturing ecosystem. Hence, it's important to identify a suitable operation mode to meet most of entities' preferences and to optimize the resource management.

In this paper, an original operation mode is designed to describe the basic decision-makings of entity in cloud manufacturing ecosystem, then three extensions, namely metabolism mode, incubation mode and outsourcing mode are proposed. Finally, an experiment to validate these synthetic operation modes are designed using an agent-based simulation method.
