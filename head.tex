% !TEX root = Eco-Model.tex

\title{The Evolution of Manufacturing Ecosystem in Cloud Manufacturing Architecture
%Cloud Manufacturing Ecosystem - Scheduling And Evolving
}

\author[label1]{Author}
\address[label1]{ZJU}
\begin{abstract}
In Cloud Manufacturing, provider publish manufacturing resource as service on the platform for demander to search and purchase it as product. This shopping model makes it possible to centralised manage distributed massive manufacturing resources, which will potentially reduce the idle rate of those resources.
However, in the manufacturing ecosystem constructed with the platform, provider, demander and other components came along with, the heterogeneity of resource's attribute like type, quality, price, processing velocity, leads tough taxonomy of these resources, and the chaotic operating model makes it hard to meet demander's need.
Thus the servitization and optimization of manufacturing resource constitutes a central issue to  in such manufacturing ecosystem.



% 提出了 Cloud Manufacturing Ecosystem 概念,建立相应的生态系统模型,从系统中的各个角色,(需要一个指标)建立系统动力学方程。通过仿真手段验证模型,结合已有的云平台原型系统,给出相关应用。
% 云制造的实施和应用提供了一种全新的制造业生态系统,该生态系统由成员(企业用户、管理员)及环境()组成。为了探究生态系统中,制造业的运作模式(相关决策),优化产业链,形成良性发展的制造生态,通过基于Agent 的建模,来刻画该系统中的个体行为,并通过仿真技术,模拟演化规律。

% 云制造生态系统中,不同的企业具有不同的生态位
% 该系统的输入
% 输出
% 调控方式

% 通过仿真,我们发现。。。

% highlight the novel aspects of the work, indicating the relevance of the paper to the conference topics


% What I do yet :
% Construct the ecosystem frame work
% study the manufacturing cooperation method with each other.
% Design the trading the method
% Simulation (agent-based model) and analysis.




 % Supplier evaluation and selection constitutes a central issue in supply chain management (SCM). How- ever, the data on which to base the corresponding choices in real life problems are often imprecise or vague, which has led to the introduction of fuzzy approaches. Predictive intelligent-based techniques, such as Artificial Neural Network (ANN) and Adaptive Neuro Fuzzy Inference System (ANFIS), have been recently applied in different research fields to model fuzzy multi-criteria decision processes where the understanding and learning of the relationships between the input and output data are the key to select suitable solutions. In this paper, a hybrid ANFIS-ANN model is proposed to assist managers in their sup- plier evaluation process. After aggregating the data set through the Analytical Hierarchy Process (AHP), the most influential criteria on the suppliers’ performance are determined by ANFIS. Then, Multi-Layer Perceptron (MLP) is used to predict and rank the suppliers’ performance based on the most effective criteria. A case study is presented to illustrate the main steps of the model and show its accuracy in prediction. A battery of parametric tests and sensitivity analyses has been implemented to evaluate the overall performance of several models based on different effective criteria combinations.








In this paper, ....

% Basically, (present what have done)

% we do not use any utility function for the modeling; however, a unique method is proposed for eliciting the information from decision makers. The proposed model is applicable for a wide variety of multi-attribute decision making problems and can be used for  future ranking or selection without managers' judgment effort. Simulation of the managers' decisions is demonstrated in detail and the design and implementation of the model are illustrated by case study.


% The research was conducted in three steps: (1) existing concepts and models for industrial sustainability were reviewed and environmental practices in manufacturing were collected and analysed; (2) gaps in knowledge and practice were identified; (3) the outcome is a manufacturing ecosystem model based on industrial ecology (IE)
% 关键是系统动力学模型的建立。


\end{abstract}

\begin{keyword}
%% keywords here, in the form: keyword \sep keyword
new \sep key \sep words
%Cloud Manufacturing  Ecosystem \sep scheduling \sep evolving \sep optimization
%% MSC codes here, in the form: \MSC code \sep code
%% or \MSC[2008] code \sep code (2000 is the default)
\end{keyword}
