% !TEX root = Eco-Model.tex

\title{The Evolution of Manufacturing Ecosystem in Cloud Manufacturing Architecture
%Cloud Manufacturing Ecosystem - Scheduling And Evolving
}

\author[label1]{Shengkai Chen}
% \address[label1]{ZJU}
% \address[label1]{Address One}
\address[label1]{Zhejiang University, Institute of manufacturing engineering and automation, Hangzhou City, Zhejiang Province, China}

% \fntext[label4]{Small city}


\begin{abstract}
In Cloud Manufacturing, users engage in manufacturing business through the well-designed platform framework, where provider publish manufacturing resource as service for demander to search and purchase it as product. This business model allows centralized management of distributed massive manufacturing resources, and could potentially reduce the idle rate of those resources.

However, in the manufacturing ecosystem constructed with the platform, provider, demander and other components came along with, the heterogeneity of resource's attribute like type, quality, price, processing velocity, leads tough taxonomy of these resources, and the chaotic operating model, makes it hard to meet demander's need.
Thus the way that each provider transforms his/her resource into product with servitization and optimization, constitutes a central issue in such manufacturing ecosystem, all these servitization and optimization activities brings the transition to the individual in the ecosystem. As a consequence, the emergence of manufacturing service pattern will be accompanied by these transitions.

In this paper, the author proposed an operation model for users to support their transitions by means of expanding or shrinking their scope of manufacturing service in the manufacturing ecosystem, and the author also proposed an adjustment approach, which depends on the ranking value of user, for the system to restrict the expanding or shrinking procedure and to control the number of users. Gradually this adjustment will guide the evolution of the ecosystem, meanwhile, good service pattern will be discovered by the user to survival. 

With the help of Repast Simphony, the author designed an agent-based model to simulate the evolution process.
As the simulation output, manufacturing resource's quality is improving and its diversity is increasing as the evolution process goes by. The emergence pattern of service and resource optimization implies the effectiveness and validity of the proposed operation model and adjustment approach, user's rank-based transitions does improve the utilization of manufacturing resource.


% highlight the novel aspects of the work, indicating the relevance of the paper to the conference topics


% What I do yet :
% Construct the ecosystem frame work
% study the manufacturing cooperation method with each other.
% Design the trading the method
% Simulation (agent-based model) and analysis.

% Supplier evaluation and selection constitutes a central issue in supply chain management (SCM). How- ever, the data on which to base the corresponding choices in real life problems are often imprecise or vague, which has led to the introduction of fuzzy approaches. Predictive intelligent-based techniques, such as Artificial Neural Network (ANN) and Adaptive Neuro Fuzzy Inference System (ANFIS), have been recently applied in different research fields to model fuzzy multi-criteria decision processes where the understanding and learning of the relationships between the input and output data are the key to select suitable solutions. In this paper, a hybrid ANFIS-ANN model is proposed to assist managers in their sup- plier evaluation process. After aggregating the data set through the Analytical Hierarchy Process (AHP), the most influential criteria on the suppliers�� performance are determined by ANFIS. Then, Multi-Layer Perceptron (MLP) is used to predict and rank the suppliers�� performance based on the most effective criteria. A case study is presented to illustrate the main steps of the model and show its accuracy in prediction. A battery of parametric tests and sensitivity analyses has been implemented to evaluate the overall performance of several models based on different effective criteria combinations.

% Basically, (present what have done)

% we do not use any utility function for the modeling; however, a unique method is proposed for eliciting the information from decision makers. The proposed model is applicable for a wide variety of multi-attribute decision making problems and can be used for  future ranking or selection without managers' judgment effort. Simulation of the managers' decisions is demonstrated in detail and the design and implementation of the model are illustrated by case study.

% The research was conducted in three steps: (1) existing concepts and models for industrial sustainability were reviewed and environmental practices in manufacturing were collected and analysed; (2) gaps in knowledge and practice were identified; (3) the outcome is a manufacturing ecosystem model based on industrial ecology (IE)



\end{abstract}

\begin{keyword}
%% keywords here, in the form: keyword \sep keyword
Cloud manufacturing ecosystem \sep
Resource servitization and optimization\sep
Service pattern recognition\sep
Manufacturing evolution\sep
Agent-based simulation

%Cloud Manufacturing  Ecosystem \sep scheduling \sep evolving \sep optimization
%% MSC codes here, in the form: \MSC code \sep code
%% or \MSC[2008] code \sep code (2000 is the default)
\end{keyword}
