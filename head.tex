% !TEX root = Eco-Model.tex

\title{The Evolution of Manufacturing Ecosystem in Cloud Manufacturing Architecture
%Cloud Manufacturing Ecosystem - Scheduling And Evolving
}

\author[label1]{Shengkai Chen}
% \address[label1]{ZJU}
% \address[label1]{Address One}
\address[label1]{Zhejiang University, Institute of manufacturing engineering and automation, Hangzhou City, Zhejiang Province, China}

% \fntext[label4]{Small city}


\begin{abstract}
In Cloud Manufacturing, users engage in manufacturing business through the well-designed platform framework, where provider publish manufacturing resource as service for demander to search and purchase it as product. This business model allows centralized management of distributed massive manufacturing resources, and could potentially reduce the idle rate of those resources.

However, in the manufacturing ecosystem constructed with the platform, provider, demander and other components came along with, the heterogeneity of resource's attribute like type, quality, price, processing velocity, leads tough taxonomy of these resources, and the chaotic operating model, makes it hard to meet demander's need.
Thus the way that each provider transforms his/her resource into product with servitization and optimization, constitutes a central issue in such manufacturing ecosystem, all these servitization and optimization activities brings the transition to the individual in the ecosystem. As a consequence, the emergence of manufacturing service pattern will be accompanied by these transitions.

In this paper, the author proposed an operation model for users to support their transitions by means of expanding or shrinking their scope of manufacturing service in the manufacturing ecosystem, and the author also proposed an adjustment approach, which depends on the ranking value of user, for the system to restrict the expanding or shrinking procedure and to control the number of users. Gradually this adjustment will guide the evolution of the ecosystem, meanwhile, good service pattern will be discovered by the user to survival. 

With the help of Repast Simphony, the author designed an agent-based model to simulate the evolution process.
As the simulation output, manufacturing resource's quality is improving and its diversity is increasing as the evolution process goes by. The emergence pattern of service and resource optimization implies the effectiveness and validity of the proposed operation model and adjustment approach, user's rank-based transitions does improve the utilization of manufacturing resource.

\end{abstract}

\begin{keyword}
Cloud manufacturing ecosystem \sep
Resource servitization and optimization\sep
Service pattern recognition\sep
Manufacturing evolution\sep
Agent-based simulation

\end{keyword}
