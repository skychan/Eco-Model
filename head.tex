% !TEX root = Eco-Model.tex

\title{Cloud Manufacturing xxx from a ecosystem view
%Cloud Manufacturing Ecosystem - Scheduling And Evolving
}

\author[label1]{Author}
\address[label1]{ZJU}
\begin{abstract}
New abstract
% 提出了 Cloud Manufacturing Ecosystem 概念,建立相应的生态系统模型,从系统中的各个角色,(需要一个指标)建立系统动力学方程。通过仿真手段验证模型,结合已有的云平台原型系统,给出相关应用。

% 关键是系统动力学模型的建立。
% In the Cloud Manufacturing Ecosystem, a benign mode and scheduling methodology for massive services, which were constructed standardly by resources is required, in order for the collaboration within the whole Industry Chain.
%This paper 
%\begin{inparaenum}[1)]
%\item studied the features of the resources and services, then proposed a mode of Cloud Manufacturing;
%\item modeled the services scheduling matter in the chain;
%\item designed evolving modules for the Cloud Manufacturing Ecosystem;
%\item opened some Ecosystem related issues. 
%\end{inparaenum}
%With the help of Data Science tools on the Cloud Platform, resource optimization assessment/schedule methods could be developed, which could
%gradually evolve the ecosystem to an optimal situation.

\end{abstract}

\begin{keyword}
%% keywords here, in the form: keyword \sep keyword
new \sep key \sep words
%Cloud Manufacturing  Ecosystem \sep scheduling \sep evolving \sep optimization
%% MSC codes here, in the form: \MSC code \sep code
%% or \MSC[2008] code \sep code (2000 is the default)
\end{keyword}