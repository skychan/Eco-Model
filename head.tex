% !TEX root = Eco-Model.tex

\title{The Evolution of Manufacturing Ecosystem in Cloud Manufacturing Architecture
%Cloud Manufacturing Ecosystem - Scheduling And Evolving
}

\author[add1]{Shengkai Chen}
\author[add1,add2]{Shuiliang Fang\corref{cor1}}
% \ead{me_fangsl@zju.edu.cn}
% \address[label1]{ZJU}
% \address[label1]{Address One}
\address[add2]{The State Key Laboratory of Fluid Power Transmission and Control, College of Mechanical Engineering, Zhejiang University, Hangzhou, 310027, China}
\address[add1]{Key Laboratory of Advanced Manufacturing Technology of Zhejiang Province, College of Mechanical Engineering, Zhejiang University, Hangzhou , 310027, China}
% \author[a,b,*]{Third Author}

% \address[a]{First affiliation, Address, City and Postcode, Country}
% \address[b]{Second affiliation, Address, City and Postcode, Country}
% \fntext[label4]{Small city}

\cortext[cor1]{Corresponding author. Tel.: +86-135-8880-3980; Email: me\_fangsl@zju.edu.cn}

\begin{abstract}
In cloud manufacturing, individuals engage in manufacturing business through the well-designed platform, where provider provide manufacturing resource for demander to search and purchase it. This business model allows platform operator to manage distributed massive manufacturing resources, which may help provider reduce the idle rate of their resources.
However, in such manufacturing ecosystem, the heterogeneity attribute of resource makes it tough to classify them, and the chaotic operation mode makes it hard to meet demander's need.
Cloud manufacturing ecosystem has more complicated relationship among individuals in it than that in normal manufacturing system, one individual can make decisions depend on the surroundings and others' with the help of integrated advanced technologies. Thus, in this paper, we designed an original operation mode with 3 extensions for the cloud manufacturing ecosystem to help describe some decision makings in individuals and the platform operator.
With the help of Repast Simphony, we designed an agent-based simulation model and an experiment to validate the effectiveness of these operation modes. The experiment result shows: with incubation mode, the job queue length is reduced and the resource idle rate is declined; with outsourcing mode, the job queue length is also reduced but not as much as that with incubation mode and more resource is required; with metabolism mode, lower resource is required than that with other modes and resource idle rate just rose up a little. The combination of incubation and metabolism mode is one ideal mode to maintain the manufacturing ecosystem with.
\end{abstract}

\begin{keyword}
Cloud manufacturing ecosystem \sep
Resource servitization and optimization\sep
Operation mode\sep
Manufacturing evolution\sep
Agent-based simulation

\end{keyword}
% \belowfrontmatterskip0pt