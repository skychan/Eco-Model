% !TEX root = Eco-Model.tex

\title{The Evolution of Manufacturing Ecosystem in Cloud Manufacturing Architecture
%Cloud Manufacturing Ecosystem - Scheduling And Evolving
}

\author[label1]{Author}
\address[label1]{ZJU}
\begin{abstract}
New abstract
% 提出了 Cloud Manufacturing Ecosystem 概念,建立相应的生态系统模型,从系统中的各个角色,(需要一个指标)建立系统动力学方程。通过仿真手段验证模型,结合已有的云平台原型系统,给出相关应用。
% 云制造的实施和应用提供了一种全新的制造业生态系统,该生态系统由成员(企业用户、管理员)及环境()组成。为了探究生态系统中,制造业的运作模式(相关决策),优化产业链,形成良性发展的制造生态,通过基于Agent 的建模,来刻画该系统中的个体行为,并通过仿真技术,模拟演化规律。

% 云制造生态系统中,不同的企业具有不同的生态位
% 该系统的输入
% 输出
% 调控方式

% 通过仿真,我们发现。。。
In this paper, ....

Basically, (present what have done)

we do not use any utility function for the modeling; however, a unique method is proposed for eliciting the information from decision makers. The proposed model is applicable for a wide variety of multi-attribute decision making problems and can be used for  future ranking or selection without managers' judgment effort. Simulation of the managers' decisions is demonstrated in detail and the design and implementation of the model are illustrated by case study.


The research was conducted in three steps: (1) existing concepts and models for industrial sustainability were reviewed and environmental practices in manufacturing were collected and analysed; (2) gaps in knowledge and practice were identified; (3) the outcome is a manufacturing ecosystem model based on industrial ecology (IE)
% 关键是系统动力学模型的建立。


\end{abstract}

\begin{keyword}
%% keywords here, in the form: keyword \sep keyword
new \sep key \sep words
%Cloud Manufacturing  Ecosystem \sep scheduling \sep evolving \sep optimization
%% MSC codes here, in the form: \MSC code \sep code
%% or \MSC[2008] code \sep code (2000 is the default)
\end{keyword}