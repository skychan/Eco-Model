% !TEX root = Eco-Model.tex

\title{Operation Mode Study in Cloud Manufacturing Ecosystem
%Cloud Manufacturing Ecosystem - Scheduling And Evolving
}

\author[add1]{Shengkai Chen}
\author[add2,add1]{Shuiliang Fang\corref{cor1}}
\author[add2,add1]{Tao Peng}
% \ead{me_fangsl@zju.edu.cn}
% \address[label1]{ZJU}
% \address[label1]{Address One}
\address[add2]{The State Key Laboratory of Fluid Power Transmission and Control, College of Mechanical Engineering, Zhejiang University, Hangzhou, 310027, China}
\address[add1]{Key Laboratory of Advanced Manufacturing Technology of Zhejiang Province, College of Mechanical Engineering, Zhejiang University, Hangzhou , 310027, China}
% \author[a,b,*]{Third Author}

% \address[a]{First affiliation, Address, City and Postcode, Country}
% \address[b]{Second affiliation, Address, City and Postcode, Country}
% \fntext[label4]{Small city}

\cortext[cor1]{Corresponding author. Email: me\_fangsl@zju.edu.cn}

\begin{abstract}
In cloud manufacturing, platform operator is able to manage distributed massive manufacturing resources for manufacturing business.
Cloud manufacturing ecosystem has more complicated relationship among individuals than that in existing manufacturing system, every individual makes decisions depend on enriched information. In this paper, an original operation mode with three extensions namely incubation, outsourcing and metabolism mode are proposed. We designed an agent-based simulation model and the validation experiment result shows platform have: 1) shorter job queue length and lower resource idle rate with incubation mode; 2) a little shorter job queue length and fewer amount of registered resource with outsourcing mode; 3) the fewest amount of registered resource but a little higher resource idle rate with metabolism mode.


\end{abstract}

\begin{keyword}
Cloud manufacturing ecosystem \sep
decision-making\sep
operation mode\sep
ecosystem evolution\sep
agent-based simulation

\end{keyword}
% \belowfrontmatterskip0pt
