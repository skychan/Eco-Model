% !TEX root = Eco-Model.tex
\section{Literature review and background} % (fold)
\label{sec:literature_review}
\subsection{Cloud manufacturing and its operation mode} % (fold)
\label{sub:cloud_manufacturing review}
% This definition is quite abstract and thus despite its corretness not very useful.

In cloud manufacturing context, centralized management for manufacturing services that encapsulated with distributed manufacturing resources can be realized by an appropriate business model\cite{Xu2012}, modular approaches and multi-layer architectures are the most common way to build the cloud manufacturing platform or system framework\cite{Tao2012,Valilai2013}, Lv use list of views to depict this multi-layer architecture\cite{LvJuly312012-Aug.22012}. 

Servitization is the key philosophy in the operating of cloud manufacturing\cite{li2010cloud}, service can be created statically that come along with provider\cite{Tao2012} or can be created dynamically according to task pattern, such method as `Multi-Composition for Each Task'\cite{Liu2013} that combine incompetent service into a whole, also service can be created by AI planning-based automatic composition framework\cite{OhJan.-March2008}. 

Simulation approach have been widely use in manufacturing system on operations planning and scheduling, real-time control, operating polities, performance analysis\cite{Smith2003}, and in the operating policies field, scheduling policies can be tested with simulation performance under given machine conditions\cite{Sabuncuoglu2003}, machine
segmentation policies can be simulated in a combined MRP and Kanban production system\cite{Felberbauer9-12Dec.2012}. Mourtzis et al.\cite{Mourtzis2015} explored a series of simulation-based solutions in industrial practice and considered research discuss trends are Internet and cloud based situation.

\subsection{Background} % (fold)
\label{sub:background}

In the next section, we will design the ecosystem from an original operation mode, then we formulize the each of the decision makings and the extension modes. Before the design of the ecosystem, we specify some basic concept as the study background. 
\begin{compactitem}
	\item Provider: the individual in ecosystem who provide resources;
	\item Resource: the basic task process visualized object with renewable capacity unique type;
	\item Task: the basic visualized object need to be processed with certain type and capacity of resources;
	\item Demander: the individual in ecosystem who publish orders that contain bunch of tasks;
	\item Platform: the place where individual interact with others and environment.
\end{compactitem}
% section literature_review (end)
