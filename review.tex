% !TEX root = Eco-Model.tex
\section{Literature review and background} % (fold)
\label{sec:literature_review}
\subsection{Cloud manufacturing and its operation mode} % (fold)
\label{sub:cloud_manufacturing review}
% This definition is quite abstract and thus despite its corretness not very useful.

Review on 1. Cloud manufacturing mode (any) study 
2. res ser combination and decision making
3. rank and review ecosystem (system changing)
4. manufacturing system simulation

In cloud manufacturing context, centralized management for manufacturing services that encapsulated with distributed manufacturing resources can be realized by an appropriate business model\cite{Xu2012}

Currently, the literature review indicates that modular approaches and multi-layer architectures are the most common way to build the cloud manufacturing platform or system framework (Tao, Zhang, et al. 2012;Valilai and Houshmand 2013)

Modular approaches are widely used to decompose a complex system into smaller subsystems according to their functions. For example, Yang and Li (2011) divided a cloud manufacturing services management and control platform into seven functional modules such as system management module, production management module and so on


Most recent studies like Wu et al.\cite{Wu2013} described the roles in Cloud Manufacturing as a tri-group model with:\begin{inparaenum}[1)]
\item Users/Customers,
\item Application providers and
\item Physical resource providers.
\end{inparaenum}


\subsection{Study background} % (fold)
\label{sub:background}

here we goes the terminology that will use in the following subsections within this section. all the related basic terms will in italics fonts.
\begin{itemize}
	\item order
	\item task correspond to capacity
	\item resource
	\item capacity is the function of a resource. like quantity
	\item service is the bundle of capacities. the proportion of the capacities.
	\item 
\end{itemize}

Combined with the background of the research problem and other scholars' previous study in cloud manufacturing, we proposed an improved framework that suits for cloud manufacturing environment



As defined above, Enterprise User is the special case of the User(short for Generalized User) that purely provides manufacturing services and the Customer User is similarly defined.

Hence, there is only two roles in the Ecosystem worth to be studied: User and Administrator, the proportion of user type could be adjusted as the runtime of the system goes by.



% section literature_review (end)
