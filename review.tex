% !TEX root = Eco-Model.tex
\section{Literature review and background} % (fold)
\label{sec:literature_review}
\subsection{Cloud manufacturing and its operation mode} % (fold)
\label{sub:cloud_manufacturing review}
% This definition is quite abstract and thus despite its corretness not very useful.

Review on 1. Cloud manufacturing mode (any) study 
2. res ser combination and decision making
3. rank and review ecosystem (system changing)
4. manufacturing system simulation

In cloud manufacturing context, centralized management for manufacturing services that encapsulated with distributed manufacturing resources can be realized by an appropriate business model\cite{Xu2012}

Currently, the literature review indicates that modular approaches and multi-layer architectures are the most common way to build the cloud manufacturing platform or system framework (\cite{Tao2012}\cite{Valilai2013}




Most recent studies like Wu et al.\cite{Wu2013} described the roles in Cloud Manufacturing as a tri-group model with:\begin{inparaenum}[1)]
\item Users/Customers,
\item Application providers and
\item Physical resource providers.
\end{inparaenum}


\subsection{Study background} % (fold)
\label{sub:background}

With the cloud manufacturing platform operating, demander comes, signs up and publish orders, the structure of the order is depicted by a so-called activity-on-node(AON) network where the nodes represent the tasks and the arcs the precedence relations. Match module in platform informs the task messages to every matched resources and services. After the select decision makes by the demander, the selected resources and services prepare to the performance of the task.
Providers find frequent task after a bunch of successful accomplishment via resource reviews, then another type of job, service-call generated by them to gather certain type and amount of resources registered in the platform to create service in order to perform the certain task more efficient than before. Most of resource related decision making are well controlled by platform with the review module.


here we goes the terminology that will use in the following subsections within this section. all the related basic terms will in italics fonts.
\begin{itemize}
	\item order
	\item task correspond to capacity
	\item resource
	\item capacity is the function of a resource. like quantity
	\item service is the bundle of capacities. the proportion of the capacities.
	\item 
\end{itemize}

Combined with the background of the research problem and other scholars' previous study in cloud manufacturing, we proposed an improved framework that suits for cloud manufacturing environment



As defined above, Enterprise User is the special case of the User(short for Generalized User) that purely provides manufacturing services and the Customer User is similarly defined.

Hence, there is only two roles in the Ecosystem worth to be studied: User and Administrator, the proportion of user type could be adjusted as the runtime of the system goes by.



% section literature_review (end)
