% !TEX root = Eco-Model.tex
\section{Review on cloud manufacturing} % (fold)
\label{sec:literature_review}
(cover a little more)

In cloud manufacturing context, platform operator can manage  manufacturing service that encapsulated with distributed manufacturing resources intensively with appropriate business model \cite{Xu2012}, modular approaches and multi-layer architectures are the most common approaches to build a cloud manufacturing platform or system framework \cite{Tao2012,Valilai2013}, Lv used the list of views to depict this multi-layer architecture \cite{LvJuly312012-Aug.22012}.

Servitization is the key philosophy to operate cloud manufacturing \cite{li2010cloud}. A  service can be created statically which comes along with a provider \cite{Tao2012}, or can be created dynamically according to task pattern, such method as `Multi-Composition for Each Task' \cite{Liu2013} that combines incompetent service as a whole. A service can also be created by AI planning-based automatic composition framework \cite{OhJan.-March2008}.

Simulation approach has been widely used in manufacturing systems on operation planning and scheduling, real-time control, operating policies, performance analysis \cite{Smith2003}. In operating policies field, scheduling policies can be tested with simulation performance under given machine conditions \cite{Sabuncuoglu2003}, machine
segmentation policies can be simulated in a combined MRP and Kanban production system \cite{Felberbauer9-12Dec.2012}. Mourtzis et al. \cite{Mourtzis2015} explored a series of simulation-based solutions in industrial practices and concluded that research trends are in  Internet- and cloud-based situations.

% section literature_review (end)
