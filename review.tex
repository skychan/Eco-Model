% !TEX root = Eco-Model.tex
\section{Review on cloud manufacturing} % (fold)
\label{sec:literature_review}

% This definition is quite abstract and thus despite its corretness not very useful.

In cloud manufacturing context, platform operator can manage  manufacturing service that encapsulated with distributed manufacturing resources intensively with appropriate business model\cite{Xu2012}, modular approaches and multi-layer architectures are the most common approaches to build the cloud manufacturing platform or system framework\cite{Tao2012,Valilai2013}, Lv use list of views to depict this multi-layer architecture\cite{LvJuly312012-Aug.22012}. 

Servitization is the key philosophy to operate cloud manufacturing\cite{li2010cloud}, service can be created statically which come along with provider\cite{Tao2012} or can be created dynamically according to task pattern, such method as `Multi-Composition for Each Task'\cite{Liu2013} that combines incompetent service into a whole, service can also be created by AI planning-based automatic composition framework\cite{OhJan.-March2008}. 

Simulation approach have been widely used in manufacturing system on operations planning and scheduling, real-time control, operating polities, performance analysis\cite{Smith2003}. In operating policies field, scheduling policies can be tested with simulation performance under given machine conditions\cite{Sabuncuoglu2003}, machine
segmentation policies can be simulated in a combined MRP and Kanban production system\cite{Felberbauer9-12Dec.2012}. Mourtzis et al.\cite{Mourtzis2015} explored a series of simulation-based solutions in industrial practice and considered that research trends are Internet and cloud based situation.

% section literature_review (end)
