% !TEX root = Eco-Model.tex
\section{Literature review and the knowledge gap} % (fold)
\label{sec:literature_review}
demonstrates that you know the field, justifies the reason for your research, allows you establish your theoretical and needs to be done.


XXX has been studied extensively in the literature....
\subsection{Cloud manufacturing} % (fold)
\label{sub:cloud_manufacturing}
% This definition is quite abstract and thus despite its corretness not very useful.
from \textbf{function, resource}, information, process view to depict the cloud manufacturing platform.

In cloud manufacturing context, centralized management for manufacturing services that encapsulated with distributed manufacturing resources can be realized by an appropriate business model\cite{Xu2012}

Currently, the literature review indicates that modular approaches and multi-layer architectures are the most common way to build the cloud manufacturing platform or system framework (Tao, Zhang, et al. 2012;Valilai and Houshmand 2013)




\textbf{Services and Tasks}

Cloud Manufacturing Ecosystem provides a platform which distributed manufacturing resources/orders gathered as resources/orders hub. In order to complete orders precisely, order will be splited into tasks. Similarly, resources will be bundled into services for the quick searhing.

\textbf{Roles}

Most recent studies like Wu et al.\cite{Wu2013} described the roles in Cloud Manufacturing as a tri-group model with:\begin{inparaenum}[1)]
\item Users/Customers,
\item Application providers and
\item Physical resource providers.
\end{inparaenum}
However, the roles in real world always not go like this model, since it parted roles only by their functions. While, users could get different function roles depend on what transaction they take, we should make some modifiacations on this model.

Before the emendation of the roles, some related standard name should be taken:
\begin{asparadesc}
\item[User] The main role in Cloud Manufacturing, which is combined of Seller and Buyer as shown in \autoref{tab:categoryuser}, usually acts as Seller or Buyer according to the specific transaction;
\item[Seller] The temporary role in a transaction, which provides the allocated part of manufacturing service and makes a evalution about the Buyer after the transaction;
\item[Buyer] The temporary role in a transaction, which submits the allocated part of manufacturing task and makes a evalution about the Seller after the transaction;
\item[Administrator] The role that gets the work for the operation and maintenance of the platform, has indirectly impact on the Ecosystem.
\end{asparadesc}
\begin{table}[htbp]
  \centering\small
  \caption{Category of Users}
    \begin{tabularx}{\textwidth}{llccX}
    \toprule
        &  & Seller & Buyer & Description \\
    \midrule
    Generalized & User & $\surd$ & $\surd$ & Act as Seller or Buyer according to the specific transaction. \\
    Enterprise & User & $\surd$ & &  Provides the allocated part of manufacturing services and makes a evalution about the Buyer after the transaction. \\
    Customer & User& & $\surd$ &  Submits the allocated part of manufacturing tasks and makes a evalution about the Seller after the transaction.\\
    \bottomrule
    \end{tabularx}%
  \label{tab:categoryuser}%
\end{table}%
As defined above, Enterprise User is the special case of the User(short for Generalized User) that purely provides manufacturing services and the Customer User is similarly defined.

Hence, there is only two roles in the Ecosystem worth to be studied: User and Administrator, the proportion of user type could be adjusted as the runtime of the system goes by.


% subsection cloud_manufacturing (end)

\subsection{Manufacturing ecosystem} % (fold)
\label{sub:manufacturing_ecosystem}
% xx of an ecosystem is a path-dependent, chaotic process, which means that a small difference in starting values can cause great difference to result.

% From a bionomic perspective, organisms and organizations are nodes in networks of relationships. As time passes and evolution proceeds, some nodes are wiped out and new ones crop up, triggering adjustments that ripple across each network. 
\subsubsection{Different ecosystem analogies}

\subsubsection{Manufacturing system}
% Constrained by its key relationships, each organism and each organization is held in its niche, pursuing the same goal -- the genetic or technological information it carries.

% coevolved to form a vast living ecosystem

% 4 different stages for ecosystem:
% birth, expansion, leadership, final(self-renew or death)

% the major difference between ecological and social system is the role of conscious choice.

% Accoring to xxx there are three critical success factors of a business ecosystem.
% First, productivity
% Second, Robustness
% Third, have the ability to create niches and opportunities for new firms.

% Complexity relevant concepts

% Self-organization
% Emergence
% Coevolution
% adaptation

% subsection manufacturing_ecosystem (end)

\subsection{Study of evolution on related fields} % (fold)
\label{sub:complex_system}
\subsubsection{System Dynamic}


\subsubsection{Agent-based Modeling}
% subsection complex_system (end)
% section literature_review (end)