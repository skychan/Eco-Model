% !TEX root = Eco-Model.tex
\section{Conclusions, Limitations, and Future Research} % (fold)
\label{sec:contributions_limitations_and_future_research}
The operation modes proposed in this paper represented ways to the arrangement of resource to handle the environmental burden, and the $RE$ value showed in \autoref{tab:averagevalue}, reduced wast and idle of manufacturing resources have partially achieved, if Mode 11 was selected as the reference standard, since this value means the amount of task that can be manufactured per unit of resource.   

As for extended modes we designed for cloud manufacturing ecosystem, incubation mode can realize a feasible solution for the formation of manufacturing service, shorten the job queue length and reduce the resource idle rate; outsourcing mode can cut down the amount of registered resource; metabolism mode can also cut down the amount of registered resource in price of a little higher resource idle rate. The combine of incubation and metabolism mode turns out to be a ideal maintenance pattern for operating.

However, we only proposed 3 extensions to combine and it may not fully describe the operation mode of cloud manufacturing system, it may also limit the evolution direction of the ecosystem, hence we will design more extensions. The assignment of service-call is oversimplified to prevent complex consequence, we will design new approaches to assign this job.
% section contributions_limitations_and_future_research (end)

\section{Acknowledgments} % (fold)
\label{sec:acknowledgments}
This work is supported by the National Natural Science Foundation of China (No. 71571161), the National High Technology Research and Development Program of China (863 Program) (No. 2015AA042101), and the Fund of Key Lab of AMT of Zhejiang Province (No. 2015QN01). The authors also gratefully acknowledge the insightful comments from anonymous reviewers which have helped to
improve significantly on the earlier version of this paper.
% section acknowledgments (end)
