% !Mode:: "TeX:UTF-8"
\documentclass{article}
\author{Shengkai Chen}
\date{\today}
\title{Response to Reviewers' Comments}
\begin{document}
\maketitle

We would like to thank the Editor and reviewers for their insightful comments and suggestions. The issues raised by the reviewers have been addressed and a detailed response are given below.

\setlength\parindent{0pt}

\section*{Reviewer 1's comment}
I think that the initial background information given in the introduction could be explained a little bit better on simpler terms.

\textbf{Response:}

Thank you for the suggestions. I modified the initial background information in Sec. 1 and added some extra details in Sec 2.1.

\subsection*{Comment 1} % (fold)
The literature review was good but could cover a little more. Additionally, the conclusion section seemed a little bit too short. It was dense but did not cover enough. Results section was good.

\textbf{Response:}
 
Thank you for the suggestions. I extended Sec 2. with more information that contains a bit more literature reviews.

\subsection*{Comment 2} % (fold)
There are several issues of tense I found throughout the paper as well as grammar issues, especially in the terms section.

\textbf{Response:}

I'm so sorry to make such mistakes, related modifications have been made in new manuscript. Thank you.

\subsection*{Comment 3} % (fold)
One issue that I would address is that the figures such as figures one and two are not clear enough.  They need to be more crisp instead of blurry. The plots and graphs seem to be good on the other hand.

\textbf{Response:}

Thank you for the suggestion. I've changed the Fig. 1 into parallel style that enlarges the words in this figure, I've also redrawn Fig. 2 and Fig. 1 so that they are more crisp now. All the figures now are vectorgraphs.

\section*{Review 2's comment} % (fold)  
The authors should get in touch with the study's relevance with the environmental issues. Otherwise, the article's relevance with the conference is regarded to be low.
I would recommend the authors to describe the 3 modes (incubation, outsourcing, and metabolism modes) so that the readers can understand them more easily.

\textbf{Response:}

Thank you for your suggestion. Previous manuscript considered environmental issues in the view of the life cycle vicissitude of resource in cloud manufacturing, so we designed bunches of operation modes to study how to optimize the manufacturing resources for the overall industry. I must admit that these contents are not obvious enough, so apart from changing the representations, I also add a section (2.1 Background) to describe the environmental issues with regard to cloud manufacturing. Additional simpler description of the proposed 3 modes are add to 3.4, 3.5, 3.6 respectively, and I also simplified the description of the original mode in 3.3.
\end{document}