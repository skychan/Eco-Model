% !Mode:: "TeX:UTF-8"
% !TEX program = xelatex
%%%%%%%%%%%%%%%%%%%%%%%%%%%%%%%%%%%%%%%%%%%%%%%%%%%%%%%%%%%%%%%
%
% Welcome to Overleaf --- just edit your LaTeX on the left,
% and we'll compile it for you on the right. If you give
% someone the link to this page, they can edit at the same
% time. See the help menu above for more info. Enjoy!
%
% Note: you can export the pdf to see the result at full
% resolution.
%
%%%%%%%%%%%%%%%%%%%%%%%%%%%%%%%%%%%%%%%%%%%%%%%%%%%%%%%%%%%%%%%
% Flowcharting techniques for easy maintenance
% Author: Brent Longborough
\documentclass[x11names]{article}
\usepackage{tikz}
\usetikzlibrary{mindmap,shapes,arrows,chains,shadows}
%%%<
\usepackage{verbatim}
%%%>
\begin{comment}
:Title: Easy-maintenance flowchart
:Tags: flowcharts
:Author: Brent Longborough
:Slug: flexible-flow-chart

  This TikZ example illustrates a number of techniques for making TikZ
  flowcharts easier to maintain:
    * Use of <on chain> and <on grid> to simplify positioning
    * Use of global <node distance> options to eliminate the need to 
      specify individual inter-node distances
    * Use of <join> to reduce the need for references to node names
    * Use of <join by> styles to tailor specific connectors
    * Use of <coordinate> nodes to provide consistent layout for
      parallel flow lines
    * A method for consistent annotation of decision box exits
    * A technique for marking coordinate nodes (for layout debugging)

    I encourage you to tinker at this file - add intermediate boxes,
    alter the global distance settings, and so on, to see how well (or
    ill!) it adapts.
\end{comment}
\colorlet{lcfree}{Green3}
\colorlet{lcnorm}{Blue3}
\colorlet{lccong}{Red3}
\begin{document}
% =================================================
% Set up a few colours

% -------------------------------------------------
% Set up a new layer for the debugging marks, and make sure it is on
% top
\pgfdeclarelayer{marx}
\pgfsetlayers{main,marx}
% A macro for marking coordinates (specific to the coordinate naming
% scheme used here). Swap the following 2 definitions to deactivate
% marks.
\providecommand{\cmark}[2][]{%
  \begin{pgfonlayer}{marx}
    \node [nmark] at (c#2#1) {#2};
  \end{pgfonlayer}{marx}
  } 
\providecommand{\cmark}[2][]{\relax} 
% -------------------------------------------------
% Start the picture
% !TEX root = ../Eco-Model.tex
\begin{tikzpicture}[%
    >=triangle 60,              % Nice arrows; your taste may be different
    start chain=going right,    % General flow is top-to-bottom
    node distance=6mm and 10mm, % Global setup of box spacing
    every join/.style={norm},   % Default linetype for connecting boxes
    ]
% ------------------------------------------------- 
% A few box styles 
% <on chain> *and* <on grid> reduce the need for manual relative
% positioning of nodes
\tikzset{
  base/.style={draw, on chain, on grid, align=center, rectangle},
  proc/.style={base, text width=6em, minimum height=8em},
  term/.style={base, text width=5em, rounded corners, minimum height = 3ex},
  dproc/.style={proc, minimum height = 20ex},
  gathe/.style={draw, on chain, on grid, align=center, circle, minimum height=4ex, text width=0.5em, node distance=10mm and 8mm},
  % coord node style is used for placing corners of connecting lines
  coord/.style={coordinate, on chain, on grid, node distance=6mm and 25mm},
  sky/.style={coordinate, on chain, on grid, node distance=6mm and 60mm},
  % nmark node style is used for coordinate debugging marks
  nmark/.style={draw, cyan, circle, font={\sffamily\bfseries}},
  % -------------------------------------------------
  % Connector line styles for different parts of the diagram
  norm/.style={->, draw, lcnorm},
  free/.style={->, draw, lcfree},
  cong/.style={->, draw, lccong},
  it/.style={font={\small\itshape}}
}
% -------------------------------------------------
% Start by placing the nodes
\node [term, fill = lcfree!25]              {Order};
\node [proc, join=by free]  (p1) {Task Hub};
\node [proc, dashed] (d) {};
\node [term, xshift = 12mm] (oc) {Order Check};
\node [term, join=by cong, fill = lccong!25]  {Submit};

\node [gathe, right=of d,yshift=5mm, xshift= 15mm] (g1) {};
\node [term, below=of d, yshift=15.5mm] (u1) {User 1};
\node [term, below=of u1] (u2) {User 2};
\node [below=-1mm of u2] (u3) {$\vdots$};
\node [term,below=of u3, yshift=-1mm] (u4) {User n};
\node [below=of u4,yshift=2.6mm] {User Cluster};
\node [gathe, below=of g1] (g2) {};
\node [coord, above=of g1, yshift=8mm] (c1) {};

\draw [->, lccong] (g2) -- (oc.west);
\draw [->, lcnorm] (p1) -- (u1.west);
\draw [->, lcnorm] (p1) -- (u2.west);
\draw [->, lcnorm] (p1) -- (u4.west);

\draw [->, lcnorm] (u1.east) -- (g1.west);
\draw [->, lccong] (u1.east) -- (g2.west);
\draw [->, lcnorm] (u2.east) -- (g1.west);
\draw [->, lccong] (u2.east) -- (g2.west);
\draw [->, lcnorm] (u4.east) -- (g1.west);
\draw [->, lccong] (u4.east) -- (g2.west);
\draw [->, lcnorm] (g1) -- (c1) -- node [black, near end, yshift=0.75em, it]
    {(Outsourcing)} ++(-3.5cm,0) -| (p1) ;

\end{tikzpicture}


% =================================================
\end{document}