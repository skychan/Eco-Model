% !Mode:: "TeX:UTF-8"
%%%%%%%%%%%%%%%%%%%%%%%%%%%%%%%%%%%%%%%%%%%%%%%%%%%%%%%%%%%%%%%
%
% Welcome to Overleaf --- just edit your LaTeX on the left,
% and we'll compile it for you on the right. If you give
% someone the link to this page, they can edit at the same
% time. See the help menu above for more info. Enjoy!
%
% Note: you can export the pdf to see the result at full
% resolution.
%
%%%%%%%%%%%%%%%%%%%%%%%%%%%%%%%%%%%%%%%%%%%%%%%%%%%%%%%%%%%%%%%
% Flowcharting techniques for easy maintenance
% Author: Brent Longborough
\documentclass{article}
\usepackage{tikz}
\usetikzlibrary{arrows,decorations,backgrounds}
% \usetikzlibrary{mindmap,shapes,arrows,chains,shadows,decorations}%,spy}
% decorations.markings
%%%<
% \usetikzlibrary{external}
% \tikzexternalize

% \usepackage{verbatim}
% \usepackage{paralist}
%\usepackage{geometry}
% \usepackage[active,tightpage]{preview}
% \PreviewEnvironment{tikzpicture}
% \setlength{\PreviewBorder}{4bp}
% \setlength{\paperwidth}{400mm}

\usepackage[psfixbb,graphics,tightpage,active]{preview}

\PreviewEnvironment{tikzpicture}
\newlength{\imagewidth}
\newlength{\imagescale}
%%%>
% \begin{comment}
% :Title: Easy-maintenance flowchart
% :Tags: flowcharts
% :Author: Brent Longborough
% :Slug: flexible-flow-chart

%   This TikZ example illustrates a number of techniques for making TikZ
%   flowcharts easier to maintain:
%     * Use of <on chain> and <on grid> to simplify positioning
%     * Use of global <node distance> options to eliminate the need to 
%       specify individual inter-node distances
%     * Use of <join> to reduce the need for references to node names
%     * Use of <join by> styles to tailor specific connectors
%     * Use of <coordinate> nodes to provide consistent layout for
%       parallel flow lines
%     * A method for consistent annotation of decision box exits
%     * A technique for marking coordinate nodes (for layout debugging)

%     I encourage you to tinker at this file - add intermediate boxes,
%     alter the global distance settings, and so on, to see how well (or
%     ill!) it adapts.
% \end{comment}
% \colorlet{lcfree}{Green3}
% \colorlet{lcnorm}{Blue3}
% \colorlet{lccong}{Red3}

% =================================================
% Set up a few colours

% -------------------------------------------------
% Set up a new layer for the debugging marks, and make sure it is on
% top
% \pgfdeclarelayer{marx}
% \pgfsetlayers{main,marx}
% A macro for marking coordinates (specific to the coordinate naming
% scheme used here). Swap the following 2 definitions to deactivate
% marks.
% \providecommand{\cmark}[2][]{%
%   \begin{pgfonlayer}{marx}
%     \node [nmark] at (c#2#1) {#2};
%   \end{pgfonlayer}{marx}
%   } 
% \providecommand{\cmark}[2][]{\relax} 

\usepackage{pgf-umlcd}
\begin{document}


% -------------------------------------------------
% Start the picture
% \begin{tikzpicture}%[show background grid]
% % !TEX root = flow_head.tex
\begin{tikzpicture}[%
    >=triangle 60,              % Nice arrows; your taste may be different
    start chain=going below,    % General flow is top-to-bottom
    node distance=6mm and 60mm, % Global setup of box spacing
    every join/.style={norm},   % Default linetype for connecting boxes
    ]
% ------------------------------------------------- 
% A few box styles 
% <on chain> *and* <on grid> reduce the need for manual relative
% positioning of nodes
\tikzset{
  base/.style={draw, on chain, on grid, align=center, minimum height=4ex},
  proc/.style={base, rectangle, text width=8em},
  test/.style={base, diamond, aspect=2, text width=5em},
  term/.style={proc, rounded corners},
  % coord node style is used for placing corners of connecting lines
  coord/.style={coordinate, on chain, on grid, node distance=6mm and 25mm},
  % nmark node style is used for coordinate debugging marks
  nmark/.style={draw, cyan, circle, font={\sffamily\bfseries}},
  % -------------------------------------------------
  % Connector line styles for different parts of the diagram
  norm/.style={->, draw, lcnorm},
  free/.style={->, draw, lcfree},
  cong/.style={->, draw, lccong},
  it/.style={font={\small\itshape}}
}
% -------------------------------------------------
% Start by placing the nodes
\node [term] {Be Submitted};
\end{tikzpicture}
% % !TEX root = flow_head.tex

\begin{interface}[text width=.45\textwidth]{AssignBehavior}{0,0}
	\operation{+ BufferEnterance(t: Task, m: Machine): boolean}
	\operation{+ Queue(j: Job, m: Machine): void}
	\operation{+ Buff(t: Task, m: Machine): void}
\end{interface}


% % !TEX root = flow_head.tex

\begin{class}[text width=8cm]{CloudPlatform}{0,0}
	\attribute{- agentIDCounter: long}
	\attribute{\# agentID: String}
	\attribute{+ finishCount: int}
	\attribute{+ userList: List}
	\operation{+ CreateProvider(): void}
	\operation{+ CreateDemander(): void}
\end{class}


% % !TEX root = flow_head.tex

\begin{class}[text width=.45\textwidth]{Demander}{2.4,-1.2}
	\inherit{User}
	\attribute{+ orderList: List}
	\attribute{+ taskMap: Map}
	\operation{+ GenerateOrder(orderID: String): void}
	\operation{+ ReadData(orderID: String): void}
\end{class}


% % !TEX root = flow_head.tex

\begin{class}[text width=8cm]{Job}{0,0}
	\attribute{+ allocated: boolean}
	\attribute{+ allocation: Map}
	\attribute{+ candidates: Map}
	\attribute{+ inNeed: boolean}
	\attribute{+ needResourceCapacity: Map}
	\attribute{+ predecessor: Set}
	\attribute{+ prepareStatus: Map}
	\attribute{- processBehavior: ProcessBehavior}
	\attribute{- selectBehavior: SelectBehavior}
	\operation{+ addCandidates(competitor: Machine): void}
	\operation{+ CheckStatus(): void}
	\operation{+ Need(): void}
	\operation{+ Review(m: Machine): void}
	\operation{+ Select(watchedAgent: Job): void}
\end{class}


% % !TEX root = flow_head.tex

\begin{class}[text width=.45\textwidth]{Machine}{0,0}
	\attribute{+ assignBehavior: AssignBehavior}
	\attribute{+ buffer: List}
	\attribute{+ competeList: List}
	\attribute{+ jobList: List}
	\attribute{+ owner: Provider}
	\attribute{+ releaseBehavior: ReleaseBehavior}
	\attribute{+ responseBehavior: ResponseBehavior}
	\attribute{+ responseList: List}
	\attribute{+ type: String}
	\operation{+ Assign(t: Task): void}
	\operation{+ Release(j: Job): void}
	\operation{+ Reply(watchedAgent: Job): void}
	\operation{+ Response(watchedAgent: Task): void}
	\operation{+ Review(j: Job):void}
\end{class}


% % !TEX root = flow_head.tex

\begin{class}[text width=8cm]{Order}{0,0}
	\attribute{+ owner: Demander}
	\attribute{+ taskList: List}
	\attribute{+ type: String}
\end{class}
% % !TEX root = flow_head.tex

\begin{class}[text width=.45\textwidth]{Service}{2.4,-3.66}
	\inherit{Machine}
	\attribute{+ resourceComposition: Map}
	\operation{+ Service()}
	\operation{+ OutSource(): void}
\end{class}
% % !TEX root = flow_head.tex


\begin{tikzpicture}
\foreach \x in {0,6.5,13,19.5}
	\draw [->,thick] (15.7cm,\x cm) node{t} (0,\x cm) -- +(15.5cm, 0) ;
\foreach \x in {0,6.5,13,19.5}
	\draw [->,thick] (0,\x cm) -- +(0,5.2 cm);
\foreach \x in {0,6.5,13,19.5}
	\draw[yshift=5.35cm] (0,\x cm) node{Capacity};
\foreach \x in {1,...,15}
	\foreach \y in {0,6.5,13,19.5}
		\draw (\x cm,\y cm) -- +(0,2mm);
\foreach \x in {1,...,15}
	\foreach \y in {0,6.5,13,19.5}
		\draw[yshift=-2mm] (\x cm,\y cm) node{$\x$};
\foreach \y in {0,6.5,13,19.5}
	\foreach \x in {1,...,10}
		\draw[yshift=\y cm] (0,0.5*\x cm) -- +(2mm,0) (-3mm,0.5*\x cm) node{$\x$};

\draw (7.5cm,-7mm) node{$r_4$, initial capacity = 10, type = 3};
\draw[yshift=6.5cm] (7.5cm,-7mm) node{$r_3$, initial capacity = 10, type = 3};
\draw[yshift=13cm] (7.5cm,-7mm) node{$r_2$, initial capacity = 10, type = 2};
\draw[yshift=19.5cm] (7.5cm,-7mm) node{$r_1$, initial capacity = 10, type = 1};

\fill[blue!20!,yshift=19.5cm] (6cm,4cm) rectangle (7cm,5cm);
\draw[yshift=19.5cm] (3cm,0) rectangle (6cm,3cm) (6cm,4cm) rectangle (8cm,5cm) (8cm,0) rectangle (13cm,3.5cm) (1cm,0) rectangle (3cm,4.5cm) (3cm,3cm) rectangle (6cm,3.5cm)  (4.5cm,1.6cm) node{$t_3$} (7cm,4.5cm) node{$sc_2$} (10.5cm,1.8cm) node{$t_2$} (2cm,2.25cm) node{$t_4$} (4.5cm,3.25cm) node{$t_5$};
\draw[dotted,yshift=19.5cm] (0,5cm) -- (7cm,5cm) (3cm,4cm) -- (15cm,4cm) (7cm,0) -- (7cm,5cm) (8cm,4cm) -- (8cm,3.5cm);
\draw[yshift=13cm] (0,0) rectangle (7cm,2.5cm) (7cm,4cm) rectangle (8cm,5cm) (8cm,0) rectangle (13cm,2.5cm) (1cm,2.5cm) rectangle (3cm,3.5cm)  (3.5cm,1.35cm) node{$t_1$} (7.5cm,4.5cm) node{$sc_2$} (10.5cm, 1.35cm) node{$t_2$}  (2cm,3cm) node{$t_4$};
\draw[dotted,yshift=13cm] (0,5cm) -- (7cm,5cm) (0,4cm) -- (15cm,4cm) (7cm,5cm) -- (7cm,0) (8cm,5cm) -- (8cm,0);
\draw[yshift=6.5cm] (0,0) rectangle (7cm,3.5cm) (7cm,2cm) rectangle (8cm,5cm) (3cm,3.5cm) rectangle (6cm,5cm) (3.5cm,1.85cm) node{$t_1$} (7.5cm,3.5cm) node{$sc_2$} (4.5cm,4.25cm) node{$t_5$};
\draw[dotted,yshift=6.5cm] (0,5cm) -- (7cm,5cm) (8cm,2cm) -- (15cm,2cm) (8cm,0) -- (8cm,2cm);
\draw (0,3.5cm) rectangle (1cm,5cm) (3cm,0) rectangle (6cm,2.5cm) (8cm,0) rectangle (13cm,3.5cm) (1cm,0) rectangle (3cm,3cm) (4.5cm, 1.35cm) node {$t_3$} (0.5cm,4.3cm) node{$sc_1$} (10.5cm,1.8cm) node{$t_2$} (2cm,1.5cm) node{$t_4$};
\draw[dotted] (0,3.5cm) -- (15cm,3.5cm) (1cm,0) -- (1cm,3.5cm);
\end{tikzpicture}

% % !TEX root = flow_head.tex
\begin{tikzpicture}

\foreach \x in {2,6,9}
	\fill[blue!20!] (\x cm, 0) rectangle +(1cm, 1cm);
\foreach \x in {2,6,9}
	\draw[xshift=.5cm] (\x cm,0.5cm) node{sc};	
\foreach \x in {0,...,10}
	\draw [thick] (\x cm, 0) rectangle +(1cm, 1cm);
\foreach \x in {0,1,3,4,5,7,8,10}
	\draw[xshift=0.5cm] (\x cm, 0.5cm) node{t};

\end{tikzpicture}

% % !TEX root = flow_head.tex
\begin{tikzpicture}

\fill[blue!20!,yshift=1.5cm] (3cm,0) rectangle (4cm,1cm);
\fill[blue!20!] (2cm,0) rectangle (3cm,1cm);
\draw[thick,yshift=1.5cm] (0,0) rectangle (4cm,1cm) (6.5cm,0) rectangle (11.5cm,1cm) (3cm,0) rectangle (4cm,1cm);
\draw[yshift=2cm] (-5mm,0) node{$mr_1$} (1.5cm,0) node{$t_{i1}$} (9cm,0) node{$t_{i2}$} (6cm,0) node{$\cdots$} (12cm,0) node{$\cdots$};
\draw[thick,dotted,xshift=5cm] (0,-5mm) -- (0,3.5cm);
\draw[thick] (0,0) rectangle (3cm,1cm) (6.5cm,0) rectangle (10.5cm,1cm) (2cm,0) rectangle (3cm,1cm);
\draw[yshift=0.5cm] (-5mm,0) node{$mr_2$}  (1cm,0) node{$t_{i2}$} (8.5cm,0) node{$t_{i1}$} (6cm,0) node{$\cdots$} (11cm,0) node{$\cdots$};
\draw[yshift=3cm] (2cm,0) node{active} (9.5cm,0) node{inactive};

\end{tikzpicture}

% % !TEX root = flow_head.tex
\begin{tikzpicture}[node distance=5mm and 5mm,
square/.style={
% The shape:
rectangle,
draw=black,
minimum size=2.9em,
text width=5em,
text centered
},
circle/.style={
rectangle,minimum size=1em,rounded corners=0.5em,
draw=black
}
]
\matrix[row sep=0.5em,column sep=2em] {
% First row:
\node (order) {Demander};& \node (task) [square]{Publish Orders}; &  & \node (select) [square]{Selection}; & & &  \node (finish) {Product}; \\
\node (start1) {}; & & & & & & \node (stop1) {}; \\
\node (plt) {Platform} ; & &  \node (tb) [square] {Task Hub} ; & \node (candidate) [square] {Resource Candidates} ; & \node (coop) [square] {Cooperate Resources}; & \node (combine) [square] {Task Assembly};\\
\node (start2) {}; & & & & & & \node (stop2) {}; & \\
\node (provider) {Provider};  & \node (resource) [square]{Provide Resources}; & \node (response) [square] {Response for Task}; & \node (add) [square]{Add to Candidates}; & \node (match) [square]{Assign Task}; & \node (process) [square] {Processing Task-part};  &\\
};
\draw[dotted,thick] (start1.west) -- (stop1.east) (start2.west) -- (stop2.east);
\path[->,dashed] (task.east) edge (tb.north) (tb) edge (response) (add) edge (candidate) (candidate) edge (select) (match) edge (coop) (coop.east) edge (process.west);
\path[->] (response) edge (add) (select.east) edge (match.west) (process) edge (combine) (combine.north) edge (finish.west);
% \path (order) edge[->] (task) (task) edge[->] (match) (match) edge[->] (coord) (coord) edge[->] (process) (process) edge[->] (finish) (provider) edge[->] (resource) (resource) edge[->] (match);
\end{tikzpicture}
% % !TEX root = flow_head.tex
\begin{tikzpicture}[node distance=5mm and 5mm,
square/.style={
% The shape:
rectangle,
draw=black,
minimum size=2.9em,
text width=5em,
text centered
},
coord/.style={
coordinate,
% on chain,
% on grid,
% node distance=6mm and 25mm
},
circle/.style={
rectangle,minimum size=2em,rounded corners=1em,
draw=black
},
skip loop/.style={to path={-- ++(0,#1) -| (\tikztotarget)}}
]
\matrix[row sep=0.5em,column sep=2em] {
% First row:
\node (add) {New Machine}; & \node (machinein) [square] {Arrival Restrictor}; & & & & & & &\\
\node (origin) {Origin Scarcity}; & & \node (compare) [circle] {\Large$\times$}; & & \node (scarcity) [square] {Machine Scarcity}; & \node (metabolism) [square] {Metabolism Module}; &\node (node) [coord] {}; &\node (end) [coord] {};\\
\node (task) {New Task}; & \node (p1) [coord] {};& & \node (p2) [coord] {}; & \node (reduce) [square] {Eliminate Machine}; & & & & \\
& & \node (tf) {Task Finish}; &  & & & & &\\
};
\path (add) edge[->] (machinein) (machinein.east) edge[->] (compare) (compare) edge[->] (scarcity) (scarcity) edge[->] (metabolism);
\path (origin) edge[->] (compare);
\path (tf) edge[->] (compare);
\path (reduce) edge (p2) (p2) edge[->] (compare);
\draw [->] (metabolism) -- (end) ;
\draw [->]  (task) -- (p1) -- (compare);
\path (node) edge[->,skip loop=5.5em] (machinein);
\draw [->] (node) |- (reduce);
\path (machinein.east) to node [near end,yshift=0.5em,xshift=0.5em] {$-$} (compare);
\path (tf) to node [near end,xshift=0.5em] {$-$} (compare);
\path (p1) to node [near end,xshift=0.8em] {$+$} (compare);
\path (p2) to node [near end,xshift=0.5em,yshift=0.3em] {$+$} (compare);
\path (node) to node [near end,xshift=2em,yshift=0.6em] {Decision value} (end);
\end{tikzpicture}
% % !TEX root = flow_head.tex
\begin{tikzpicture}[node distance=5mm and 5mm,
square/.style={
% The shape:
rectangle,
draw=black,
minimum size=2.9em,
text width=5em,
text centered
},
coord/.style={
coordinate,
% on chain,
% on grid,
% node distance=6mm and 25mm
},
circle/.style={
rectangle,minimum size=2em,rounded corners=1em,
draw=black
},
skip loop/.style={to path={-- ++(0,#1) -| (\tikztotarget)}}
]
\matrix[row sep=0.5em,column sep=2em] {
% First row:
\node (origin) {Origin Frequency} ; & \node (compare) [circle] {\Large$\times$}; & & \node (frequency) [square] {Task Frequency} ; & \node (incubate) [square] {Incubation Module} ; & \node (node) [coord] {} ; & \node (end) [coord] {}; \\
& & \node (p1) [coord] {} ; & \node	(sc) [square] {Incubate Service} ; & & & \\
& \node (tf) {Task Finish}; & & & & &\\
};
\path (origin) edge[->] (compare) (tf) edge[->] (compare) (compare) edge[->] (frequency) (frequency) edge[->] (incubate) (incubate) edge[->] (end);
\path (sc) edge (p1) (p1) edge[->] (compare);
\draw [->] (node) |- (sc) ;
\path (tf) to node [near end,xshift=0.3em] {$+$} (compare) (p1) to node [near end,xshift=0.5em] {$-$} (compare) (incubate) to node [yshift=0.7em,xshift = 1.5em] {Decision value} (end);
\end{tikzpicture}
% % !TEX root = flow_head.tex
\begin{tikzpicture}[node distance=5mm and 5mm,
square/.style={
% The shape:
rectangle,
draw=black,
minimum size=2.9em,
text width=5em,
text centered
},
coord/.style={
coordinate,
% on chain,
% on grid,
% node distance=6mm and 25mm
},
circle/.style={
rectangle,minimum size=2em,rounded corners=1em,
draw=black
},
skip loop/.style={to path={-- ++(0,#1) -| (\tikztotarget)}}
]
\matrix[row sep=0.5em,column sep=2em] {
% First row:
\node (origin) {Origin Length} ; & \node (compare) [circle] {\Large$\times$}; & & \node (length) [square] {Queue Length} ; & \node (outsource) [square] {Outsourcing Module} ; & \node (node) [coord] {} ; & \node (end) [coord] {}; \\
& & \node (p1) [coord] {} ; & \node	(out) [square] {Outsource Task} ; & & & \\
& \node (tf) {Task Finish}; & & & & &\\
};
\path (origin) edge[->] (compare) (tf) edge[->] (compare) (compare) edge[->] (length) (length) edge[->] (outsource) (outsource) edge[->] (end);
\path (out) edge (p1) (p1) edge[->] (compare);
\draw [->] (node) |- (out) ;
\path (tf) to node [near end,xshift=0.3em] {$+$} (compare) (p1) to node [near end,xshift=0.5em] {$-$} (compare) (outsource) to node [yshift=0.7em,xshift = 1.5em] {Decision value} (end);
\end{tikzpicture}
% \end{tikzpicture}
%% !TEX root = flow_head.tex
\begin{tikzpicture}[
 >=triangle 60,              % Nice arrows; your taste may be different
 start chain=going right,    % General flow is top-to-bottom
 node distance=12em and 3em, % Global setup of box spacing
 every join/.style={norm},   % Default linetype for connecting boxes
]
% ------------------------------------------------- 
% A few box styles 
% <on chain> *and* <on grid> reduce the need for manual relative
% positioning of nodes
\tikzset{
  every node/.style={rectangle split, rectangle split parts=2, rectangle split horizontal, rectangle split part fill={lccong!25, yellow!10}, draw, anchor=center, minimum height=9em, on chain, on grid, text width=2em},
  base/.style={draw, on chain, on grid, align=center, minimum height=4ex},
  proc/.style={base, rectangle, text width=8em},
  Arrow/.style={thick, decoration={markings,mark=at position
   1 with {\arrow[semithick]{open triangle 60}}},
   double distance=1.4pt, shorten >= 5.5pt,
   preaction = {decorate},
   postaction = {draw,line width=1.4pt, white,shorten >= 4.5pt},fill = lcnorm!25},
  test/.style={base, diamond, aspect=2, text width=5em},
  term/.style={proc, rounded corners},
  % coord node style is used for placing corners of connecting lines
  coord/.style={coordinate, on chain, on grid, node distance=6mm and 25mm},
  % nmark node style is used for coordinate debugging marks
  nmark/.style={draw, cyan, circle, font={\sffamily\bfseries}},
  % -------------------------------------------------
  % Connector line styles for different parts of the diagram
  norm/.style={->, draw, lcnorm, thick},
  free/.style={->, draw, lcfree},
  cong/.style={->, draw, lccong},
  it/.style={font={\small\itshape}},
  tw/.style={text width=9.5em},
  th/.style={text width=2em}
}

\newcommand{\one}[1]{\nodepart[th]{one}\rotatebox{90}{\bf\shortstack{#1}}}
\newcommand{\two}{\nodepart[tw]{two}}

\node (n1)
{	
	\one{Submit\\ Order}
	\two
	\begin{inparaenum}
	\item 1.Original\\
	\item 2.Outsourcing
	\end{inparaenum}
};

\node[join] (n2) 
{
	\one{Task\\ Construct}
	\two
	\begin{inparaenum}
	\item 1.Order Decomposition\\
	\item 2.Task Merging\\
	\item 3.Task Splitting
	\end{inparaenum}
};

\node[join] (n3)
{
	\one{Service\\ Matching}
	\two
	\begin{inparaenum}
	\item 1.Recommandation\\
	\item 2.Service Selection\\
	\item 3.Seller's Rank
	\end{inparaenum}
};

\node[below=of n3,join] (n4)
{
	\one{Confirm\\ by Seller}
	\two
	\begin{inparaenum}
	\item 1.Accept Tasks\\
	\item 2.Check Due-date\\
	\item 3.Buyer Rank
	\end{inparaenum}
};

\node[below=of n2,join] (n5)
{
	\one{Wait for\\ Delivery}
	\two
	\begin{inparaenum}
	\item 1.Wait the Seller\\
	\item 2.Confirm the Prouct\\
	\item 3.Pay for it
	\end{inparaenum}
};

\node[below=of n1,join] (n6)
{
	\one{Mutual\\ Rating}
	\two
	\begin{inparaenum}
	\item 1.Rate Seller\\
	\item 2.Change Rank\\
	\item 3.Affect Recommand
	\end{inparaenum}
};
\end{tikzpicture}
% !TEX root = flow_head.tex


\begin{tikzpicture}
\fill[blue!20!] (10mm,0) rectangle (20mm,4mm);
\draw (0,0) rectangle (20mm,4mm) (24mm,2mm) node{$mr_2$};
\draw (5mm,2mm) node{5} (15mm,2mm) node{5} (-1cm,2mm) node{$\alpha=2$} (12mm,2mm);
\draw[dotted] (10mm,0) -- (10mm,4mm);

\draw[->] (15mm,0) edge (15mm,-4mm) (17mm,-12mm) edge (23mm,-8mm) (46mm,-12mm) edge (30mm,-8mm);
% \draw[->,xshift=30mm](41mm,-12mm) edge (20mm,-8mm);

\draw[dotted,yshift=-2mm] (-1.5cm,0) -- (5.5cm,0);

\draw[yshift=-8mm,xshift=10mm] (0,0) rectangle (24mm,4mm) (34mm,2mm) node{$ms_l$ (new)};
\draw[yshift=-8mm,xshift=10mm] (10mm,0) -- (10mm,4mm);
\draw[yshift=-8mm,dotted,xshift=10mm] (16mm,0) -- (16mm,4mm);
\draw[yshift=-8mm,xshift=10mm] (5mm,2mm) node{5} (13mm,2mm) node{3} (20mm,2mm) node{4} (-2cm,2mm) node{Service};

\draw[dotted,yshift=-10mm] (-1.5cm,0) -- (5.5cm,0);

\fill[blue!20!,yshift=-16mm] (14mm,0) rectangle (20mm,4mm);
\fill[blue!20!,yshift=-16mm,xshift=30mm](20mm,4mm) rectangle +(-8mm,-4mm);
\draw[yshift=-16mm] (0,0) rectangle (20mm,4mm) (24mm,2mm) node{$mr_4$};
\draw[yshift=-16mm,xshift=30mm] (0,0) rectangle (20mm,4mm) (24mm,2mm) node{$mr_3$};
\draw[yshift=-16mm,dotted] (14mm,0) -- +(0,4mm) ;
\draw[yshift=-16mm,dotted,xshift=30mm] (12mm,0) -- +(0,4mm);
\draw[yshift=-16mm] (7mm,2mm) node{7} (17mm,2mm) node{3} (-1cm,2mm) node{$\alpha=3$};
\draw[yshift=-16mm,xshift=30mm] (6mm,2mm) node{6} (16mm,2mm) node{4};
\end{tikzpicture}

% % !TEX root = flow_head.tex
\begin{tikzpicture}[node distance=5mm and 5mm,
square/.style={
% The shape:
rectangle,
draw=black,
minimum size=2.9em,
text width=5em,
text centered
},
coord/.style={
coordinate,
% on chain,
% on grid,
% node distance=6mm and 25mm
},
circle/.style={
rectangle,minimum size=2em,rounded corners=1em,
draw=black
},
skip loop/.style={to path={-- ++(0,#1) -| (\tikztotarget)}}
]
\matrix[row sep=0.5em,column sep=2em] {
% First row:
\node (s1) [coord] {}; & & & & & \node (s2) [coord]{} ; & & & & & & \\
& & &   \node   (out) [square] {Outsource Task} ; &  \node (originl) {Origin Length} ; &  &\node (origin) {Origin Frequency} ; & \node (compare) [circle] {\Large$\times$}; &  \node (frequency) [square] {Task Frequency} ; & \node (incubate) [square] {Incubation Module} ; & \node (node) [coord] {} ; & \node (end) [coord] {}; \\
\node (end1)  {}; & \node (node1) [coord] {} ; & \node (outsource) [square] {Outsourcing Module} ; &\node (length) [square] {Queue Length} ; & \node (compare1) [circle] {\Large$\times$};  & & & & \node (sc) [square] {Incubate Service} ; & & & \\
\\
&& \node (o1){};& &\node (o2){}; \\
\node (s3) [coord]{}; &  & & & & \node (s4) [coord]{}; &  & & & & &\\
\node (tfs) {};& & \node (metam1) {};& &\node (metam2) {};& &  \node (tf) {Task Finish}; & &\node (incm1){}; & \node (incm2){};& &\node (tfe) {};\\
&&&&& & & \node (machinein) [square] {Arrival Restrictor}; & \node (add) {};  &  & &\\
&&&&&&&&&&&\\
&&&&&&&&&&&\\
& & && &\node (origin2) {};  & \node (compare2) [circle] {\Large$\times$}; &  \node (scarcity) [square] {Machine Scarcity}; & \node (metabolism) [square] {Metabolism Module}; &\node (node2) [coord] {}; &\node (end2) [coord] {};\\
&&&&&\node (task) {}; &  & \node (reduce) [square] {Eliminate Machine}; &  & & &\\
&&&&& &  &  & & & & &&  \\
};
\path (origin) edge[->] (compare) (tf) edge[->] (compare) (compare) edge[->] (frequency) (frequency) edge[->] (incubate) (incubate) edge[->] (end);
\path (sc.west) edge[->] (compare);
\draw [->] (node) |- (sc) ;
\path (tf) to node [near end,xshift=2em,yshift=1em] {$+$} (compare) (sc) to node [near end,xshift=0.5em] {$-$} (compare) (incubate) to node [yshift=0.7em,xshift = 1.5em] {Decision var.} (end);
\draw[dashed] (s1) -- (s2) -- (s4) -- (s3) -- (s1);
\draw[dashed,thick](tfs.west) -- (tf.west) (tf.east) -- (tfe.east);

\path (originl) edge[->] (compare1) (compare1) edge[->] (length) (length) edge[->] (outsource) (outsource) edge[->] (end1);
\path (out.east) edge[->] (compare1);
\draw [->] (node1) |- (out) ;
\path (out) to node [near end,xshift=0.5em,yshift=0.7em] {$-$} (compare1) (outsource) to node [near end,yshift=-0.7em,xshift = -0.5em] {Decision var.} (end1);

\draw[->] (tf) edge (compare1) (tf) edge (compare2);
\path (tf) to node [yshift=0.9em,xshift=-0.8em,near end] {$+$} (compare1) (tf) to node [xshift=-0.8em,yshift=-0.5em,near end] {$-$} (compare2);

\draw (sc.south) edge[->] (machinein.north);

\path (add) edge[->] (machinein) (machinein.west) edge[->] (compare2) (compare2) edge[->] (scarcity) (scarcity) edge[->] (metabolism);
\path (origin2) edge[->] (compare2);
% \path (tf) edge[->] (compare2);
\path (reduce.west)  edge[->] (compare2);
\draw [->] (metabolism) -- (end2) ;
\draw [->]  (task) -- (compare2);
\path (node2) edge[->,skip loop=2.2em] (machinein);
\draw [->] (node2) |- (reduce);
\path (machinein.west) to node [near end,yshift=-0.5em,xshift=0.5em] {$-$} (compare2);
% \path (tf) to node [near end,xshift=0.5em] {$-$} (compare2);
\path (task) to node [near end,xshift=0.8em] {$+$} (compare2);
\path (reduce) to node [near end,xshift=-0.6em,yshift=-0.8em] {$+$} (compare2);
\path (node2) to node [near end,xshift=2em,yshift=0.6em] {Decision var.} (end2);
\path (add) to node [near start, xshift=7em] {New Registered Resource}(machinein) (origin2) to node [near start,xshift=-4.8em]{Origin Scarcity}(compare2) (task) to node [near start,xshift=-3.8em,yshift=-0.5em]{New Task} (compare2);

\path (o1) to node {\textbf{Outsource mode} (only with service)} (o2);
\path (incm1) to node [yshift=1em]{\textbf{Incubation mode}} (incm2) (incm1) to node [yshift=-1em]{\textbf{Metabolism mode}} (incm2);
\end{tikzpicture}
% =================================================
\end{document}