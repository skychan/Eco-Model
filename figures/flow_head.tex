% !Mode:: "TeX:UTF-8"
% !TEX program = xelatex
%%%%%%%%%%%%%%%%%%%%%%%%%%%%%%%%%%%%%%%%%%%%%%%%%%%%%%%%%%%%%%%
%
% Welcome to Overleaf --- just edit your LaTeX on the left,
% and we'll compile it for you on the right. If you give
% someone the link to this page, they can edit at the same
% time. See the help menu above for more info. Enjoy!
%
% Note: you can export the pdf to see the result at full
% resolution.
%
%%%%%%%%%%%%%%%%%%%%%%%%%%%%%%%%%%%%%%%%%%%%%%%%%%%%%%%%%%%%%%%
% Flowcharting techniques for easy maintenance
% Author: Brent Longborough
\documentclass[x11names]{article}
\usepackage{tikz}
\usetikzlibrary{mindmap,shapes,arrows,chains,shadows,decorations.markings,spy}
%%%<
\usepackage{verbatim}
\usepackage{paralist}
%\usepackage{geometry}
\usepackage[active,tightpage]{preview}
\PreviewEnvironment{tikzpicture}
\setlength{\PreviewBorder}{4bp}
%\setlength{\paperwidth}{400mm}
%%%>
\begin{comment}
:Title: Easy-maintenance flowchart
:Tags: flowcharts
:Author: Brent Longborough
:Slug: flexible-flow-chart

  This TikZ example illustrates a number of techniques for making TikZ
  flowcharts easier to maintain:
    * Use of <on chain> and <on grid> to simplify positioning
    * Use of global <node distance> options to eliminate the need to 
      specify individual inter-node distances
    * Use of <join> to reduce the need for references to node names
    * Use of <join by> styles to tailor specific connectors
    * Use of <coordinate> nodes to provide consistent layout for
      parallel flow lines
    * A method for consistent annotation of decision box exits
    * A technique for marking coordinate nodes (for layout debugging)

    I encourage you to tinker at this file - add intermediate boxes,
    alter the global distance settings, and so on, to see how well (or
    ill!) it adapts.
\end{comment}
\colorlet{lcfree}{Green3}
\colorlet{lcnorm}{Blue3}
\colorlet{lccong}{Red3}
\begin{document}
% =================================================
% Set up a few colours

% -------------------------------------------------
% Set up a new layer for the debugging marks, and make sure it is on
% top
\pgfdeclarelayer{marx}
\pgfsetlayers{main,marx}
% A macro for marking coordinates (specific to the coordinate naming
% scheme used here). Swap the following 2 definitions to deactivate
% marks.
\providecommand{\cmark}[2][]{%
  \begin{pgfonlayer}{marx}
    \node [nmark] at (c#2#1) {#2};
  \end{pgfonlayer}{marx}
  } 
\providecommand{\cmark}[2][]{\relax} 
% -------------------------------------------------
% Start the picture
% !TEX root = ..\Eco-Model.tex
\begin{tikzpicture}
[mindmap,
every node/.style={concept, text=white, execute at begin node=\hskip0pt,circular drop shadow},
root concept/.append style={concept color=black, fill=white, line width=1ex, text=black, font=\Large\scshape},
grow cyclic,
level 1/.append style={level distance=4.5cm,sibling angle=90, font=\scshape},
level 2/.append style={level distance=3cm,sibling angle=45, font=\scriptsize\bf},
level 3/.append style={level distance=2cm,sibling angle=60, font=\scriptsize\bf},
level 4/.append style={level distance=1cm,sibling angle=60, font=\tiny\bf}]
\node [root concept] {Cloud Manufacturing Ecosystem}  % root
child [concept color = gray]{ node {Schedule}
child { node {Service Standardization}}
child { node {Decision Making Support}}
child { node {Platform Assistance}}
child { node {Match Regualtion}}
}
child [concept color = gray]{ node {User}
  child { node {Submit Order}}
  child { node {Provide Resource}}
  child { node {Make Transaction}}
  child { node {Evalute Others}}
  }
child [concept color = gray]{ node {Envolve}
  child { node {Optimization}
    child { node {Manufacturing}}
    child { node {Industrial Chain}}
    }
  child { node {Member}
    child { node {Introduce}}
    child { node {Eliminate}}
    }
  child { node {Adjust User Proportion}
    child { node {Rating}}
    child { node {Recommand}}
    }
  child { node {Development Guidance}}
  }
child [concept color = gray]{ node {Platform}
  child { node {Authority Control}}
  child { node {Transaction Bridge}}
  child { node {Dashbord}
    child { node {Service}}
    child { node {Task}}
    child { node {User}}
    }
child { node {Administrate}}
  child { node {Hub}
    child { node {Service}}
    child { node {Resource}}
    child { node {Ranking}}
    }
  child { node {Scheduling Assistance}}
}
;
\end{tikzpicture}


% =================================================
\end{document}