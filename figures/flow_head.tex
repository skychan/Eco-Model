% !Mode:: "TeX:UTF-8"
%%%%%%%%%%%%%%%%%%%%%%%%%%%%%%%%%%%%%%%%%%%%%%%%%%%%%%%%%%%%%%%
%
% Welcome to Overleaf --- just edit your LaTeX on the left,
% and we'll compile it for you on the right. If you give
% someone the link to this page, they can edit at the same
% time. See the help menu above for more info. Enjoy!
%
% Note: you can export the pdf to see the result at full
% resolution.
%
%%%%%%%%%%%%%%%%%%%%%%%%%%%%%%%%%%%%%%%%%%%%%%%%%%%%%%%%%%%%%%%
% Flowcharting techniques for easy maintenance
% Author: Brent Longborough
\documentclass[x11names]{article}
\usepackage{tikz}
\usetikzlibrary{mindmap,shapes,arrows,chains,shadows,decorations.markings,spy}
%%%<

\usepackage{verbatim}
\usepackage{paralist}
%\usepackage{geometry}
\usepackage[active,tightpage]{preview}
\PreviewEnvironment{tikzpicture}
\setlength{\PreviewBorder}{4bp}
%\setlength{\paperwidth}{400mm}
%%%>
\begin{comment}
:Title: Easy-maintenance flowchart
:Tags: flowcharts
:Author: Brent Longborough
:Slug: flexible-flow-chart

  This TikZ example illustrates a number of techniques for making TikZ
  flowcharts easier to maintain:
    * Use of <on chain> and <on grid> to simplify positioning
    * Use of global <node distance> options to eliminate the need to 
      specify individual inter-node distances
    * Use of <join> to reduce the need for references to node names
    * Use of <join by> styles to tailor specific connectors
    * Use of <coordinate> nodes to provide consistent layout for
      parallel flow lines
    * A method for consistent annotation of decision box exits
    * A technique for marking coordinate nodes (for layout debugging)

    I encourage you to tinker at this file - add intermediate boxes,
    alter the global distance settings, and so on, to see how well (or
    ill!) it adapts.
\end{comment}
\colorlet{lcfree}{Green3}
\colorlet{lcnorm}{Blue3}
\colorlet{lccong}{Red3}

% =================================================
% Set up a few colours

% -------------------------------------------------
% Set up a new layer for the debugging marks, and make sure it is on
% top
\pgfdeclarelayer{marx}
\pgfsetlayers{main,marx}
% A macro for marking coordinates (specific to the coordinate naming
% scheme used here). Swap the following 2 definitions to deactivate
% marks.
\providecommand{\cmark}[2][]{%
  \begin{pgfonlayer}{marx}
    \node [nmark] at (c#2#1) {#2};
  \end{pgfonlayer}{marx}
  } 
\providecommand{\cmark}[2][]{\relax} 

\usepackage{pgf-umlcd}
\begin{document}


% -------------------------------------------------
% Start the picture
\begin{tikzpicture}%[show background grid]
% % !TEX root = flow_head.tex

\begin{interface}[text width=.45\textwidth]{AssignBehavior}{0,0}
	\operation{+ BufferEnterance(t: Task, m: Machine): boolean}
	\operation{+ Queue(j: Job, m: Machine): void}
	\operation{+ Buff(t: Task, m: Machine): void}
\end{interface}


% % !TEX root = flow_head.tex

\begin{class}[text width=8cm]{CloudPlatform}{0,0}
	\attribute{- agentIDCounter: long}
	\attribute{\# agentID: String}
	\attribute{+ finishCount: int}
	\attribute{+ userList: List}
	\operation{+ CreateProvider(): void}
	\operation{+ CreateDemander(): void}
\end{class}


% % !TEX root = flow_head.tex

\begin{class}[text width=.45\textwidth]{Demander}{2.4,-1.2}
	\inherit{User}
	\attribute{+ orderList: List}
	\attribute{+ taskMap: Map}
	\operation{+ GenerateOrder(orderID: String): void}
	\operation{+ ReadData(orderID: String): void}
\end{class}


% % !TEX root = flow_head.tex

\begin{class}[text width=8cm]{Job}{0,0}
	\attribute{+ allocated: boolean}
	\attribute{+ allocation: Map}
	\attribute{+ candidates: Map}
	\attribute{+ inNeed: boolean}
	\attribute{+ needResourceCapacity: Map}
	\attribute{+ predecessor: Set}
	\attribute{+ prepareStatus: Map}
	\attribute{- processBehavior: ProcessBehavior}
	\attribute{- selectBehavior: SelectBehavior}
	\operation{+ addCandidates(competitor: Machine): void}
	\operation{+ CheckStatus(): void}
	\operation{+ Need(): void}
	\operation{+ Review(m: Machine): void}
	\operation{+ Select(watchedAgent: Job): void}
\end{class}


% % !TEX root = flow_head.tex

\begin{class}[text width=.45\textwidth]{Machine}{0,0}
	\attribute{+ assignBehavior: AssignBehavior}
	\attribute{+ buffer: List}
	\attribute{+ competeList: List}
	\attribute{+ jobList: List}
	\attribute{+ owner: Provider}
	\attribute{+ releaseBehavior: ReleaseBehavior}
	\attribute{+ responseBehavior: ResponseBehavior}
	\attribute{+ responseList: List}
	\attribute{+ type: String}
	\operation{+ Assign(t: Task): void}
	\operation{+ Release(j: Job): void}
	\operation{+ Reply(watchedAgent: Job): void}
	\operation{+ Response(watchedAgent: Task): void}
	\operation{+ Review(j: Job):void}
\end{class}


% % !TEX root = flow_head.tex

\begin{class}[text width=8cm]{Order}{0,0}
	\attribute{+ owner: Demander}
	\attribute{+ taskList: List}
	\attribute{+ type: String}
\end{class}
% % !TEX root = flow_head.tex

\begin{class}[text width=.45\textwidth]{Service}{2.4,-3.66}
	\inherit{Machine}
	\attribute{+ resourceComposition: Map}
	\operation{+ Service()}
	\operation{+ OutSource(): void}
\end{class}
% !TEX root = flow_head.tex


\begin{tikzpicture}
\foreach \x in {0,6.5,13,19.5}
	\draw [->,thick] (15.7cm,\x cm) node{t} (0,\x cm) -- +(15.5cm, 0) ;
\foreach \x in {0,6.5,13,19.5}
	\draw [->,thick] (0,\x cm) -- +(0,5.2 cm);
\foreach \x in {0,6.5,13,19.5}
	\draw[yshift=5.35cm] (0,\x cm) node{Capacity};
\foreach \x in {1,...,15}
	\foreach \y in {0,6.5,13,19.5}
		\draw (\x cm,\y cm) -- +(0,2mm);
\foreach \x in {1,...,15}
	\foreach \y in {0,6.5,13,19.5}
		\draw[yshift=-2mm] (\x cm,\y cm) node{$\x$};
\foreach \y in {0,6.5,13,19.5}
	\foreach \x in {1,...,10}
		\draw[yshift=\y cm] (0,0.5*\x cm) -- +(2mm,0) (-3mm,0.5*\x cm) node{$\x$};

\draw (7.5cm,-7mm) node{$r_4$, initial capacity = 10, type = 3};
\draw[yshift=6.5cm] (7.5cm,-7mm) node{$r_3$, initial capacity = 10, type = 3};
\draw[yshift=13cm] (7.5cm,-7mm) node{$r_2$, initial capacity = 10, type = 2};
\draw[yshift=19.5cm] (7.5cm,-7mm) node{$r_1$, initial capacity = 10, type = 1};

\fill[blue!20!,yshift=19.5cm] (6cm,4cm) rectangle (7cm,5cm);
\draw[yshift=19.5cm] (3cm,0) rectangle (6cm,3cm) (6cm,4cm) rectangle (8cm,5cm) (8cm,0) rectangle (13cm,3.5cm) (1cm,0) rectangle (3cm,4.5cm) (3cm,3cm) rectangle (6cm,3.5cm)  (4.5cm,1.6cm) node{$t_3$} (7cm,4.5cm) node{$sc_2$} (10.5cm,1.8cm) node{$t_2$} (2cm,2.25cm) node{$t_4$} (4.5cm,3.25cm) node{$t_5$};
\draw[dotted,yshift=19.5cm] (0,5cm) -- (7cm,5cm) (3cm,4cm) -- (15cm,4cm) (7cm,0) -- (7cm,5cm) (8cm,4cm) -- (8cm,3.5cm);
\draw[yshift=13cm] (0,0) rectangle (7cm,2.5cm) (7cm,4cm) rectangle (8cm,5cm) (8cm,0) rectangle (13cm,2.5cm) (1cm,2.5cm) rectangle (3cm,3.5cm)  (3.5cm,1.35cm) node{$t_1$} (7.5cm,4.5cm) node{$sc_2$} (10.5cm, 1.35cm) node{$t_2$}  (2cm,3cm) node{$t_4$};
\draw[dotted,yshift=13cm] (0,5cm) -- (7cm,5cm) (0,4cm) -- (15cm,4cm) (7cm,5cm) -- (7cm,0) (8cm,5cm) -- (8cm,0);
\draw[yshift=6.5cm] (0,0) rectangle (7cm,3.5cm) (7cm,2cm) rectangle (8cm,5cm) (3cm,3.5cm) rectangle (6cm,5cm) (3.5cm,1.85cm) node{$t_1$} (7.5cm,3.5cm) node{$sc_2$} (4.5cm,4.25cm) node{$t_5$};
\draw[dotted,yshift=6.5cm] (0,5cm) -- (7cm,5cm) (8cm,2cm) -- (15cm,2cm) (8cm,0) -- (8cm,2cm);
\draw (0,3.5cm) rectangle (1cm,5cm) (3cm,0) rectangle (6cm,2.5cm) (8cm,0) rectangle (13cm,3.5cm) (1cm,0) rectangle (3cm,3cm) (4.5cm, 1.35cm) node {$t_3$} (0.5cm,4.3cm) node{$sc_1$} (10.5cm,1.8cm) node{$t_2$} (2cm,1.5cm) node{$t_4$};
\draw[dotted] (0,3.5cm) -- (15cm,3.5cm) (1cm,0) -- (1cm,3.5cm);
\end{tikzpicture}

% % !TEX root = flow_head.tex
\begin{tikzpicture}

\foreach \x in {2,6,9}
	\fill[blue!20!] (\x cm, 0) rectangle +(1cm, 1cm);
\foreach \x in {2,6,9}
	\draw[xshift=.5cm] (\x cm,0.5cm) node{sc};	
\foreach \x in {0,...,10}
	\draw [thick] (\x cm, 0) rectangle +(1cm, 1cm);
\foreach \x in {0,1,3,4,5,7,8,10}
	\draw[xshift=0.5cm] (\x cm, 0.5cm) node{t};

\end{tikzpicture}

% % !TEX root = flow_head.tex
\begin{tikzpicture}

\fill[blue!20!,yshift=1.5cm] (3cm,0) rectangle (4cm,1cm);
\fill[blue!20!] (2cm,0) rectangle (3cm,1cm);
\draw[thick,yshift=1.5cm] (0,0) rectangle (4cm,1cm) (6.5cm,0) rectangle (11.5cm,1cm) (3cm,0) rectangle (4cm,1cm);
\draw[yshift=2cm] (-5mm,0) node{$mr_1$} (1.5cm,0) node{$t_{i1}$} (9cm,0) node{$t_{i2}$} (6cm,0) node{$\cdots$} (12cm,0) node{$\cdots$};
\draw[thick,dotted,xshift=5cm] (0,-5mm) -- (0,3.5cm);
\draw[thick] (0,0) rectangle (3cm,1cm) (6.5cm,0) rectangle (10.5cm,1cm) (2cm,0) rectangle (3cm,1cm);
\draw[yshift=0.5cm] (-5mm,0) node{$mr_2$}  (1cm,0) node{$t_{i2}$} (8.5cm,0) node{$t_{i1}$} (6cm,0) node{$\cdots$} (11cm,0) node{$\cdots$};
\draw[yshift=3cm] (2cm,0) node{active} (9.5cm,0) node{inactive};

\end{tikzpicture}

\end{tikzpicture}
%% !TEX root = flow_head.tex
\begin{tikzpicture}[
 >=triangle 60,              % Nice arrows; your taste may be different
 start chain=going right,    % General flow is top-to-bottom
 node distance=12em and 3em, % Global setup of box spacing
 every join/.style={norm},   % Default linetype for connecting boxes
]
% ------------------------------------------------- 
% A few box styles 
% <on chain> *and* <on grid> reduce the need for manual relative
% positioning of nodes
\tikzset{
  every node/.style={rectangle split, rectangle split parts=2, rectangle split horizontal, rectangle split part fill={lccong!25, yellow!10}, draw, anchor=center, minimum height=9em, on chain, on grid, text width=2em},
  base/.style={draw, on chain, on grid, align=center, minimum height=4ex},
  proc/.style={base, rectangle, text width=8em},
  Arrow/.style={thick, decoration={markings,mark=at position
   1 with {\arrow[semithick]{open triangle 60}}},
   double distance=1.4pt, shorten >= 5.5pt,
   preaction = {decorate},
   postaction = {draw,line width=1.4pt, white,shorten >= 4.5pt},fill = lcnorm!25},
  test/.style={base, diamond, aspect=2, text width=5em},
  term/.style={proc, rounded corners},
  % coord node style is used for placing corners of connecting lines
  coord/.style={coordinate, on chain, on grid, node distance=6mm and 25mm},
  % nmark node style is used for coordinate debugging marks
  nmark/.style={draw, cyan, circle, font={\sffamily\bfseries}},
  % -------------------------------------------------
  % Connector line styles for different parts of the diagram
  norm/.style={->, draw, lcnorm, thick},
  free/.style={->, draw, lcfree},
  cong/.style={->, draw, lccong},
  it/.style={font={\small\itshape}},
  tw/.style={text width=9.5em},
  th/.style={text width=2em}
}

\newcommand{\one}[1]{\nodepart[th]{one}\rotatebox{90}{\bf\shortstack{#1}}}
\newcommand{\two}{\nodepart[tw]{two}}

\node (n1)
{	
	\one{Submit\\ Order}
	\two
	\begin{inparaenum}
	\item 1.Original\\
	\item 2.Outsourcing
	\end{inparaenum}
};

\node[join] (n2) 
{
	\one{Task\\ Construct}
	\two
	\begin{inparaenum}
	\item 1.Order Decomposition\\
	\item 2.Task Merging\\
	\item 3.Task Splitting
	\end{inparaenum}
};

\node[join] (n3)
{
	\one{Service\\ Matching}
	\two
	\begin{inparaenum}
	\item 1.Recommandation\\
	\item 2.Service Selection\\
	\item 3.Seller's Rank
	\end{inparaenum}
};

\node[below=of n3,join] (n4)
{
	\one{Confirm\\ by Seller}
	\two
	\begin{inparaenum}
	\item 1.Accept Tasks\\
	\item 2.Check Due-date\\
	\item 3.Buyer Rank
	\end{inparaenum}
};

\node[below=of n2,join] (n5)
{
	\one{Wait for\\ Delivery}
	\two
	\begin{inparaenum}
	\item 1.Wait the Seller\\
	\item 2.Confirm the Prouct\\
	\item 3.Pay for it
	\end{inparaenum}
};

\node[below=of n1,join] (n6)
{
	\one{Mutual\\ Rating}
	\two
	\begin{inparaenum}
	\item 1.Rate Seller\\
	\item 2.Change Rank\\
	\item 3.Affect Recommand
	\end{inparaenum}
};
\end{tikzpicture}


% =================================================
\end{document}