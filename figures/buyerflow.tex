% !TEX root = flow_head.tex
\begin{tikzpicture}[
 >=triangle 60,              % Nice arrows; your taste may be different
 start chain=going right,    % General flow is top-to-bottom
 node distance=12em and 3em, % Global setup of box spacing
 every join/.style={norm},   % Default linetype for connecting boxes
]
% ------------------------------------------------- 
% A few box styles 
% <on chain> *and* <on grid> reduce the need for manual relative
% positioning of nodes
\tikzset{
  every node/.style={rectangle split, rectangle split parts=2, rectangle split horizontal, rectangle split part fill={lccong!25, yellow!10}, draw, anchor=center, minimum height=9em, on chain, on grid, text width=2em},
  base/.style={draw, on chain, on grid, align=center, minimum height=4ex},
  proc/.style={base, rectangle, text width=8em},
  Arrow/.style={thick, decoration={markings,mark=at position
   1 with {\arrow[semithick]{open triangle 60}}},
   double distance=1.4pt, shorten >= 5.5pt,
   preaction = {decorate},
   postaction = {draw,line width=1.4pt, white,shorten >= 4.5pt},fill = lcnorm!25},
  test/.style={base, diamond, aspect=2, text width=5em},
  term/.style={proc, rounded corners},
  % coord node style is used for placing corners of connecting lines
  coord/.style={coordinate, on chain, on grid, node distance=6mm and 25mm},
  % nmark node style is used for coordinate debugging marks
  nmark/.style={draw, cyan, circle, font={\sffamily\bfseries}},
  % -------------------------------------------------
  % Connector line styles for different parts of the diagram
  norm/.style={->, draw, lcnorm, thick},
  free/.style={->, draw, lcfree},
  cong/.style={->, draw, lccong},
  it/.style={font={\small\itshape}},
  tw/.style={text width=9.5em},
  th/.style={text width=2em}
}

\newcommand{\one}[1]{\nodepart[th]{one}\rotatebox{90}{\bf\shortstack{#1}}}
\newcommand{\two}{\nodepart[tw]{two}}

\node (n1)
{	
	\one{Submit\\ Order}
	\two
	\begin{inparaenum}
	\item 1.Original\\
	\item 2.Outsourcing
	\end{inparaenum}
};

\node[join] (n2) 
{
	\one{Task\\ Construct}
	\two
	\begin{inparaenum}
	\item 1.Order Decomposition\\
	\item 2.Task Merging\\
	\item 3.Task Splitting
	\end{inparaenum}
};

\node[join] (n3)
{
	\one{Service\\ Matching}
	\two
	\begin{inparaenum}
	\item 1.Recommandation\\
	\item 2.Service Selection\\
	\item 3.Seller's Rank
	\end{inparaenum}
};

\node[below=of n3,join] (n4)
{
	\one{Confirm\\ by Seller}
	\two
	\begin{inparaenum}
	\item 1.Accept Tasks\\
	\item 2.Check Due-date\\
	\item 3.Buyer Rank
	\end{inparaenum}
};

\node[below=of n2,join] (n5)
{
	\one{Wait for\\ Delivery}
	\two
	\begin{inparaenum}
	\item 1.Wait the Seller\\
	\item 2.Confirm the Prouct\\
	\item 3.Pay for it
	\end{inparaenum}
};

\node[below=of n1,join] (n6)
{
	\one{Mutual\\ Rating}
	\two
	\begin{inparaenum}
	\item 1.Rate Seller\\
	\item 2.Change Rank\\
	\item 3.Affect Recommand
	\end{inparaenum}
};
\end{tikzpicture}