% !TEX root = Eco-Model.tex
\section{Experiment and result Analysis} % (fold)
\label{sec:Experiment_results}

Entities in the manufacturing ecosystem we designed above have these traits, they are: \begin{inparaenum}
\item autonomous and self-directed, 
\item social, interacting with other entities, 
\item controlled by the environment, 
\item self-contained.
\end{inparaenum} They optimize their multi-objective decision makings to interact with others and the environment, hence, agent-based modeling and simulation(ABMS) technique\cite{Macal2009,north2007managing} is suitable for the study of the complex system. With the help of Repast Simphony\cite{North2013} package, we realized the ecosystem model and performed some experiments on it.
\subsection{Agent-based modeling and simulation design} % (fold)
\label{sub:agent_based_modeling_and_simulation}
\subsubsection{Model realization}
% \begin{asparaenum}
% \item Platform
% \suspend{asparaenum}
\begin{figure}[htbp]
	\centering
	\tiny
	\begin{tikzpicture}
	% !TEX root = flow_head.tex

\begin{class}[text width=.24\textwidth]{CloudPlatform}{0,0}
	\attribute{- agentIDCounter: long}
	\attribute{\# agentID: String}
	\attribute{+ finishCount: int}
	\attribute{+ userList: List}
	\operation{+ CreateProvider(): void}
	\operation{+ CreateDemander(): void}
\end{class}


	\end{tikzpicture}
	\caption{Cloud Platform}
	\label{fig:platform}
\end{figure}
% \resume{asparaenum}
% \item User related
% \suspend{asparaenum}
\begin{figure}[htbp]
    \centering
    \tiny
    \begin{tikzpicture}
    % !TEX root = flow_head.tex

\begin{class}[text width=8cm]{User}{0,0}
	\attribute{- agentIDCounter: long}
	\attribute{\# agentID: String}
\end{class}
    % !TEX root = flow_head.tex

\begin{class}[text width=8cm]{Provider}{0,0}
	\attribute{+ candidates: List}
	\attribute{+ rank: double}
	\attribute{+ resourceList: List}
	\attribute{+ serviceList: List}
	\operation{+ GenerateResource(): void}
	\operation{+ GenerateService(serviceData: List): void}
	\operation{+ GenerateServiceCall(): void}
	\operation{+ RemoveServiceCall(sc: ServiceCall): void}
	
\end{class}
% !TEX root = flow_head.tex

\begin{class}[text width=.45\textwidth]{Demander}{2.4,-1.5}
	\inherit{User}
	\attribute{+ orderList: List}
	\attribute{+ taskMap: Map}
	\operation{+ GenerateOrder(orderID: String): void}
	\operation{+ ReadData(orderID: String): void}
\end{class}


    \end{tikzpicture}
    \caption{User related classes}
    \label{fig:userclassed}
\end{figure}
% \resume{asparaenum}
% \item Job related
% \suspend{asparaenum}
\begin{figure}[htbp]
    \centering
    \tiny
    \begin{tikzpicture}
    % !TEX root = flow_head.tex

\begin{class}[text width=8cm]{Job}{0,0}
	\attribute{+ allocated: boolean}
	\attribute{+ allocation: Map}
	\attribute{+ candidates: Map}
	\attribute{+ inNeed: boolean}
	\attribute{+ needResourceCapacity: Map}
	\attribute{+ predecessor: Set}
	\attribute{+ prepareStatus: Map}
	\attribute{- processBehavior: ProcessBehavior}
	\attribute{- selectBehavior: SelectBehavior}
	\operation{+ addCandidates(competitor: Machine): void}
	\operation{+ CheckStatus(): void}
	\operation{+ Need(): void}
	\operation{+ Review(m: Machine): void}
	\operation{+ Select(watchedAgent: Job): void}
\end{class}


    % !TEX root = flow_head.tex

\begin{class}[text width=.45\textwidth]{Task}{-2.5,-3.7}
	\inherit{Job}
	\attribute{+ master: Order}
	\attribute{+ owner: Map}
	\attribute{+ pause: boolean}
	\attribute{+ processingTime: int}
	\attribute{+ remainingTime: int}
	\attribute{+ type: Strihng}
	\operation{+ Task()}
	\operation{+ addOwner(user: User, p: int): void}
	\operation{+ CheckStatus(): void}
	\operation{+ Process(watchedAgent: Task): void}
	\operation{+ setParameters(taskData: Map): void}
\end{class}
    % !TEX root = flow_head.tex

\begin{class}[text width=.45\textwidth]{ServiceCall}{2.4,-5.1}
	\inherit{Job}
	\attribute{+ owner: Provider}
	\attribute{+ type: String}
	\operation{+ ServiceCall()}
	\operation{+ setParameter(scData: Map): void}
	\operation{+ Recall(): void}
\end{class}
    \end{tikzpicture}
    \caption{Job related classes}
    \label{fig:jobclassed}
\end{figure}
% \resume{asparaenum}
% \item Machine related
% \end{asparaenum}
\begin{figure}[htbp]
    \centering
    \tiny
    \begin{tikzpicture}
    % !TEX root = flow_head.tex

\begin{class}[text width=8cm]{Machine}{0,0}
	\attribute{+ assignBehavior: AssignBehavior}
	\attribute{+ buffer: List}
	\attribute{+ competeList: List}
	\attribute{+ jobList: List}
	\attribute{+ owner: Provider}
	\attribute{+ releaseBehavior: ReleaseBehavior}
	\attribute{+ responseBehavior: ResponseBehavior}
	\attribute{+ responseList: List}
	\attribute{+ type: String}
	\operation{+ Assign(t: Task): void}
	\operation{+ Release(j: Job): void}
	\operation{+ Reply(watchedAgent: Job): void}
	\operation{+ Response(watchedAgent: Task): void}
	\operation{+ Review(j: Job):void}
\end{class}


    % !TEX root = flow_head.tex

\begin{class}[text width=.45\textwidth]{Resource}{-2.4,-5.5}
	\inherit{Machine}
	\attribute{+ available: int}
	\attribute{+ capacity: int}
	\attribute{+ master: Map}
	\attribute{+ needCap: Map}
	\attribute{+ sourceable: int}
	\operation{+ Resource()}
	\operation{+ addMaster(ser: Service, amonut: int): void}
	\operation{+ Assign(sc: ServiceCall, amount: int): void}
	\operation{+ Response(watchedAgent: ServiceCall): void}
\end{class}
    % !TEX root = flow_head.tex

\begin{class}[text width=.45\textwidth]{Service}{2.4,-3.66}
	\inherit{Machine}
	\attribute{+ resourceComposition: Map}
	\operation{+ Service()}
	\operation{+ OutSource(): void}
\end{class}
    \end{tikzpicture}
    \caption{Machine related classes}
    \label{fig:machineclassed}
\end{figure}

\subsubsection{Behavior interfaces and classes}
% \begin{asparaenum}
% \item Assign behavior
% \suspend{asparaenum}
\begin{figure}[htbp]
    \centering
    \tiny
    \begin{tikzpicture}
    % !TEX root = flow_head.tex

\begin{interface}[text width=.45\textwidth]{AssignBehavior}{0,0}
	\operation{+ BufferEnterance(t: Task, m: Machine): boolean}
	\operation{+ Queue(j: Job, m: Machine): void}
	\operation{+ Buff(t: Task, m: Machine): void}
\end{interface}


    % !TEX root = flow_head.tex

\begin{interface}[text width=8cm]{SelectBehavior}{0,0}
	\operation{+ Allocate(theOnes: Map): Map}
	\operation{+ Assign(allocation: Map, j: Job): void}
	\operation{+ Evaluation(j: Job, candidates: List): void}
	\operation{+ Select(j: Job, candidates: List): void}
\end{interface}
    % !TEX root = flow_head.tex

\begin{interface}[text width=.45\textwidth]{ResponseBehavior}{4.8,0}
	\operation{+ Exist(j: Job, m: Machine): boolean}
\end{interface}
    % !TEX root = flow_head.tex

\begin{interface}[text width=.45\textwidth]{ProcessBehavior}{0,0}
	\operation{+ Process(j: Job): void}
\end{interface}
    % !TEX root = flow_head.tex

\begin{interface}[text width=.45\textwidth]{ReleaseBehavior}{4.8,-1.1}
	\operation{+ Release(j: Job, m: Machine): void}
	\operation{+ Next(m: Machine): void}
\end{interface}
    % % !TEX root = flow_head.tex

\begin{class}[text width=8cm]{ServiceAssign}{0,0}
	\operation{+ BufferEnterance(t: Task, m: Machine): boolean}
	\operation{+ Queue(j: Job, m: Machine): void}
	\operation{+ Buff(t: Task, m: Machine): void}
\end{class}
    % % !TEX root = flow_head.tex

\begin{class}[text width=8cm]{ResourceAssign}{0,0}
	\operation{+ Buff(t: Task, m: Machine): void}
	\operation{+ BufferEnterance(t: Task, m: Machine): boolean}
	\operation{+ Queue(j: Job, m: Machine): void}
\end{class}
    \end{tikzpicture}
    \caption{Assign interface and related classes}
    \label{fig:assigninterface}
\end{figure}

% \resume{asparaenum}
% \item Process behavior
% \suspend{asparaenum}

% \begin{figure}[htbp]
%     \centering
%     \scriptsize
%     \begin{tikzpicture}
%     % !TEX root = flow_head.tex

\begin{interface}[text width=.45\textwidth]{ProcessBehavior}{0,0}
	\operation{+ Process(j: Job): void}
\end{interface}
%     % !TEX root = flow_head.tex

\begin{class}[text width=8cm]{ProcessInTask}{0,0}
	\operation{+ Process(j: Job): void}
\end{class}
%     % !TEX root = flow_head.tex

\begin{class}[text width=8cm]{ProcessInServiceCall}{0,0}
	\operation{+ Process(j: Job): void}
\end{class}
%     \end{tikzpicture}
%     \caption{Process interface and related classes}
%     \label{fig:processinterface}
% \end{figure}
% \resume{asparaenum}
% \item Release behavior
% \suspend{asparaenum}

% \begin{figure}[htbp]
%     \centering
%     \scriptsize
%     \begin{tikzpicture}
%     % !TEX root = flow_head.tex

\begin{interface}[text width=.45\textwidth]{ReleaseBehavior}{4.8,-1.1}
	\operation{+ Release(j: Job, m: Machine): void}
	\operation{+ Next(m: Machine): void}
\end{interface}
%     % !TEX root = flow_head.tex

\begin{class}[text width=8cm]{ResourceRelease}{0,0}
	\operation{+ Release(j: Job, m: Machine): void}
	\operation{+ Next(m: Machine): void}
\end{class}
%     % !TEX root = flow_head.tex

\begin{class}[text width=.45\textwidth]{ServiceRelease}{-2.4,-1.8}
	\implement{ReleaseBehavior}
	\operation{+ Release(j: Job, m: Machine): void}
	\operation{+ Next(m: Machine): void}
\end{class}
%     \end{tikzpicture}
%     \caption{Release interface and related classes}
%     \label{fig:releaseinterface}
% \end{figure}
% \resume{asparaenum}
% \item Response behavior
% \suspend{asparaenum}

% \begin{figure}[htbp]
%     \centering
%     \scriptsize
%     \begin{tikzpicture}
%     % !TEX root = flow_head.tex

\begin{interface}[text width=.45\textwidth]{ResponseBehavior}{4.8,0}
	\operation{+ Exist(j: Job, m: Machine): boolean}
\end{interface}
%     % !TEX root = flow_head.tex

\begin{class}[text width=.45\textwidth]{ResourceResponse}{2.4,-1.5}
	\implement{ResponseBehavior}
	\operation{+ Exist(j: Job, m: Machine): boolean}
\end{class}
%     % !TEX root = flow_head.tex

\begin{class}[text width=8cm]{ServiceResponse}{0,0}
	\operation{+ Exist(j: Job, m: Machine): boolean}
\end{class}
%     \end{tikzpicture}
%     \caption{Response interface and related classes}
%     \label{fig:reponseinterface}
% \end{figure}
% \resume{asparaenum}
% \item Select behavior
% \end{asparaenum}

% \begin{figure}[htbp]
%     \centering
%     \scriptsize
%     \begin{tikzpicture}
%     % !TEX root = flow_head.tex

\begin{interface}[text width=8cm]{SelectBehavior}{0,0}
	\operation{+ Allocate(theOnes: Map): Map}
	\operation{+ Assign(allocation: Map, j: Job): void}
	\operation{+ Evaluation(j: Job, candidates: List): void}
	\operation{+ Select(j: Job, candidates: List): void}
\end{interface}
%     \input{figures/cSelectInTask}
%     % !TEX root = flow_head.tex

\begin{class}[text width=8cm]{SelectInServiceCall}{0,0}
	\operation{+ Allocate(theOnes: Map): Map}
	\operation{+ Assign(allocation: Map, j: Job): void}
	\operation{+ Evaluation(j: Job, candidates: List): void}
	\operation{+ Select(j: Job, candidates: List): void}
\end{class}
%     \end{tikzpicture}
%     \caption{Select interface and related classes}
%     \label{fig:selectinterface}
% \end{figure}

\subsection{Experiments} % (fold)
\label{ssub:case_design}
We use the RanGen\cite{Demeulemeester2003,Vanhoucke2008} to generate the order data in the well-known Patterson format with the parameters setting listed in tab..., the control parameters are .... we design these experiments to validate service-incubate mode, outsource mode and metabolism mode \autoref{tab:grouping}

\begin{table}[htbp]
  \centering
  \scriptsize
  \caption{Experiments grouping}
    \begin{tabular}{llcc}
    \toprule
          &       & With metabolism & Without metabolism \\
    \midrule
    \multicolumn{1}{l}{\multirow{2}[0]{*}{Service-incubate}} & with outsource & Exp. 11 &Exp. 12 \\\cline{2-4}
    \multicolumn{1}{l}{} & without outsource & Exp. 21 & Exp. 22 \\\hline
    \multicolumn{2}{l}{Resource-only} & Exp. 31 & Exp. 32 \\
    \bottomrule
    \end{tabular}%
  \label{tab:grouping}%
\end{table}%


% subsubsection case_design (end)

\subsubsection{Parameters setting} % (fold)
\label{ssub:parameters_setting}
All the following experiments are simulated with same random seed to make no difference on the coming of order, provider and other irrelevant configuration to our validation.
% subsubsection parameters_setting (end)

\subsubsection{Result and analysis} % (fold)
\label{ssub:result_and_analysis}

% subsubsection result_and_analysis (end)