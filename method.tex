% !TEX root = Eco-Model.tex
\section{Methodological approach} % (fold)
\label{sec:methodological_approach}

This section starts with a brief overview of ABMS. Then .... Finally ...
i.e. the swarm model of the system.

In a word, the center of the research is the design of the user agent, and the core of the agent is to make decisions. so the core of the evolution is to make decision?

core of the paper: agent simulation, decision-making.

\subsection{Framework of Agent-based modeling} % (fold)
\label{sub:framework of agent based modeling}
\begin{figure}[htbp]
\centering\small
\subfloat[Order Flow]{\resizebox{0.45\textwidth}{!}{% !TEX root = flow_head.tex
\begin{tikzpicture}[%
    >=triangle 60,              % Nice arrows; your taste may be different
    start chain=going below,    % General flow is top-to-bottom
    node distance=6mm and 60mm, % Global setup of box spacing
    every join/.style={norm},   % Default linetype for connecting boxes
    ]
% ------------------------------------------------- 
% A few box styles 
% <on chain> *and* <on grid> reduce the need for manual relative
% positioning of nodes
\tikzset{
  base/.style={draw, on chain, on grid, align=center, minimum height=4ex},
  proc/.style={base, rectangle, text width=8em},
  test/.style={base, diamond, aspect=2, text width=5em},
  term/.style={proc, rounded corners},
  % coord node style is used for placing corners of connecting lines
  coord/.style={coordinate, on chain, on grid, node distance=6mm and 25mm},
  % nmark node style is used for coordinate debugging marks
  nmark/.style={draw, cyan, circle, font={\sffamily\bfseries}},
  % -------------------------------------------------
  % Connector line styles for different parts of the diagram
  norm/.style={->, draw, lcnorm},
  free/.style={->, draw, lcfree},
  cong/.style={->, draw, lccong},
  it/.style={font={\small\itshape}}
}
% -------------------------------------------------
% Start by placing the nodes
\node [term] {Be Submitted};
\end{tikzpicture}}\label{fig:orderflow}} \hspace{0.09\textwidth}
\subfloat[User Rank]{\resizebox{0.45\textwidth}{!}{% !TEX root = flow_head.tex
\begin{tikzpicture}[%
    >=triangle 60,              % Nice arrows; your taste may be different
    start chain=going below,    % General flow is top-to-bottom
    node distance=6mm and 60mm, % Global setup of box spacing
    every join/.style={norm},   % Default linetype for connecting boxes
    ]
% ------------------------------------------------- 
% A few box styles 
% <on chain> *and* <on grid> reduce the need for manual relative
% positioning of nodes
\tikzset{
  base/.style={draw, on chain, on grid, align=center, minimum height=4ex},
  proc/.style={base, rectangle, text width=8em},
  test/.style={base, diamond, aspect=2, text width=5em},
  term/.style={proc, rounded corners},
  % coord node style is used for placing corners of connecting lines
  coord/.style={coordinate, on chain, on grid, node distance=6mm and 25mm},
  % nmark node style is used for coordinate debugging marks
  nmark/.style={draw, cyan, circle, font={\sffamily\bfseries}},
  % -------------------------------------------------
  % Connector line styles for different parts of the diagram
  norm/.style={->, draw, lcnorm},
  free/.style={->, draw, lcfree},
  cong/.style={->, draw, lccong},
  it/.style={font={\small\itshape}}
}
% -------------------------------------------------
% Start by placing the nodes
\node [term]              (m1)  {Be Introduced};
\node [proc, join]        (p8)  {Get Initial Rank $R$};
\node [coord, yshift=2mm] (c1)  {}; \cmark{1}
\node [test, yshift=2mm]  (t1)  {Transact?};
\node [proc, yshift=-3mm] (p1)  {Increase $R$};


\node [proc, right=of t1] (p4)  {Wait};
\node [proc]              (p5)  {Compete to Get Task};
\node [test, join]        (t5)  {Make it?};
\node [coord, yshift=2mm] (c3)  {}; \cmark{3}

\node [test, yshift=2mm]  (t6)  {Outsourcing?};
\node [test]              (t7)  {Partly or Whole?};
\node [proc]              (p6)  {Submit Outsourcing Request};
\node [proc]              (p7)  {Decrease $R$};
\node [test, join]        (t8)  {$R < R_{base}$?};
\node [term]              (m2)  {Eliminate};

\node [test, left=of c3]  (t2)  {All Tasks Accomplished?};
\node [proc]              (p2)  {Process Task};
\node [test]              (t3)  {Accepted?};
\node [proc]              (p3)  {Pass the Outsouring Task};
\node [test]              (t4)  {Task Accomplished?};

\node [coord, left=of p1](c2)  {}; \cmark{2}

\draw [->, lcnorm] (p4) |- (c1);
\draw [->, lcnorm] (p8) -- (t1);
\draw [->, lcnorm] (p1) --(c2) |- (c1);

\path (t1.east)   to node [near start, yshift=1em] {$n$} (p4);
\path (t1.south) to node [near start, xshift=1em] {$y$} (p1);
  \draw [o->, lcnorm] (t1.east) -- (p4);
  \draw [*->, lcnorm] (t1.south) |- (p5);

\end{tikzpicture}}\label{fig:userrank}}
% \resizebox{0.9\textwidth}{!}{% !TEX root = ../Eco-Model.tex
\begin{tikzpicture}[%
    >=triangle 60,              % Nice arrows; your taste may be different
    start chain=going right,    % General flow is top-to-bottom
    node distance=6mm and 10mm, % Global setup of box spacing
    every join/.style={norm},   % Default linetype for connecting boxes
    ]
% ------------------------------------------------- 
% A few box styles 
% <on chain> *and* <on grid> reduce the need for manual relative
% positioning of nodes
\tikzset{
  base/.style={draw, on chain, on grid, align=center, rectangle},
  proc/.style={base, text width=6em, minimum height=8em},
  term/.style={base, text width=5em, rounded corners, minimum height = 3ex},
  dproc/.style={proc, minimum height = 20ex},
  gathe/.style={draw, on chain, on grid, align=center, circle, minimum height=4ex, text width=0.5em, node distance=10mm and 8mm},
  % coord node style is used for placing corners of connecting lines
  coord/.style={coordinate, on chain, on grid, node distance=6mm and 25mm},
  sky/.style={coordinate, on chain, on grid, node distance=6mm and 60mm},
  % nmark node style is used for coordinate debugging marks
  nmark/.style={draw, cyan, circle, font={\sffamily\bfseries}},
  % -------------------------------------------------
  % Connector line styles for different parts of the diagram
  norm/.style={->, draw, lcnorm},
  free/.style={->, draw, lcfree},
  cong/.style={->, draw, lccong},
  it/.style={font={\small\itshape}}
}
% -------------------------------------------------
% Start by placing the nodes
\node [term, fill = lcfree!25]              {Order};
\node [proc, join=by free]  (p1) {Task Hub};
\node [proc, dashed] (d) {};
\node [term, xshift = 12mm] (oc) {Order Check};
\node [term, join=by cong, fill = lccong!25]  {Submit};

\node [gathe, right=of d,yshift=5mm, xshift= 15mm] (g1) {};
\node [term, below=of d, yshift=15.5mm] (u1) {User 1};
\node [term, below=of u1] (u2) {User 2};
\node [below=-1mm of u2] (u3) {$\vdots$};
\node [term,below=of u3, yshift=-1mm] (u4) {User n};
\node [below=of u4,yshift=2.6mm] {User Cluster};
\node [gathe, below=of g1] (g2) {};
\node [coord, above=of g1, yshift=8mm] (c1) {};

\draw [->, lccong] (g2) -- (oc.west);
\draw [->, lcnorm] (p1) -- (u1.west);
\draw [->, lcnorm] (p1) -- (u2.west);
\draw [->, lcnorm] (p1) -- (u4.west);

\draw [->, lcnorm] (u1.east) -- (g1.west);
\draw [->, lccong] (u1.east) -- (g2.west);
\draw [->, lcnorm] (u2.east) -- (g1.west);
\draw [->, lccong] (u2.east) -- (g2.west);
\draw [->, lcnorm] (u4.east) -- (g1.west);
\draw [->, lccong] (u4.east) -- (g2.west);
\draw [->, lcnorm] (g1) -- (c1) -- node [black, near end, yshift=0.75em, it]
    {(Outsourcing)} ++(-3.5cm,0) -| (p1) ;

\end{tikzpicture}}
\caption{Simple Flow Chart of Order}
% \label{fig:simpleorderflow}
\end{figure}

\subsubsection{Assumptions and notation}
\begin{itemize}
    \item one provider can publish only one order at one time;
    \item service quality give the initial rank, the product quality depends on the service quality, review depends on the service quality, and the rank depends on the review;
    \item when it comes to rank value, we refer two different rank values: user rank(provider) and service rank, their relationship should be determined.
\end{itemize}


\subsection{Agents in the ecosystem model} % (fold)
\label{sub:agents_in_the_ecosystem_model}
Using UML
% subsection agents_in_the_ecosystem_model (end)

\subsubsection{Meta agent} % (fold)
\label{ssub:meta_agent}
is a decision maker
% subsubsection meta_agent (end)

\subsubsection{Demander agent} % (fold)
\label{ssub:demander_agent}
After pricing, chose the provider and the amount considering the time limit of the project and the budget.
% subsubsection demander_agent (end)

\subsubsection{Provider agent} % (fold)
\label{ssub:provider_agent}
Pricing function specify two main parts:
unit cost and
provided amount
% subsubsection provider_agent (end)

\subsubsection{Complex user agent} % (fold)
\label{ssub:complex_user_agent}

% subsubsection complex_user_agent (end)

\subsubsection{Platform agent} % (fold)
\label{ssub:platform_agent}

% subsubsection platform_agent (end)


\subsection{Ecosystem model} % (fold)
\label{sub:ecosystem model}

% subsection model_realization (end)

\subsubsection{Design philosophy} % (fold)
\label{ssub:design_philosophy}
platform create demander and provider
demander, a subclass of user, generate need i.e. create order (the defined task bunch)
    task is a subclass of job, need to be processed with specified resource type and amount(type,amount) , have its process time.
provider, a subclass of user, provider resource with different type and capacity, these resources can be allocated to task job and service-call job, service-call is subclass of job, which is the job to combined to create service.

the two type of job, task and service call, have something in common that need to be processd with resource. for task, the resource is renewable, for service-call, the resource is unrenewable.

service can be generated after the finish of service-call, so it belongs to the user who generate the service-call. service more like a contract

service-call come out driven by the task need, when a task appears frequently, provider proposed the job that have the same configuration. so the service generated by the service-call will suit for the specificed task.

good for service: efficieny, do not have to coordinated anymore, flexibe wile process with task, with the outsource trick, slice the taks, co-worked by all the useful resources.
bad for service: task-oriented, can only process specified task,

% subsubsection design_philosophy (end)

\subsubsection{Main functions} % (fold)
\label{ssub:main_functions}
service - task outsource
% subsubsection main_functions (end)

\subsubsection{Main problem} % (fold)
\label{ssub:main_problem}

% subsubsection main_problem (end)

\subsubsection{Main flow design} % (fold)
\label{ssub:main_flow_design}
\begin{itemize}
    \item generate provider and demander
    \item generate order (with task)
    \item generate resources
    \item process tasks (with only resource)
    \item organize service
    \item process task (plus service)
    \item Review and elimination of provider and service
\end{itemize}
% subsubsection main_flow_design (end)

\subsubsection{Main modules design}
\label{subsub:main_modules}
inherent relationships, execution sequence/queue.

for service provider to modify their service area, i.e. breadth and depth.

Decision-making involves choosing some course of action among various alternatives. For a decision maker, fulfilling the con

\subsubsection{System Model}
\begin{itemize}
    \item Co-evolution
    \item Self-organization
    \item Emergence
    \item Adaption
\end{itemize}

Assumptions:
\begin{enumerate}
    \item substitute transport with time consume
    \item manufacturing agent randomly show up in the map
\end{enumerate}

Co-evolution consist of:
\begin{itemize}
    \item Scarcity of customers
    \item Conscious choice that enables the organizations to change
    \item Interconnectedness of the enables the organizations to change
    \item Feedback processes that carry the long-term consequences of co-evolution
\end{itemize}

% subsection modeling (end)

\subsection{Simulation design} % (fold)
\label{sub:simulation_design}

% subsection simulation_design (end)

\subsection{Agent-based modeling and simulation} % (fold)
\label{sub:agent_based_modeling_and_simulation}
\subsubsection{Agent Classes}
\begin{asparaenum}
\item User related
\suspend{asparaenum}
\begin{figure}[htbp]
    \centering
    \scriptsize
    \begin{tikzpicture}
    % !TEX root = flow_head.tex

\begin{class}[text width=.45\textwidth]{User}{0,0}
	\attribute{- agentIDCounter: long}
	\attribute{\# agentID: String}
\end{class}
    % !TEX root = flow_head.tex

\begin{class}[text width=.45\textwidth]{Provider}{-2.4,-1.2}
	\inherit{User}
	\attribute{+ candidates: List}
	\attribute{+ rank: double}
	\attribute{+ resourceList: List}
	\attribute{+ serviceList: List}
	\operation{+ GenerateResource(): void}
	\operation{+ GenerateService(serviceData: List): void}
	\operation{+ GenerateServiceCall(): void}
	\operation{+ RemoveServiceCall(sc: ServiceCall): void}
	
\end{class}
% !TEX root = flow_head.tex

\begin{class}[text width=.45\textwidth]{Demander}{2.4,-1.2}
	\inherit{User}
	\attribute{+ orderList: List}
	\attribute{+ taskMap: Map}
	\operation{+ GenerateOrder(orderID: String): void}
	\operation{+ ReadData(orderID: String): void}
\end{class}


    \end{tikzpicture}
    \caption{User related classes}
    \label{fig:userclassed}
\end{figure}
\resume{asparaenum}
\item Job related
\suspend{asparaenum}
\begin{figure}[htbp]
    \centering
    \scriptsize
    \begin{tikzpicture}
    % !TEX root = flow_head.tex

\begin{class}[text width=8cm]{Job}{0,0}
	\attribute{+ allocated: boolean}
	\attribute{+ allocation: Map}
	\attribute{+ candidates: Map}
	\attribute{+ inNeed: boolean}
	\attribute{+ needResourceCapacity: Map}
	\attribute{+ predecessor: Set}
	\attribute{+ prepareStatus: Map}
	\attribute{- processBehavior: ProcessBehavior}
	\attribute{- selectBehavior: SelectBehavior}
	\operation{+ addCandidates(competitor: Machine): void}
	\operation{+ CheckStatus(): void}
	\operation{+ Need(): void}
	\operation{+ Review(m: Machine): void}
	\operation{+ Select(watchedAgent: Job): void}
\end{class}


    % !TEX root = flow_head.tex

\begin{class}[text width=8cm]{Task}{0,0}
	\attribute{+ master: Order}
	\attribute{+ owner: Map}
	\attribute{+ pause: boolean}
	\attribute{+ processingTime: int}
	\attribute{+ remainingTime: int}
	\attribute{+ type: Strihng}
	\operation{+ Task()}
	\operation{+ addOwner(user: User, p: int): void}
	\operation{+ CheckStatus(): void}
	\operation{+ Process(watchedAgent: Task): void}
	\operation{+ setParameters(taskData: Map): void}
\end{class}
    % !TEX root = flow_head.tex

\begin{class}[text width=.225\textwidth]{ServiceCall}{1.08,-3.7}
	\inherit{Job}
	\attribute{+ owner: Provider}
	\attribute{+ type: String}
	\operation{+ ServiceCall()}
	\operation{+ setParameter(scData: Map): void}
	\operation{+ Recall(): void}
\end{class}
    \end{tikzpicture}
    \caption{Job related classes}
    \label{fig:jobclassed}
\end{figure}
\resume{asparaenum}
\item Machine related
\end{asparaenum}
\begin{figure}[htbp]
    \centering
    \scriptsize
    \begin{tikzpicture}
    % !TEX root = flow_head.tex

\begin{class}[text width=.45\textwidth]{Machine}{0,0}
	\attribute{+ assignBehavior: AssignBehavior}
	\attribute{+ buffer: List}
	\attribute{+ competeList: List}
	\attribute{+ jobList: List}
	\attribute{+ owner: Provider}
	\attribute{+ releaseBehavior: ReleaseBehavior}
	\attribute{+ responseBehavior: ResponseBehavior}
	\attribute{+ responseList: List}
	\attribute{+ type: String}
	\operation{+ Assign(t: Task): void}
	\operation{+ Release(j: Job): void}
	\operation{+ Reply(watchedAgent: Job): void}
	\operation{+ Response(watchedAgent: Task): void}
	\operation{+ Review(j: Job):void}
\end{class}


    % !TEX root = flow_head.tex

\begin{class}[text width=.45\textwidth]{Resource}{-2.4,-3.66}
	\inherit{Machine}
	\attribute{+ available: int}
	\attribute{+ capacity: int}
	\attribute{+ master: Map}
	\attribute{+ needCap: Map}
	\attribute{+ sourceable: int}
	\operation{+ Resource()}
	\operation{+ addMaster(ser: Service, amonut: int): void}
	\operation{+ Assign(sc: ServiceCall, amount: int): void}
	\operation{+ Response(watchedAgent: ServiceCall): void}
\end{class}
    % !TEX root = flow_head.tex

\begin{class}[text width=.45\textwidth]{Service}{2.4,-3.66}
	\inherit{Machine}
	\attribute{+ resourceComposition: Map}
	\operation{+ Service()}
	\operation{+ OutSource(): void}
\end{class}
    \end{tikzpicture}
    \caption{Machine related classes}
    \label{fig:machineclassed}
\end{figure}
\subsubsection{Behavior interface and classes}
\begin{asparaenum}
\item Assign behavior
\suspend{asparaenum}
\begin{figure}[htbp]
    \centering
    \scriptsize
    \begin{tikzpicture}
    % !TEX root = flow_head.tex

\begin{interface}[text width=.45\textwidth]{AssignBehavior}{0,0}
	\operation{+ BufferEnterance(t: Task, m: Machine): boolean}
	\operation{+ Queue(j: Job, m: Machine): void}
	\operation{+ Buff(t: Task, m: Machine): void}
\end{interface}


    % !TEX root = flow_head.tex

\begin{class}[text width=8cm]{ServiceAssign}{0,0}
	\operation{+ BufferEnterance(t: Task, m: Machine): boolean}
	\operation{+ Queue(j: Job, m: Machine): void}
	\operation{+ Buff(t: Task, m: Machine): void}
\end{class}
    % !TEX root = flow_head.tex

\begin{class}[text width=.45\textwidth]{ResourceAssign}{2.4,-2.5}
	\implement{AssignBehavior}
	\operation{+ Buff(t: Task, m: Machine): void}
	\operation{+ BufferEnterance(t: Task, m: Machine): boolean}
	\operation{+ Queue(j: Job, m: Machine): void}
\end{class}
    \end{tikzpicture}
    \caption{Assign interface and related classes}
    \label{fig:assigninterface}
\end{figure}

\resume{asparaenum}
\item Process behavior
\suspend{asparaenum}

\begin{figure}[htbp]
    \centering
    \scriptsize
    \begin{tikzpicture}
    % !TEX root = flow_head.tex

\begin{interface}[text width=.45\textwidth]{ProcessBehavior}{4.8,-2.35}
	\operation{+ Process(j: Job): void}
\end{interface}
    % !TEX root = flow_head.tex

\begin{class}[text width=.45\textwidth]{ProcessInTask}{-2.4,-1.5}
	\implement{ProcessBehavior}
	\operation{+ Process(j: Job): void}
\end{class}
    % !TEX root = flow_head.tex

\begin{class}[text width=.45\textwidth]{ProcessInServiceCall}{2.4,-1.5}
	\implement{ProcessBehavior}
	\operation{+ Process(j: Job): void}
\end{class}
    \end{tikzpicture}
    \caption{Process interface and related classes}
    \label{fig:processinterface}
\end{figure}
\resume{asparaenum}
\item Release behavior
\suspend{asparaenum}

\begin{figure}[htbp]
    \centering
    \scriptsize
    \begin{tikzpicture}
    % !TEX root = flow_head.tex

\begin{interface}[text width=8cm]{ReleaseBehavior}{0,0}
	\operation{+ Release(j: Job, m: Machine): void}
	\operation{+ Next(m: Machine): void}
\end{interface}
    % !TEX root = flow_head.tex

\begin{class}[text width=.45\textwidth]{ResourceRelease}{2.4,-1.8}
	\implement{ReleaseBehavior}
	\operation{+ Release(j: Job, m: Machine): void}
	\operation{+ Next(m: Machine): void}
\end{class}
    % !TEX root = flow_head.tex

\begin{class}[text width=.45\textwidth]{ServiceRelease}{-2.4,-1.8}
	\implement{ReleaseBehavior}
	\operation{+ Release(j: Job, m: Machine): void}
	\operation{+ Next(m: Machine): void}
\end{class}
    \end{tikzpicture}
    \caption{Release interface and related classes}
    \label{fig:releaseinterface}
\end{figure}
\resume{asparaenum}
\item Response behavior
\suspend{asparaenum}

\begin{figure}[htbp]
    \centering
    \scriptsize
    \begin{tikzpicture}
    % !TEX root = flow_head.tex

\begin{interface}[text width=.45\textwidth]{ResponseBehavior}{0,0}
	\operation{+ Exist(j: Job, m: Machine): boolean}
\end{interface}
    % !TEX root = flow_head.tex

\begin{class}[text width=8cm]{ResourceResponse}{0,0}
	\operation{+ Exist(j: Job, m: Machine): boolean}
\end{class}
    % !TEX root = flow_head.tex

\begin{class}[text width=.45\textwidth]{ServiceResponse}{-2.4,-1.5}
	\implement{ResponseBehavior}
	\operation{+ Exist(j: Job, m: Machine): boolean}
\end{class}
    \end{tikzpicture}
    \caption{Response interface and related classes}
    \label{fig:reponseinterface}
\end{figure}
\resume{asparaenum}
\item Select behavior
\end{asparaenum}

\begin{figure}[htbp]
    \centering
    \scriptsize
    \begin{tikzpicture}
    % !TEX root = flow_head.tex

\begin{interface}[text width=.45\textwidth]{SelectBehavior}{0,-1.7}
	\operation{+ Allocate(theOnes: Map): Map}
	\operation{+ Assign(allocation: Map, j: Job): void}
	\operation{+ Evaluation(j: Job, candidates: List): void}
	\operation{+ Select(j: Job, candidates: List): void}
\end{interface}
    % !TEX root = flow_head.tex

\begin{class}[text width=8cm]{SelectInTask}{0,0}
	\operation{+ Allocate(theOnes: Map): Map}
	\operation{+ Assign(allocation: Map, j: Job): void}
	\operation{+ Evaluation(j: Job, candidates: List): void}
	\operation{+ Select(j: Job, candidates: List): void}
\end{class}
    % !TEX root = flow_head.tex

\begin{class}[text width=8cm]{SelectInServiceCall}{0,0}
	\operation{+ Allocate(theOnes: Map): Map}
	\operation{+ Assign(allocation: Map, j: Job): void}
	\operation{+ Evaluation(j: Job, candidates: List): void}
	\operation{+ Select(j: Job, candidates: List): void}
\end{class}
    \end{tikzpicture}
    \caption{Select interface and related classes}
    \label{fig:selectinterface}
\end{figure}
\subsubsection{Interaction among agents}
some mathematical formulation display here to describe the whole ecosystem wide phenomenon, like Little formulation.


% subsection agent_based_model_and_simulation (end)
% section methodological_approach (end)
