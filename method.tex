% !TEX root = Eco-Model.tex
\section{Methodological approach} % (fold)
\label{sec:methodological_approach}

\subsection{Modeling} % (fold)
\label{sub:modeling}

\subsubsection{Assumptions and notation}

Ecosystem application model including:
\subsubsection{Enterprise model}

The Model can be described with two main parts, namely object itself(attribute) and the interaction with other enterprises(function or method)

Users from platform managers, enterpirse users, customers. Even though they characteristic can be overlaped. just inherent it.

Enterprise can be described as the object with attributes like 
\begin{itemize}
	\item Rank(reputation)
	\item Resources(type, quantity, quality)
	\item Capbility
	\item Processing Status
\end{itemize}

\subsubsection{Main modules design}
\label{subsub:main_modules}
for service provider to modify their service area, i.e. breadth and depth.

\subsubsection{System Model}
\begin{itemize}
	\item Coevolution
	\item Self-organization
	\item Emergence	
	\item Adaption
\end{itemize}

Assumptions:
\begin{enumerate}
	\item substitude transport with time consume
	\item manufacturing agent randomly show up in the map
\end{enumerate}

Coevolution consist of:
\begin{itemize}
	\item Scarcity of customers
	\item Conscious choice that enables the organizations to change
	\item Interconnectedness of the enables the organizations to change
	\item Feedback processes that carry the long-term consequences of coevolution
\end{itemize}

% subsection modeling (end)

% \subsection{Mathematical Formulation} % (fold)
% \label{sub:mathematical_formulation}
% \subsubsection{Comprehended Model}
% % subsection mathematical_formulation (end)

% \subsection{Verification and Validation} % (fold)
% \label{sub:verification_and_validation}

% % subsection verification_and_validation (end)

\subsection{Agent-based modeling and simulation} % (fold)
\label{sub:agent_based_modeling_and_simulation}
\subsubsection{Agent design}

\subsubsection{Logical process}
\begin{figure}[htbp]
\centering\small
\subfloat[Order Flow]{\resizebox{0.45\textwidth}{!}{% !TEX root = flow_head.tex
\begin{tikzpicture}[%
    >=triangle 60,              % Nice arrows; your taste may be different
    start chain=going below,    % General flow is top-to-bottom
    node distance=6mm and 60mm, % Global setup of box spacing
    every join/.style={norm},   % Default linetype for connecting boxes
    ]
% ------------------------------------------------- 
% A few box styles 
% <on chain> *and* <on grid> reduce the need for manual relative
% positioning of nodes
\tikzset{
  base/.style={draw, on chain, on grid, align=center, minimum height=4ex},
  proc/.style={base, rectangle, text width=8em},
  test/.style={base, diamond, aspect=2, text width=5em},
  term/.style={proc, rounded corners},
  % coord node style is used for placing corners of connecting lines
  coord/.style={coordinate, on chain, on grid, node distance=6mm and 25mm},
  % nmark node style is used for coordinate debugging marks
  nmark/.style={draw, cyan, circle, font={\sffamily\bfseries}},
  % -------------------------------------------------
  % Connector line styles for different parts of the diagram
  norm/.style={->, draw, lcnorm},
  free/.style={->, draw, lcfree},
  cong/.style={->, draw, lccong},
  it/.style={font={\small\itshape}}
}
% -------------------------------------------------
% Start by placing the nodes
\node [term] {Be Submitted};
\end{tikzpicture}}\label{fig:orderflow}} \hspace{0.09\textwidth}
\subfloat[User Rank]{\resizebox{0.45\textwidth}{!}{% !TEX root = flow_head.tex
\begin{tikzpicture}[%
    >=triangle 60,              % Nice arrows; your taste may be different
    start chain=going below,    % General flow is top-to-bottom
    node distance=6mm and 60mm, % Global setup of box spacing
    every join/.style={norm},   % Default linetype for connecting boxes
    ]
% ------------------------------------------------- 
% A few box styles 
% <on chain> *and* <on grid> reduce the need for manual relative
% positioning of nodes
\tikzset{
  base/.style={draw, on chain, on grid, align=center, minimum height=4ex},
  proc/.style={base, rectangle, text width=8em},
  test/.style={base, diamond, aspect=2, text width=5em},
  term/.style={proc, rounded corners},
  % coord node style is used for placing corners of connecting lines
  coord/.style={coordinate, on chain, on grid, node distance=6mm and 25mm},
  % nmark node style is used for coordinate debugging marks
  nmark/.style={draw, cyan, circle, font={\sffamily\bfseries}},
  % -------------------------------------------------
  % Connector line styles for different parts of the diagram
  norm/.style={->, draw, lcnorm},
  free/.style={->, draw, lcfree},
  cong/.style={->, draw, lccong},
  it/.style={font={\small\itshape}}
}
% -------------------------------------------------
% Start by placing the nodes
\node [term]              (m1)  {Be Introduced};
\node [proc, join]        (p8)  {Get Initial Rank $R$};
\node [coord, yshift=2mm] (c1)  {}; \cmark{1}
\node [test, yshift=2mm]  (t1)  {Transact?};
\node [proc, yshift=-3mm] (p1)  {Increase $R$};


\node [proc, right=of t1] (p4)  {Wait};
\node [proc]              (p5)  {Compete to Get Task};
\node [test, join]        (t5)  {Make it?};
\node [coord, yshift=2mm] (c3)  {}; \cmark{3}

\node [test, yshift=2mm]  (t6)  {Outsourcing?};
\node [test]              (t7)  {Partly or Whole?};
\node [proc]              (p6)  {Submit Outsourcing Request};
\node [proc]              (p7)  {Decrease $R$};
\node [test, join]        (t8)  {$R < R_{base}$?};
\node [term]              (m2)  {Eliminate};

\node [test, left=of c3]  (t2)  {All Tasks Accomplished?};
\node [proc]              (p2)  {Process Task};
\node [test]              (t3)  {Accepted?};
\node [proc]              (p3)  {Pass the Outsouring Task};
\node [test]              (t4)  {Task Accomplished?};

\node [coord, left=of p1](c2)  {}; \cmark{2}

\draw [->, lcnorm] (p4) |- (c1);
\draw [->, lcnorm] (p8) -- (t1);
\draw [->, lcnorm] (p1) --(c2) |- (c1);

\path (t1.east)   to node [near start, yshift=1em] {$n$} (p4);
\path (t1.south) to node [near start, xshift=1em] {$y$} (p1);
  \draw [o->, lcnorm] (t1.east) -- (p4);
  \draw [*->, lcnorm] (t1.south) |- (p5);

\end{tikzpicture}}\label{fig:userrank}}
% \resizebox{0.9\textwidth}{!}{% !TEX root = ../Eco-Model.tex
\begin{tikzpicture}[%
    >=triangle 60,              % Nice arrows; your taste may be different
    start chain=going right,    % General flow is top-to-bottom
    node distance=6mm and 10mm, % Global setup of box spacing
    every join/.style={norm},   % Default linetype for connecting boxes
    ]
% ------------------------------------------------- 
% A few box styles 
% <on chain> *and* <on grid> reduce the need for manual relative
% positioning of nodes
\tikzset{
  base/.style={draw, on chain, on grid, align=center, rectangle},
  proc/.style={base, text width=6em, minimum height=8em},
  term/.style={base, text width=5em, rounded corners, minimum height = 3ex},
  dproc/.style={proc, minimum height = 20ex},
  gathe/.style={draw, on chain, on grid, align=center, circle, minimum height=4ex, text width=0.5em, node distance=10mm and 8mm},
  % coord node style is used for placing corners of connecting lines
  coord/.style={coordinate, on chain, on grid, node distance=6mm and 25mm},
  sky/.style={coordinate, on chain, on grid, node distance=6mm and 60mm},
  % nmark node style is used for coordinate debugging marks
  nmark/.style={draw, cyan, circle, font={\sffamily\bfseries}},
  % -------------------------------------------------
  % Connector line styles for different parts of the diagram
  norm/.style={->, draw, lcnorm},
  free/.style={->, draw, lcfree},
  cong/.style={->, draw, lccong},
  it/.style={font={\small\itshape}}
}
% -------------------------------------------------
% Start by placing the nodes
\node [term, fill = lcfree!25]              {Order};
\node [proc, join=by free]  (p1) {Task Hub};
\node [proc, dashed] (d) {};
\node [term, xshift = 12mm] (oc) {Order Check};
\node [term, join=by cong, fill = lccong!25]  {Submit};

\node [gathe, right=of d,yshift=5mm, xshift= 15mm] (g1) {};
\node [term, below=of d, yshift=15.5mm] (u1) {User 1};
\node [term, below=of u1] (u2) {User 2};
\node [below=-1mm of u2] (u3) {$\vdots$};
\node [term,below=of u3, yshift=-1mm] (u4) {User n};
\node [below=of u4,yshift=2.6mm] {User Cluster};
\node [gathe, below=of g1] (g2) {};
\node [coord, above=of g1, yshift=8mm] (c1) {};

\draw [->, lccong] (g2) -- (oc.west);
\draw [->, lcnorm] (p1) -- (u1.west);
\draw [->, lcnorm] (p1) -- (u2.west);
\draw [->, lcnorm] (p1) -- (u4.west);

\draw [->, lcnorm] (u1.east) -- (g1.west);
\draw [->, lccong] (u1.east) -- (g2.west);
\draw [->, lcnorm] (u2.east) -- (g1.west);
\draw [->, lccong] (u2.east) -- (g2.west);
\draw [->, lcnorm] (u4.east) -- (g1.west);
\draw [->, lccong] (u4.east) -- (g2.west);
\draw [->, lcnorm] (g1) -- (c1) -- node [black, near end, yshift=0.75em, it]
    {(Outsourcing)} ++(-3.5cm,0) -| (p1) ;

\end{tikzpicture}}
\caption{Simple Flow Chart of Order}
% \label{fig:simpleorderflow}
\end{figure}

\subsubsection{Interaction among agents}
some mathematical formulation display here to describe the whole ecosystem wide phenomenon, like Little formulation.


% subsection agent_based_model_and_simulation (end)
% section methodological_approach (end)