% !TEX root = Eco-Model.tex
\section{Methodological Approach} % (fold)
\label{sec:methodological_approach}

\subsection{Modeling} % (fold)
\label{sub:modeling}

\subsubsection{Assumptions and Notation}

Ecosystem application model including:
\subsubsection{Enterprise Model}

The Model can be described with two main parts, namely object itself(attribute) and the interaction with other enterprises(function or method)

Users from platform managers, enterpirse users, customers. Even though they characteristic can be overlaped. just inherent it.

Enterprise can be described as the object with attributes like 
\begin{itemize}
	\item Rank(reputation)
	\item Resources(type, quantity, quality)
	\item Capbility
	\item Processing Status
\end{itemize}


\subsubsection{System Model}
\begin{itemize}
	\item Coevolution
	\item Self-organization
	\item Emergence	
	\item Adaption
\end{itemize}

Coevolution consist of:
\begin{itemize}
	\item Scarcity of customers
	\item Conscious choice that enables the organizations to change
	\item Interconnectedness of the enables the organizations to change
	\item Feedback processes that carry the long-term consequences of coevolution
\end{itemize}

\subsubsection{Management Model}

% subsection modeling (end)

\subsection{Mathematical Formulation} % (fold)
\label{sub:mathematical_formulation}
\subsubsection{Comprehended Model}
% subsection mathematical_formulation (end)

\subsection{Model Validation} % (fold)
\label{sub:model_validation}

% subsection model_validation (end)

% section methodological_approach (end)