% !TEX root = Eco-Model.tex
\section{Preliminaries} % (fold)
 
\subsection{Services and Tasks}
Cloud Manufacturing Ecosystem provides a platform where distributed manufacturing resources/orders gathered as resources/orders hub. In order to complete orders precisely, order will be splited into tasks. Similarly, resources will be bundled into services for the quick searhing.

\subsection{Roles}
Most recent studies like Wu et al.\cite{Wu2013} described the roles in Cloud Manufacturing as a tri-group model with:\begin{inparaenum}[1)]
\item Users/Customers,
\item Application providers and
\item Physical resource providers.
\end{inparaenum}
However, the roles in real world always not go like this model, since it parted roles only by their functions. While, users could get different function roles depend on what transaction they take, we should make some modifiacations on this model.

Before the emendation of the roles, some related standard name should be taken:
\begin{asparadesc}
\item[User] The main role in Cloud Manufacturing, which is combined of Seller and Buyer as shown in \autoref{tab:categoryuser}, usually acts as Seller or Buyer according to the specific transaction;
\item[Seller] The temporary role in a transaction, which provides the allocated part of manufacturing service and makes a evalution about the Buyer after the transaction;
\item[Buyer] The temporary role in a transaction, which submits the allocated part of manufacturing task and makes a evalution about the Seller after the transaction;
\item[Administrator] The role that gets the work for the operation and maintenance of the platform, has indirectly impact on the Ecosystem.
\end{asparadesc}
\begin{table}[htbp]
  \centering\small
  \caption{Category of Users}
    \begin{tabularx}{\textwidth}{llccX}
    \toprule
        &  & Seller & Buyer & Description \\
    \midrule
    Generalized & User & $\surd$ & $\surd$ & Act as Seller or Buyer according to the specific transaction. \\
    Enterprise & User & $\surd$ & &  Provides the allocated part of manufacturing services and makes a evalution about the Buyer after the transaction. \\
    Customer & User& & $\surd$ &  Submits the allocated part of manufacturing tasks and makes a evalution about the Seller after the transaction.\\
    \bottomrule
    \end{tabularx}%
  \label{tab:categoryuser}%
\end{table}%
As defined above, Enterprise User is the special case of the User(short for Generalized User) that purely provides manufacturing services and the Customer User is similarly defined.

Hence, there is only two roles in the Ecosystem worth to be studied: User and Administrator, the proportion of user type could be adjusted as the runtime of the system goes by.

\subsection{Operation Mode of Ecosystem}
The style the Cloud Manufacturing Ecosystem appears mainly depends on the way the operation mode takes, therefore, research on features and issues in the Ecosystem would be taken within the scope of operation mode we are going to set up. As shown in \autoref{fig:platformstruct}, the principal struct of Ecosystem can be described into four main branch, how these branches grow determines how the Operation Mode will come out, and it will be easy to figure out the specified mode we make with some flow charts.

\begin{figure}[!h]
\centering
%\resizebox{0.7\textwidth}{!}{% !TEX root = ..\Eco-Model.tex
\begin{tikzpicture}
[mindmap,
every node/.style={concept, text=white, execute at begin node=\hskip0pt,circular drop shadow},
root concept/.append style={concept color=black, fill=white, line width=1ex, text=black, font=\Large\scshape},
grow cyclic,
level 1/.append style={level distance=4.5cm,sibling angle=90, font=\scshape},
level 2/.append style={level distance=3cm,sibling angle=45, font=\scriptsize\bf},
level 3/.append style={level distance=2cm,sibling angle=60, font=\scriptsize\bf},
level 4/.append style={level distance=1cm,sibling angle=60, font=\tiny\bf}]
\node [root concept] {Cloud Manufacturing Ecosystem}  % root
child [concept color = gray]{ node {Schedule}
child { node {Service Standardization}}
child { node {Decision Making Support}}
child { node {Platform Assistance}}
child { node {Match Regualtion}}
}
child [concept color = gray]{ node {User}
  child { node {Submit Order}}
  child { node {Provide Resource}}
  child { node {Make Transaction}}
  child { node {Evalute Others}}
  }
child [concept color = gray]{ node {Envolve}
  child { node {Optimization}
    child { node {Manufacturing}}
    child { node {Industrial Chain}}
    }
  child { node {Member}
    child { node {Introduce}}
    child { node {Eliminate}}
    }
  child { node {Adjust User Proportion}
    child { node {Rating}}
    child { node {Recommand}}
    }
  child { node {Development Guidance}}
  }
child [concept color = gray]{ node {Platform}
  child { node {Authority Control}}
  child { node {Transaction Bridge}}
  child { node {Dashbord}
    child { node {Service}}
    child { node {Task}}
    child { node {User}}
    }
child { node {Administrate}}
  child { node {Hub}
    child { node {Service}}
    child { node {Resource}}
    child { node {Ranking}}
    }
  child { node {Scheduling Assistance}}
}
;
\end{tikzpicture}}
\caption{Principal Struct of Cloud Manufacturing Ecosystem}
\label{fig:platformstruct}
\end{figure}

Because the metioned system is too complex to analyse in every detail, we focus on two important operation flows -- Order Flow and User Rank -- to help illustrate the Operation Mode. 

\subsubsection{Order Flow}
Each manufacturing order (order for short) appears when Users make transaction in the platform, it might then go through some different modules and a simple version of its flow was shown on \autoref{fig:orderflow}.

If an order accomplished smoothly, it will be constructed into tasks pieces in order to match the appropriate manufacturing services usually, similar to a large dataset processed with MapReduce\cite{Dean2008} schema. Then each piece could be processed by distributed resources with the outsourcing strategy User makes.

\begin{figure}[!h]
\centering\small
%\subfloat[Order Flow]{\resizebox{0.45\textwidth}{!}{% !TEX root = flow_head.tex
\begin{tikzpicture}[%
    >=triangle 60,              % Nice arrows; your taste may be different
    start chain=going below,    % General flow is top-to-bottom
    node distance=6mm and 60mm, % Global setup of box spacing
    every join/.style={norm},   % Default linetype for connecting boxes
    ]
% ------------------------------------------------- 
% A few box styles 
% <on chain> *and* <on grid> reduce the need for manual relative
% positioning of nodes
\tikzset{
  base/.style={draw, on chain, on grid, align=center, minimum height=4ex},
  proc/.style={base, rectangle, text width=8em},
  test/.style={base, diamond, aspect=2, text width=5em},
  term/.style={proc, rounded corners},
  % coord node style is used for placing corners of connecting lines
  coord/.style={coordinate, on chain, on grid, node distance=6mm and 25mm},
  % nmark node style is used for coordinate debugging marks
  nmark/.style={draw, cyan, circle, font={\sffamily\bfseries}},
  % -------------------------------------------------
  % Connector line styles for different parts of the diagram
  norm/.style={->, draw, lcnorm},
  free/.style={->, draw, lcfree},
  cong/.style={->, draw, lccong},
  it/.style={font={\small\itshape}}
}
% -------------------------------------------------
% Start by placing the nodes
\node [term] {Be Submitted};
\end{tikzpicture}}\label{fig:orderflow}} \hspace{0.09\textwidth}
%\subfloat[User Rank]{\resizebox{0.45\textwidth}{!}{% !TEX root = flow_head.tex
\begin{tikzpicture}[%
    >=triangle 60,              % Nice arrows; your taste may be different
    start chain=going below,    % General flow is top-to-bottom
    node distance=6mm and 60mm, % Global setup of box spacing
    every join/.style={norm},   % Default linetype for connecting boxes
    ]
% ------------------------------------------------- 
% A few box styles 
% <on chain> *and* <on grid> reduce the need for manual relative
% positioning of nodes
\tikzset{
  base/.style={draw, on chain, on grid, align=center, minimum height=4ex},
  proc/.style={base, rectangle, text width=8em},
  test/.style={base, diamond, aspect=2, text width=5em},
  term/.style={proc, rounded corners},
  % coord node style is used for placing corners of connecting lines
  coord/.style={coordinate, on chain, on grid, node distance=6mm and 25mm},
  % nmark node style is used for coordinate debugging marks
  nmark/.style={draw, cyan, circle, font={\sffamily\bfseries}},
  % -------------------------------------------------
  % Connector line styles for different parts of the diagram
  norm/.style={->, draw, lcnorm},
  free/.style={->, draw, lcfree},
  cong/.style={->, draw, lccong},
  it/.style={font={\small\itshape}}
}
% -------------------------------------------------
% Start by placing the nodes
\node [term]              (m1)  {Be Introduced};
\node [proc, join]        (p8)  {Get Initial Rank $R$};
\node [coord, yshift=2mm] (c1)  {}; \cmark{1}
\node [test, yshift=2mm]  (t1)  {Transact?};
\node [proc, yshift=-3mm] (p1)  {Increase $R$};


\node [proc, right=of t1] (p4)  {Wait};
\node [proc]              (p5)  {Compete to Get Task};
\node [test, join]        (t5)  {Make it?};
\node [coord, yshift=2mm] (c3)  {}; \cmark{3}

\node [test, yshift=2mm]  (t6)  {Outsourcing?};
\node [test]              (t7)  {Partly or Whole?};
\node [proc]              (p6)  {Submit Outsourcing Request};
\node [proc]              (p7)  {Decrease $R$};
\node [test, join]        (t8)  {$R < R_{base}$?};
\node [term]              (m2)  {Eliminate};

\node [test, left=of c3]  (t2)  {All Tasks Accomplished?};
\node [proc]              (p2)  {Process Task};
\node [test]              (t3)  {Accepted?};
\node [proc]              (p3)  {Pass the Outsouring Task};
\node [test]              (t4)  {Task Accomplished?};

\node [coord, left=of p1](c2)  {}; \cmark{2}

\draw [->, lcnorm] (p4) |- (c1);
\draw [->, lcnorm] (p8) -- (t1);
\draw [->, lcnorm] (p1) --(c2) |- (c1);

\path (t1.east)   to node [near start, yshift=1em] {$n$} (p4);
\path (t1.south) to node [near start, xshift=1em] {$y$} (p1);
  \draw [o->, lcnorm] (t1.east) -- (p4);
  \draw [*->, lcnorm] (t1.south) |- (p5);

\end{tikzpicture}}\label{fig:userrank}}
\caption{ Primary Flow Chart of Order and User}
\end{figure}
\subsubsection{User Rank}
Apart from the above mentioned order related behavior, User can also submit resources, change standrd, make recommandation and so on, according to the permission degree the User has. User Rank provides the basis to set the permission degree, it gets the initial value when User was introduced in the system and it will increase or decrease by good or bad performance in each transaction, the follow equation shows a simple way to evaluate the User Rank. 

\begin{equation}
  \begin{cases}
      Rank_{[T+1]} = Rank_{[T]} + Gain - Lose, & T \geqslant 1\\
      Rank_{[0]} = Rank_{[init]}
  \end{cases}
\label{equ:simplerank}
\end{equation}
Subscript $T$ in \eqref{equ:simplerank} counts the transaction User takes. 

User Rank will be adjusted all the time only if the User was eliminated for a low rank value, the adjustment was based on the evalution messages, which seller and buyer would post after each transaction, basically. Moreover, the adjustment of rank will be conducted through outsourcing pipline, which was formed when a task be outsourced many times. Introduction and elimination of User implies the evolution of the Cloud Manufacturing Ecosystem. Advanced techniques will be discuessed in \autoref{sub:rating_schema}.

\subsubsection{Platform}
As related roles mentioned above, Administrators are different for all the actions they can take limited within the platform. Administrators help mantain the platform with main functions like reforme orders/resources into tasks/service, introduce/eliminate Users, give permissions to Users, hold resources search/recommand engine and so on.

Platform itself can be viewed as a control pannel that administrator can tune permissions according to user's rank. 

\subsection{Data Science Tools}
Along with the convenience platform provides, there comes massive amounts of data Administrators must to deal with, like:
\begin{inparaenum}[1)]
  \item Transaction histroy;
  \item Ranking histroy;
  \item Status of resources, services, orders and tasks;
  \item Decision pattern and so on.
\end{inparaenum}

Dealing with data is the everyday life to Data Scientists, administrator of platform can use their daily tools to do the data-works. Data Science Tools can be categorized with:

%\renewcommand{\labelenumi}{\theenmui}
\begin{asparadesc}
\item[Data Analysis] The R Project for Statistical Computing is still the most popular. I agree with William that it's best used through R Studio.
Pandas is a set of Python libraries if you don't want to learn a new language
The Julia Language is an upcoming alternative to R. I spent a little bit of time learning it and would like to keep track of where the project goes.
\item[Data Warehousing] MySQL: It can comfortably handle datasets that are a few GBs. Don't prematurely go to Hive. MySQL is optimized to death and is super good at latency for running ad-hoc queries.
CSV Files: You'd be surprised how far you can get with using these as your primary storage.
Hive/Shark/Redshift: For when your big data is actually big. Hive can do giant joins while Redshift is better for latency but more limited in its joins.
Data Visualization
D3.js for pretty visualizations to put on the web
Matplotlib for ad-hoc Python plotting
ggplot2 for R.
\item[Machine Learning] I'm a little rusty on this one - the state of the art is moving quickly here.
I've mostly trained models in R and used it directly or over Hive.
I've played around with scikit-learn in the past and it seems to be maturing.
I've used Weka for trying out standard algorithms quickly on new datasets in the past but its hard to productionize anything with it.
\item[Social Network Analysis] It's been a while since I've done hands-on SNA, so things might have changed recently.
NetworkX is a pretty good Python library for SNA but isn't distributed
Stanford Network Analysis Project
Apache Giraph is an open-source implementation of the Google Pregel paper. 
\end{asparadesc}

Data Science Tools can help mine patterns, status, quality and so like of users, this latent imformation will play a important role in improving the mode of Cloud Manufacturing.
