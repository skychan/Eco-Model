% !TEX root = Eco-Model.tex
\section{Preliminaries} % (fold)
 
\subsection{Services and Tasks}
Resources and Services, Orders and Tasks,
Similary to the relation of service and resource, order and task 

transaction

\subsection{Roles}
Most recent studies like Wu et al.\cite{Wu2013} described the roles in Cloud Manufacturing as a tri-group model with:\begin{inparaenum}[1)]
\item Users/Customers,
\item Application providers and
\item Physical resource providers.
\end{inparaenum}
However, the roles in real world always not go like this model, since it parted roles only by their functions. While, users could get different function roles depend on what transaction they take, we should make some modifiacations on this model.

Before the emendation of the roles, some related standard name should be taken:
\begin{asparadesc}
\item[User] The main role in Cloud Manufacturing, which is combined of Seller and Buyer as shown in \tabref{tab:categoryuser}, usually acts as Seller or Buyer according to the specific transaction;
\item[Seller] The temporary role in a transaction, which provides the allocated part of manufacturing service and makes a evalution about the Buyer after the transaction;
\item[Buyer] The temporary role in a transaction, which submits the allocated part of manufacturing task and makes a evalution about the Seller after the transaction;
\item[Administrator] The role that gets the work for the operation and maintenance of the platform, has indirectly impact on the Ecosystem.
\end{asparadesc}
\begin{table}[htbp]
  \centering\small
  \caption{Category of Users}
    \begin{tabularx}{\textwidth}{llccX}
    \toprule
        &  & Seller & Buyer & Description \\
    \midrule
    Generalized & User & $\surd$ & $\surd$ & Act as Seller or Buyer according to the specific transaction. \\
    Enterprise & User & $\surd$ & &  Provides the allocated part of manufacturing services and makes a evalution about the Buyer after the transaction. \\
    Customer & User& & $\surd$ &  Submits the allocated part of manufacturing tasks and makes a evalution about the Seller after the transaction.\\
    \bottomrule
    \end{tabularx}%
  \label{tab:categoryuser}%
\end{table}%
As defined above, Enterprise User is the special case of the User(short for Generalized User) that purely provides manufacturing services and the Customer User is similarly defined.

Hence, there is only two roles in the Ecosystem worth to be studied: User and Administrator, the proportion of user type could be adjusted as the runtime of the system goes by.

\subsection{Operation Mode of Ecosystem}
The style the Cloud Manufacturing Ecosystem appears mainly depends on the way the operation mode takes, therefore, research on features and issues in the Ecosystem would be taken within the scope of operation mode we are going to set up. As shown in \figref{fig:platformstruct}, the principal struct of Ecosystem can be described into four main branch, how these branches grow determines how the Operation Mode will come out, and it will be easy to figure out the specified mode we make with some flow charts.

\begin{figure}[!h]
\centering
%\resizebox{0.7\textwidth}{!}{% !TEX root = ..\Eco-Model.tex
\begin{tikzpicture}
[mindmap,
every node/.style={concept, text=white, execute at begin node=\hskip0pt,circular drop shadow},
root concept/.append style={concept color=black, fill=white, line width=1ex, text=black, font=\Large\scshape},
grow cyclic,
level 1/.append style={level distance=4.5cm,sibling angle=90, font=\scshape},
level 2/.append style={level distance=3cm,sibling angle=45, font=\scriptsize\bf},
level 3/.append style={level distance=2cm,sibling angle=60, font=\scriptsize\bf},
level 4/.append style={level distance=1cm,sibling angle=60, font=\tiny\bf}]
\node [root concept] {Cloud Manufacturing Ecosystem}  % root
child [concept color = gray]{ node {Schedule}
child { node {Service Standardization}}
child { node {Decision Making Support}}
child { node {Platform Assistance}}
child { node {Match Regualtion}}
}
child [concept color = gray]{ node {User}
  child { node {Submit Order}}
  child { node {Provide Resource}}
  child { node {Make Transaction}}
  child { node {Evalute Others}}
  }
child [concept color = gray]{ node {Envolve}
  child { node {Optimization}
    child { node {Manufacturing}}
    child { node {Industrial Chain}}
    }
  child { node {Member}
    child { node {Introduce}}
    child { node {Eliminate}}
    }
  child { node {Adjust User Proportion}
    child { node {Rating}}
    child { node {Recommand}}
    }
  child { node {Development Guidance}}
  }
child [concept color = gray]{ node {Platform}
  child { node {Authority Control}}
  child { node {Transaction Bridge}}
  child { node {Dashbord}
    child { node {Service}}
    child { node {Task}}
    child { node {User}}
    }
child { node {Administrate}}
  child { node {Hub}
    child { node {Service}}
    child { node {Resource}}
    child { node {Ranking}}
    }
  child { node {Scheduling Assistance}}
}
;
\end{tikzpicture}}
\caption{Principal Struct of Cloud Manufacturing Ecosystem}
\label{fig:platformstruct}
\end{figure}

Because the metioned system is too complex to analyse in every detail, we focus on two important operation flows -- Order Flow and User Rank -- to help illustrate the Operation Mode. 

\subsubsection{Order Flow}
Each manufacturing order (order for short) appears when Users make transaction in the platform, it might then go through some different modules and a simple version of its flow was shown on \figref{fig:orderflow}.

If an order accomplished smoothly, it will be constructed into tasks pieces in order to match the appropriate manufacturing services usually, similar to a large dataset processed with MapReduce\cite{Dean2008} schema. Then each piece could be processed by distributed resources with the outsourcing strategy User makes.

\begin{figure}[!h]
\centering\small
%\subfloat[Order Flow]{\resizebox{0.45\textwidth}{!}{% !TEX root = Eco-Model.tex
\begin{tikzpicture}[%
    >=triangle 60,              % Nice arrows; your taste may be different
    start chain=going below,    % General flow is top-to-bottom
    node distance=6mm and 60mm, % Global setup of box spacing
    every join/.style={norm},   % Default linetype for connecting boxes
    ]
% ------------------------------------------------- 
% A few box styles 
% <on chain> *and* <on grid> reduce the need for manual relative
% positioning of nodes
\tikzset{
  base/.style={draw, on chain, on grid, align=center, minimum height=4ex},
  proc/.style={base, rectangle, text width=8em},
  test/.style={base, diamond, aspect=2, text width=5em},
  term/.style={proc, rounded corners},
  % coord node style is used for placing corners of connecting lines
  coord/.style={coordinate, on chain, on grid, node distance=6mm and 25mm},
  sky/.style={coordinate, on chain, on grid, node distance=6mm and 60mm},
  % nmark node style is used for coordinate debugging marks
  nmark/.style={draw, cyan, circle, font={\sffamily\bfseries}},
  % -------------------------------------------------
  % Connector line styles for different parts of the diagram
  norm/.style={->, draw, lcnorm},
  free/.style={->, draw, lcfree},
  cong/.style={->, draw, lccong},
  it/.style={font={\small\itshape}}
}
% -------------------------------------------------
% Start by placing the nodes
\node [term]              {Be Submitted};
\node [proc, join]        {Constrcut into Tasks};
\node [proc, join]  (p1)  {Evaluate the Seller};
\node [test, join]  (t1)  {Transact?};
\node [test]        (t2)  {Task Remain?};
\node [test]        (t3)  {Outsourcing?};
\node [proc]        (p2)  {Process Task and Submit Result};
\node [test, join]  (t4)  {Order Finished?};
\node [term]        (m1)  {End Smothly};

\node [sky, right=of p1] (c5) {};%\cmark{5}
\node [proc, right=of t1] (p3)  {Select Next Seller};
\node [test]        (t5)  {Partly or Whole?};
\node [proc]        (p5)  {Process Un-Outsouring Part and Submit Result};
\node [test, join]  (t6)  {Process Smothly?};
\node [proc]        (p4)  {Submit Outsourcing Part};
\node [coord, yshift=2.5mm]       (c4)  {}; %\cmark{4}
\node [term, right=of m1] (m2)  {End with Error};

\node [coord, left=of t2] (c1) {}; %\cmark{1}
\node [coord, right=of t3](c2) {}; %\cmark{2}
\node [coord, right=of t4](c3) {}; %\cmark{3}
\node [coord, right=of t5, xshift=3mm](c6) {}; %\cmark{6}
\node [coord, left=of c4, xshift=-1mm](c8) {}; %\cmark{8}

\path (t1.east) to node [near start, yshift=1em] {$n$} (p3);
  \draw [o->, lcnorm] (t1.east) -- (p3);
  \draw [->, lcnorm] (p3) |- (c5);
\path (t1.south) to node [near start, xshift=1em] {$y$} (t2);
  \draw [*->, lcnorm] (t1.south) -- (t2);
\path (t2.west) to node [near start, yshift=1em]  {$n$} (p2.west);
  \draw [o->, lcnorm] (t2.west) -- (c1) |- (p2.west);
\path (t2.south) to node [near start, xshift=1em] {$y$} (t3);
  \draw [*->, lcnorm] (t2.south)  -- (t3);
\path (t3.east) to node [near start, yshift=1em] {$y$}  (c2);
  \draw [*->, lcnorm] (t3.east) -| (c2) |- (t5.west);
\path (t3.south) to node [near start, xshift=1em] {$n$} (p2);
  \draw [o->, lcnorm] (t3.south) -- (p2);
\path (t4.south) to node [near start, xshift=1em] {$y$} (m1);
  \draw [*->, lcnorm] (t4.south) -- (m1);
\path (t4.east) to node [near start, yshift=-1em] {$n$} (c3);
  \draw [o->, lcnorm] (t4.east) -- (c3) |- (c4) -- (m2);
\path (t5.east) to node [near start, yshift=1em] {whole} (c6);
\path (t5.south) to node [near start, xshift=2em] {partly} (p5);
  \draw [o->, lcnorm] (t5.east) -- (c6) |- (p1);
  \draw [*->, lcnorm] (t5.south) -- (p5);
\path (t6.south) to node [near start, xshift=1em] {$y$} (p4);
\path (t6.west) to node [near start, yshift=1em] {$n$} (c8);
  \draw [o->, lcnorm] (t6.west) -| (c8); 
  \draw [*->, lcnorm] (t6.south) -- (p4);
  \draw [->, lcnorm] (p4.east) -| (c6);
\end{tikzpicture}}\label{fig:orderflow}} \hspace{0.09\textwidth}
%\subfloat[User Rank]{\resizebox{0.45\textwidth}{!}{% !TEX root = ../Eco-Model.tex
\begin{tikzpicture}[%
    >=triangle 60,              % Nice arrows; your taste may be different
    start chain=going below,    % General flow is top-to-bottom
    node distance=6mm and 60mm, % Global setup of box spacing
    every join/.style={norm},   % Default linetype for connecting boxes
    ]
% ------------------------------------------------- 
% A few box styles 
% <on chain> *and* <on grid> reduce the need for manual relative
% positioning of nodes
\tikzset{
  base/.style={draw, on chain, on grid, align=center, minimum height=4ex},
  proc/.style={base, rectangle, text width=8em},
  test/.style={base, diamond, aspect=2, text width=5em},
  term/.style={proc, rounded corners},
  % coord node style is used for placing corners of connecting lines
  coord/.style={coordinate, on chain, on grid, node distance=6mm and 25mm},
  % nmark node style is used for coordinate debugging marks
  nmark/.style={draw, cyan, circle, font={\sffamily\bfseries}},
  % -------------------------------------------------
  % Connector line styles for different parts of the diagram
  norm/.style={->, draw, lcnorm},
  free/.style={->, draw, lcfree},
  cong/.style={->, draw, lccong},
  it/.style={font={\small\itshape}}
}
% -------------------------------------------------
% Start by placing the nodes
\node [term, fill = lcfree!25]              (m1)  {Be Introduced};
\node [proc, join]        (p8)  {Get Initial Rank $R$};
\node [coord, yshift=2mm] (c1)  {}; %\cmark{1}
\node [test, yshift=2mm]  (t1)  {Transact?};
\node [proc, yshift=-2mm] (p1)  {Increase $R$};

\node [coord, yshift=0mm] (c7)  {}; %\cmark{7}

\node [test, yshift=0mm] (t2)  {All Tasks Done?};
\node [proc]              (p2)  {Process Un-Outsourceing Task};
\node [test]              (t3)  {Accepted?};
\node [proc]              (p3)  {Pass the Outsouring Task};
\node [test, join]        (t4)  {Task Done?};

\node [proc, right=of t1] (p4)  {Wait};
\node [proc]              (p5)  {Compete to Get Task};
\node [test, join]        (t5)  {Make it?};

\node [coord, yshift=2mm] (c4)  {}; %\cmark{4}

\node [test, yshift=2mm]  (t6)  {Outsourcing?};
\node [test]              (t7)  {Partly or Whole?};
\node [proc]              (p6)  {Submit Outsourcing Request};
\node [proc]              (p7)  {Decrease $R$};
\node [test, join]        (t8)  {$R < R_{base}$?};
\node [term, fill=lccong!25]              (m2)  {Eliminate};

\node [coord, left=of p1] (c2)  {}; %\cmark{2}
\node [coord, right=of t5](c3)  {}; %\cmark{3}
\node [coord, right=of p2](c5)  {}; %\cmark{5}
\node [coord, right=of t3](c6)  {}; %\cmark{6}
\node [coord, left=of c7] (c8)  {}; %\cmark{8}
\node [coord, right=of t4](c9)  {}; %\cmark{9}

\draw [->, lcnorm] (p4) |- (c1);
\draw [->, lcnorm] (p8) -- (t1);
\draw [->, lcnorm] (p1) --(c2) |- (c1);
\draw [->, lcnorm] (p2) -- (t2);
\draw [->, lcnorm] (p6) -| (c6) -- (t3);

\path (t1.east)   to node [near start, yshift=1em] {$n$} (p4);
\path (t1.south)  to node [near start, xshift=1em] {$y$} (p1);
  \draw [o->, lcnorm] (t1.east) -- (p4);
  \draw [*->, lcnorm] (t1.south) |- (p5);
\path (t2.east)   to node [near start, yshift=-1em]{$n$} ++(2mm,0);
\path (t2.north)  to node [near start, xshift=1em] {$y$} (p1);
  \draw [o->, lcnorm] (t2.east) -- (c4) ;
  \draw [*->, lcnorm] (t2.north) -- (p1);
\path (t5.east)   to node [near start, yshift=1em] {$n$} (c3);
\path (t5.south)  to node [near start, xshift=1em] {$y$} (t6);
  \draw [o->, lcnorm] (t5.east) -- (c3) |- (p4);
  \draw [*->, lcnorm] (t5.south) -- (t6);
\path (t6.west)   to node [near start, yshift=-1em]{$n$} (c5);
\path (t6.south)  to node [near start, xshift=1em] {$y$} (t7);
  \draw [*->, lcnorm] (t6.south) -- (t7);
  \draw [o->, lcnorm] (t6.west) -| (c5) -- (p2);
\path (t7.west)   to node [near start, yshift=1em] {partly} ++(-1mm,0);
\path (t7.south)  to node [near start, xshift=2em] {whole} (p6);
  \draw [*->, lcnorm] (t7.west) -| (c5);
  \draw [o->, lcnorm] (t7.south) -- (p6);
\path (t8.east)   to node [near start, yshift=1em] {$n$} ++(1mm,0);
\path (t8.south)  to node [near start, xshift=1em] {$y$} (m2);
 \draw [o->, lcnorm] (t8.east) -| (c3);
 \draw [*->, lcnorm] (t8.south) -- (m2);
\path (t3.north)  to node [near start, xshift=1em] {$n$} (p2);
\path (t3.south)  to node [near start, xshift=1em] {$y$} (p3);
  \draw [o->, lcnorm] (t3.north) -- (p2);
  \draw [*->, lcnorm] (t3.south) -- (p3);
\path (t4.west)   to node [near start, yshift=1em] {$y$} ++(-1mm,0);
\path (t4.east)   to node [near start, yshift=1em] {$n$} (c9);
  \draw [*->, lcnorm] (t4.west) -| (c8) -- (c7);
  \draw [o->, lcnorm] (t4.east) -- (c9) |- (p7);
\end{tikzpicture}}\label{fig:userrank}}
\caption{ Primary Flow Chart of Order and User}
\end{figure}
\subsubsection{User Rank}
Apart from the above mentioned order related behavior, User can also submit resources, change standrd, make recommandation and so on, according to the permission degree the User has. User Rank provides the basis to set the permission degree, it gets the initial value when User was introduced in the system and it will increase or decrease by good or bad performance in each transaction, a simple flow of User Rank was shown on \figref{fig:userrank}.

User Rank will be adjusted all the time only if the User was eliminated for a low rank value, the adjustment was based on the evalution messages, which seller and buyer would post after each transaction, basically. Introduction and elimination of User implies the evolution of the Cloud Manufacturing Ecosystem. 

\subsubsection{Platform}
Need a picture with cms run pic

provide the method of order - > task , resources -> service
gain the evalution messages for analysis
decide to introduce or eliminate.

\subsection{Data Science Tools}


simply descript some related tools and where to use.
