% !TEX root = Eco-Model.tex
\section{Background} % (fold)
\label{sec:background}

As an application of networked manufacturing, cloud manufacturing proposed by ...

(problems background to form the ecosystem)

\subsection{Cloud manufacturing} % (fold)
\label{sub:cloud_manufacturing}

\subsubsection{Cloud manufacturing architecture}

Combined with the background of the research problem and other scholars' previous study in cloud manufacturing, we proposed an improved framework that suits for cloud manufacturing environment as shown in \autoref{fig:structure}.
\begin{figure}[htbp]
\centering
\includegraphics[scale = .6, trim = 0 15 0 35]{Cloud_Mfg_Structure.pdf}
\caption{Cloud manufacturing architecture with main flows}
\label{fig:structure}
\end{figure}

The architecture of cloud manufacturing is mainly composed of three parts, namely, \begin{inparaenum}[1)]
\item user,
\item platform and
\item physical base
\end{inparaenum}, that are connected by material or information flows. A simple version of conception for cloud manufacturing architecture we study here can be described as follows:
\begin{compactdesc}
\item [User] part describes the generalized role who takes part in the trading activities in the cloud manufacturing system, such as manufacturing service provider and manufacturing service demander, whom can be called as provider and demander for short respectively, from the functional point of view. Meanwhile, these part can also be classified into enterprise user and customer, from the practical point of view. In addition, it's possible for user to act as both demander and provider simultaneously if administrator give the authority to the user, so we can categorize the user in 3 basic types as described in \autoref{fig:usertype} and \autoref{tab:usertype}. Both functional and practical are important perspective for later analysis in following sections.
\item [Platform] is operating with the control of administrator and this part consists of 3 main layers namely
	\begin{inparaenum}[1)]
	\item application/interface layer,
	\item service layer and
	\item resource \& production layer
	\end{inparaenum}.
Service is the core concept in cloud manufacturing, so service layer is the core in the cloud manufacturing platform. With the help of related technology in servitization, manufacturing resource and production can be encapsulated into services, then these services can be acquired by users with the well designed applications or interfaces.
\item [Physical Base] part includes the basic infrastructure, resource and production. It's worth mentioning that part of resources and production can even be virtualized with some services the platform provides.
\end{compactdesc}


\begin{figure}[htbp]
	\centering
	\includegraphics[width=.7\textwidth]{usertype.pdf}
	\caption{Simple explanation of user type}
	\label{fig:usertype}
\end{figure}

\begin{table}[htbp]
	\caption{Simple explanation of user type}
	\label{tab:usertype}
	\centering

	\begin{tabularx}{\textwidth}{llX}
	\toprule
	\textbf{Type} & \textbf{Name} &  \textbf{Concise Description}\\
	\midrule
	I			& Pure demander &  Submit manufacturing tasks and make evaluation of provider.\\
	II 			& Complex user 	&  Usually a pure provider submit part of its task, it becomes the complex user and changes the operation model of the user.\\
	III 		& Pure provider &  Provide manufacturing services and makes evaluation demander.\\
	\bottomrule
	\end{tabularx}
\end{table}
% Before the emendation of the roles, some related standard name should be taken:

As defined above, Enterprise User is the special case of the User(short for Generalized User) that purely provides manufacturing services and the Customer User is similarly defined.

Hence, there is only two roles in the Ecosystem worth to be studied: User and Administrator, the proportion of user type could be adjusted as the runtime of the system goes by.

\begin{figure}[htbp]
\centering
\includegraphics[scale = .6, trim = 0 15 0 10]{supplementary.pdf}
\caption{Supplementary illustration for \autoref{fig:structure}}
\label{fig:supplementary}
\end{figure}

With the supplementary illustration of \autoref{fig:supplementary}, which categorizes the main applications in cloud manufacturing environment with the fore-mentioned flows that marked by circled number (i.e. \textcircled{\small{1}}) and corresponding to the same symbol in \autoref{fig:structure}, we can simply describe the main activities within the cloud manufacturing system. In \autoref{fig:supplementary}, we denote provider, demander, platform and physical base as \textbf{P}, \textbf{D}, \textbf{PF} and \textbf{PB} respectively. The overlap among application districts implies the complexity of the system, so that we will analyze these applications coordinately from a systematic view.

\subsubsection{Operation model of platform}
\label{ssub:operation_model_of_platform}
Platform carries most of the activities illustrated in \autoref{fig:supplementary}, so it's important to design an operation model for administrator to manage these activities. This model enables modules that appeared in xxx such as \textit{service packager}, \textit{task decomposer} , \textit{service search engine}, etc., but we only study the modules in for they're the core in the evolutionary process in this paper.

\begin{table}[tb]
	\caption{Core modules in operation model of platform}
	\label{tab:core_module_in_platform}
	\centering

	\begin{tabularx}{\textwidth}{llX}
	\toprule

	\textbf{Module} & \multicolumn{2}{l}{\textbf{Function with concise description}}  \\
	\midrule
	\multirow{2}*{Population controller}		& Adder: 		& To introduce resource provider into the system\\
												& Subtracter: 	& To eliminate resource provider from the system\\
	\hline
	\multirow{2}*{Authority tuner}				& Raiser:	& Raise the authority value of the provider \\
												& Reducer: & Reduce the authority value of the provider\\
	\hline
	\multirow{3}*{Recommender}					& Matcher: & Match task with a bunch of alternative services \\
												& Adjuster: & Change provider's rank based on history\\
												& Promoter: & To promote fresh resource provider \\
	\bottomrule
	\end{tabularx}
\end{table}

\subsubsection{Operation model of users}
\label{subs:Operation model of users}
business model is emerging as a new unit of analysis; business models emphasize a system-level, holistic approach to explaining how firms do business; firm activities play an important role in the various conceptualizations of business models that have been proposed; business models seek to explain how value is created, not just how it is captured.
\begin{figure}[htbp]
\centering
\subfloat[Demander]{\includegraphics[width=0.8\textwidth]{figures/demander_main_process}\label{fig:demanderprocess}}\\
\subfloat[Provider]{\includegraphics[width=0.8\textwidth]{figures/provider_main_process}\label{fig:providerprocess}}
\caption{Main operational processes}
\end{figure}

\begin{figure}[!h]
\centering\small
% \subfloat[Order Flow]{\resizebox{0.45\textwidth}{!}{% !TEX root = Eco-Model.tex
\begin{tikzpicture}[%
    >=triangle 60,              % Nice arrows; your taste may be different
    start chain=going below,    % General flow is top-to-bottom
    node distance=6mm and 60mm, % Global setup of box spacing
    every join/.style={norm},   % Default linetype for connecting boxes
    ]
% ------------------------------------------------- 
% A few box styles 
% <on chain> *and* <on grid> reduce the need for manual relative
% positioning of nodes
\tikzset{
  base/.style={draw, on chain, on grid, align=center, minimum height=4ex},
  proc/.style={base, rectangle, text width=8em},
  test/.style={base, diamond, aspect=2, text width=5em},
  term/.style={proc, rounded corners},
  % coord node style is used for placing corners of connecting lines
  coord/.style={coordinate, on chain, on grid, node distance=6mm and 25mm},
  sky/.style={coordinate, on chain, on grid, node distance=6mm and 60mm},
  % nmark node style is used for coordinate debugging marks
  nmark/.style={draw, cyan, circle, font={\sffamily\bfseries}},
  % -------------------------------------------------
  % Connector line styles for different parts of the diagram
  norm/.style={->, draw, lcnorm},
  free/.style={->, draw, lcfree},
  cong/.style={->, draw, lccong},
  it/.style={font={\small\itshape}}
}
% -------------------------------------------------
% Start by placing the nodes
\node [term]              {Be Submitted};
\node [proc, join]        {Constrcut into Tasks};
\node [proc, join]  (p1)  {Evaluate the Seller};
\node [test, join]  (t1)  {Transact?};
\node [test]        (t2)  {Task Remain?};
\node [test]        (t3)  {Outsourcing?};
\node [proc]        (p2)  {Process Task and Submit Result};
\node [test, join]  (t4)  {Order Finished?};
\node [term]        (m1)  {End Smothly};

\node [sky, right=of p1] (c5) {};%\cmark{5}
\node [proc, right=of t1] (p3)  {Select Next Seller};
\node [test]        (t5)  {Partly or Whole?};
\node [proc]        (p5)  {Process Un-Outsouring Part and Submit Result};
\node [test, join]  (t6)  {Process Smothly?};
\node [proc]        (p4)  {Submit Outsourcing Part};
\node [coord, yshift=2.5mm]       (c4)  {}; %\cmark{4}
\node [term, right=of m1] (m2)  {End with Error};

\node [coord, left=of t2] (c1) {}; %\cmark{1}
\node [coord, right=of t3](c2) {}; %\cmark{2}
\node [coord, right=of t4](c3) {}; %\cmark{3}
\node [coord, right=of t5, xshift=3mm](c6) {}; %\cmark{6}
\node [coord, left=of c4, xshift=-1mm](c8) {}; %\cmark{8}

\path (t1.east) to node [near start, yshift=1em] {$n$} (p3);
  \draw [o->, lcnorm] (t1.east) -- (p3);
  \draw [->, lcnorm] (p3) |- (c5);
\path (t1.south) to node [near start, xshift=1em] {$y$} (t2);
  \draw [*->, lcnorm] (t1.south) -- (t2);
\path (t2.west) to node [near start, yshift=1em]  {$n$} (p2.west);
  \draw [o->, lcnorm] (t2.west) -- (c1) |- (p2.west);
\path (t2.south) to node [near start, xshift=1em] {$y$} (t3);
  \draw [*->, lcnorm] (t2.south)  -- (t3);
\path (t3.east) to node [near start, yshift=1em] {$y$}  (c2);
  \draw [*->, lcnorm] (t3.east) -| (c2) |- (t5.west);
\path (t3.south) to node [near start, xshift=1em] {$n$} (p2);
  \draw [o->, lcnorm] (t3.south) -- (p2);
\path (t4.south) to node [near start, xshift=1em] {$y$} (m1);
  \draw [*->, lcnorm] (t4.south) -- (m1);
\path (t4.east) to node [near start, yshift=-1em] {$n$} (c3);
  \draw [o->, lcnorm] (t4.east) -- (c3) |- (c4) -- (m2);
\path (t5.east) to node [near start, yshift=1em] {whole} (c6);
\path (t5.south) to node [near start, xshift=2em] {partly} (p5);
  \draw [o->, lcnorm] (t5.east) -- (c6) |- (p1);
  \draw [*->, lcnorm] (t5.south) -- (p5);
\path (t6.south) to node [near start, xshift=1em] {$y$} (p4);
\path (t6.west) to node [near start, yshift=1em] {$n$} (c8);
  \draw [o->, lcnorm] (t6.west) -| (c8); 
  \draw [*->, lcnorm] (t6.south) -- (p4);
  \draw [->, lcnorm] (p4.east) -| (c6);
\end{tikzpicture}}\label{fig:orderflow}} \hspace{0.09\textwidth}
% \subfloat[User Rank]{\resizebox{0.45\textwidth}{!}{% !TEX root = ../Eco-Model.tex
\begin{tikzpicture}[%
    >=triangle 60,              % Nice arrows; your taste may be different
    start chain=going below,    % General flow is top-to-bottom
    node distance=6mm and 60mm, % Global setup of box spacing
    every join/.style={norm},   % Default linetype for connecting boxes
    ]
% ------------------------------------------------- 
% A few box styles 
% <on chain> *and* <on grid> reduce the need for manual relative
% positioning of nodes
\tikzset{
  base/.style={draw, on chain, on grid, align=center, minimum height=4ex},
  proc/.style={base, rectangle, text width=8em},
  test/.style={base, diamond, aspect=2, text width=5em},
  term/.style={proc, rounded corners},
  % coord node style is used for placing corners of connecting lines
  coord/.style={coordinate, on chain, on grid, node distance=6mm and 25mm},
  % nmark node style is used for coordinate debugging marks
  nmark/.style={draw, cyan, circle, font={\sffamily\bfseries}},
  % -------------------------------------------------
  % Connector line styles for different parts of the diagram
  norm/.style={->, draw, lcnorm},
  free/.style={->, draw, lcfree},
  cong/.style={->, draw, lccong},
  it/.style={font={\small\itshape}}
}
% -------------------------------------------------
% Start by placing the nodes
\node [term, fill = lcfree!25]              (m1)  {Be Introduced};
\node [proc, join]        (p8)  {Get Initial Rank $R$};
\node [coord, yshift=2mm] (c1)  {}; %\cmark{1}
\node [test, yshift=2mm]  (t1)  {Transact?};
\node [proc, yshift=-2mm] (p1)  {Increase $R$};

\node [coord, yshift=0mm] (c7)  {}; %\cmark{7}

\node [test, yshift=0mm] (t2)  {All Tasks Done?};
\node [proc]              (p2)  {Process Un-Outsourceing Task};
\node [test]              (t3)  {Accepted?};
\node [proc]              (p3)  {Pass the Outsouring Task};
\node [test, join]        (t4)  {Task Done?};

\node [proc, right=of t1] (p4)  {Wait};
\node [proc]              (p5)  {Compete to Get Task};
\node [test, join]        (t5)  {Make it?};

\node [coord, yshift=2mm] (c4)  {}; %\cmark{4}

\node [test, yshift=2mm]  (t6)  {Outsourcing?};
\node [test]              (t7)  {Partly or Whole?};
\node [proc]              (p6)  {Submit Outsourcing Request};
\node [proc]              (p7)  {Decrease $R$};
\node [test, join]        (t8)  {$R < R_{base}$?};
\node [term, fill=lccong!25]              (m2)  {Eliminate};

\node [coord, left=of p1] (c2)  {}; %\cmark{2}
\node [coord, right=of t5](c3)  {}; %\cmark{3}
\node [coord, right=of p2](c5)  {}; %\cmark{5}
\node [coord, right=of t3](c6)  {}; %\cmark{6}
\node [coord, left=of c7] (c8)  {}; %\cmark{8}
\node [coord, right=of t4](c9)  {}; %\cmark{9}

\draw [->, lcnorm] (p4) |- (c1);
\draw [->, lcnorm] (p8) -- (t1);
\draw [->, lcnorm] (p1) --(c2) |- (c1);
\draw [->, lcnorm] (p2) -- (t2);
\draw [->, lcnorm] (p6) -| (c6) -- (t3);

\path (t1.east)   to node [near start, yshift=1em] {$n$} (p4);
\path (t1.south)  to node [near start, xshift=1em] {$y$} (p1);
  \draw [o->, lcnorm] (t1.east) -- (p4);
  \draw [*->, lcnorm] (t1.south) |- (p5);
\path (t2.east)   to node [near start, yshift=-1em]{$n$} ++(2mm,0);
\path (t2.north)  to node [near start, xshift=1em] {$y$} (p1);
  \draw [o->, lcnorm] (t2.east) -- (c4) ;
  \draw [*->, lcnorm] (t2.north) -- (p1);
\path (t5.east)   to node [near start, yshift=1em] {$n$} (c3);
\path (t5.south)  to node [near start, xshift=1em] {$y$} (t6);
  \draw [o->, lcnorm] (t5.east) -- (c3) |- (p4);
  \draw [*->, lcnorm] (t5.south) -- (t6);
\path (t6.west)   to node [near start, yshift=-1em]{$n$} (c5);
\path (t6.south)  to node [near start, xshift=1em] {$y$} (t7);
  \draw [*->, lcnorm] (t6.south) -- (t7);
  \draw [o->, lcnorm] (t6.west) -| (c5) -- (p2);
\path (t7.west)   to node [near start, yshift=1em] {partly} ++(-1mm,0);
\path (t7.south)  to node [near start, xshift=2em] {whole} (p6);
  \draw [*->, lcnorm] (t7.west) -| (c5);
  \draw [o->, lcnorm] (t7.south) -- (p6);
\path (t8.east)   to node [near start, yshift=1em] {$n$} ++(1mm,0);
\path (t8.south)  to node [near start, xshift=1em] {$y$} (m2);
 \draw [o->, lcnorm] (t8.east) -| (c3);
 \draw [*->, lcnorm] (t8.south) -- (m2);
\path (t3.north)  to node [near start, xshift=1em] {$n$} (p2);
\path (t3.south)  to node [near start, xshift=1em] {$y$} (p3);
  \draw [o->, lcnorm] (t3.north) -- (p2);
  \draw [*->, lcnorm] (t3.south) -- (p3);
\path (t4.west)   to node [near start, yshift=1em] {$y$} ++(-1mm,0);
\path (t4.east)   to node [near start, yshift=1em] {$n$} (c9);
  \draw [*->, lcnorm] (t4.west) -| (c8) -- (c7);
  \draw [o->, lcnorm] (t4.east) -- (c9) |- (p7);
\end{tikzpicture}}\label{fig:userrank}}
\resizebox{0.9\textwidth}{!}{% !TEX root = ../Eco-Model.tex
\begin{tikzpicture}[%
    >=triangle 60,              % Nice arrows; your taste may be different
    start chain=going right,    % General flow is top-to-bottom
    node distance=6mm and 10mm, % Global setup of box spacing
    every join/.style={norm},   % Default linetype for connecting boxes
    ]
% ------------------------------------------------- 
% A few box styles 
% <on chain> *and* <on grid> reduce the need for manual relative
% positioning of nodes
\tikzset{
  base/.style={draw, on chain, on grid, align=center, rectangle},
  proc/.style={base, text width=6em, minimum height=8em},
  term/.style={base, text width=5em, rounded corners, minimum height = 3ex},
  dproc/.style={proc, minimum height = 20ex},
  gathe/.style={draw, on chain, on grid, align=center, circle, minimum height=4ex, text width=0.5em, node distance=10mm and 8mm},
  % coord node style is used for placing corners of connecting lines
  coord/.style={coordinate, on chain, on grid, node distance=6mm and 25mm},
  sky/.style={coordinate, on chain, on grid, node distance=6mm and 60mm},
  % nmark node style is used for coordinate debugging marks
  nmark/.style={draw, cyan, circle, font={\sffamily\bfseries}},
  % -------------------------------------------------
  % Connector line styles for different parts of the diagram
  norm/.style={->, draw, lcnorm},
  free/.style={->, draw, lcfree},
  cong/.style={->, draw, lccong},
  it/.style={font={\small\itshape}}
}
% -------------------------------------------------
% Start by placing the nodes
\node [term, fill = lcfree!25]              {Order};
\node [proc, join=by free]  (p1) {Task Hub};
\node [proc, dashed] (d) {};
\node [term, xshift = 12mm] (oc) {Order Check};
\node [term, join=by cong, fill = lccong!25]  {Submit};

\node [gathe, right=of d,yshift=5mm, xshift= 15mm] (g1) {};
\node [term, below=of d, yshift=15.5mm] (u1) {User 1};
\node [term, below=of u1] (u2) {User 2};
\node [below=-1mm of u2] (u3) {$\vdots$};
\node [term,below=of u3, yshift=-1mm] (u4) {User n};
\node [below=of u4,yshift=2.6mm] {User Cluster};
\node [gathe, below=of g1] (g2) {};
\node [coord, above=of g1, yshift=8mm] (c1) {};

\draw [->, lccong] (g2) -- (oc.west);
\draw [->, lcnorm] (p1) -- (u1.west);
\draw [->, lcnorm] (p1) -- (u2.west);
\draw [->, lcnorm] (p1) -- (u4.west);

\draw [->, lcnorm] (u1.east) -- (g1.west);
\draw [->, lccong] (u1.east) -- (g2.west);
\draw [->, lcnorm] (u2.east) -- (g1.west);
\draw [->, lccong] (u2.east) -- (g2.west);
\draw [->, lcnorm] (u4.east) -- (g1.west);
\draw [->, lccong] (u4.east) -- (g2.west);
\draw [->, lcnorm] (g1) -- (c1) -- node [black, near end, yshift=0.75em, it]
    {(Outsourcing)} ++(-3.5cm,0) -| (p1) ;

\end{tikzpicture}}
\caption{Simple Flow Chart of Order}
\label{fig:simpleorderflow}
\end{figure}


% subsection cloud_manufacturing(end)

% \subsection{Main complexity concepts} % (fold)
% \label{sub:main_complexity_concepts}

% \subsubsection{Self-organization}

% \subsubsection{Emergence}

% \subsubsection{Co-evolution}

% \subsubsection{Adaption}
% % subsection main_complexity_concepts (end)

\subsection{The evolution of manufacturing ecosystem} % (fold)
\label{sub:the_evolution_of_manufacturing_ecosystem}
Before the realization of the operation model mentioned above(\autoref{subs:Operation model of users}), we will analyze the evolution path in both macro and micro view of manufacturing ecosystem w.r.t. the domain in cloud manufacturing..
\subsubsection{Macroscopic evolution} % (fold)
\label{ssub:macroscopic_evolution}

% subsubsection microscopic_evolution (end)
\subsubsection{Microscopic evolution} % (fold)
\label{ssub:microscopic_evolution}

% subsubsection macroscopic_evolution (end)

\subsection{Main topics and their academic importance}
% 提一下云制造生态环境中的应用、模块,主要强调这些模块之间的交互,会产生怎样的作用。 目前云制造缺乏相关运作式的讨论,而商业模式中的不同行为会影响、改变、涌现。
As shown in \autoref{fig:structure},
Rating system, decision support system, recommender system,
detail info will seen in
Modular approaches are widely used to decompose a complex system into smaller subsystems according to their functions. For example, Yang and Li (2011) divided a cloud manufacturing services management and control platform into seven functional modules such as system management module, production management module and so on

% subsection the_evolution_of_manufacturing_ecosystem (end)
% section background (end)
