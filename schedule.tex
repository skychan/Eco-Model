% !TEX root = Eco-Model.tex
\section{Services Scheduling} % (fold)
\label{sec:schedule}

In order to properly accomplish the proposed tasks with optimal allocation of the manufacturing resources, a series of methodologies were combined to assist the manufacturing services scheduling process in Cloud Manufacturing Ecosystem, since manufacturing serivices are contrusted from manufacturing resources. 

Most studies on the service scheduling issue in Cloud Manufacturing like Lartigau et al.\cite{Lartigau2012} considered the whole Cloud Manufacturing system as one giant enterprise, so that it's easy to transform the problem into the classical flexible job shop problem. However, the heteroid Users in Cloud Manufacturing always not have the consistent objects. Hence, we should take the look at personalized scheduling support in both self-oriented and platform-auxiliary scheduling. (See \autoref{sub:decision_making_support} and \ref{sub:platform_assistance} respectively)

Most of the related methodologies can be archived into \textbf{four} main modules, namely
\begin{inparaenum}[1)]
\item Service standardization;
\item Decision making support;
\item Matching regulation and
\item Platform assistance,
\end{inparaenum} which are going to be explained in detail respectively.

The scheduling effect can be estimated and observed through \textit{Dashboard}, which also provides the control function that will help the evolution in the Ecosystem.

\subsection{Service Standardization} % (fold)
\label{sub:service_standardization}
Service standardization based mainly on  resource virtualization and encapsulation, where many models and methods were build on software and hardware like \cite{Li2011,Wu2011,Li2011a,Liu2011}. However, even if these resources were contrcuted into services with these models or methods, the difficulty of services scheduling changed slightly, for the absence of service standardization and similar type of services always turn out in different apperances.

Standardization is a process of Principal Component Analysis, like Guo et al.\cite{Guo2011} mined the correlation relationship of service composition in Cloud Manufacturing, we can use Online Learning\cite{Shalev-Shwartz2011} to mine out service prototypes and to refine them as case arrives by, where the sequential study cases are the continuous transactions.
% subsection service_standardization (end)

\subsection{Decision Making Support} % (fold)
\label{sub:decision_making_support}

For users in Cloud Manufacturing platfrom, their main acitvities are related to orders, no matter buyer or seller role they play in. The principal flow of order is shown in \autoref{fig:orderflow}, 
\begin{figure}[!h]
\centering\small\resizebox{0.6\textwidth}{!}{% !TEX root = Eco-Model.tex
\begin{tikzpicture}[%
    >=triangle 60,              % Nice arrows; your taste may be different
    start chain=going below,    % General flow is top-to-bottom
    node distance=6mm and 60mm, % Global setup of box spacing
    every join/.style={norm},   % Default linetype for connecting boxes
    ]
% ------------------------------------------------- 
% A few box styles 
% <on chain> *and* <on grid> reduce the need for manual relative
% positioning of nodes
\tikzset{
  base/.style={draw, on chain, on grid, align=center, minimum height=4ex},
  proc/.style={base, rectangle, text width=8em},
  test/.style={base, diamond, aspect=2, text width=5em},
  term/.style={proc, rounded corners},
  % coord node style is used for placing corners of connecting lines
  coord/.style={coordinate, on chain, on grid, node distance=6mm and 25mm},
  sky/.style={coordinate, on chain, on grid, node distance=6mm and 60mm},
  % nmark node style is used for coordinate debugging marks
  nmark/.style={draw, cyan, circle, font={\sffamily\bfseries}},
  % -------------------------------------------------
  % Connector line styles for different parts of the diagram
  norm/.style={->, draw, lcnorm},
  free/.style={->, draw, lcfree},
  cong/.style={->, draw, lccong},
  it/.style={font={\small\itshape}}
}
% -------------------------------------------------
% Start by placing the nodes
\node [term]              {Be Submitted};
\node [proc, join]        {Constrcut into Tasks};
\node [proc, join]  (p1)  {Evaluate the Seller};
\node [test, join]  (t1)  {Transact?};
\node [test]        (t2)  {Task Remain?};
\node [test]        (t3)  {Outsourcing?};
\node [proc]        (p2)  {Process Task and Submit Result};
\node [test, join]  (t4)  {Order Finished?};
\node [term]        (m1)  {End Smothly};

\node [sky, right=of p1] (c5) {};%\cmark{5}
\node [proc, right=of t1] (p3)  {Select Next Seller};
\node [test]        (t5)  {Partly or Whole?};
\node [proc]        (p5)  {Process Un-Outsouring Part and Submit Result};
\node [test, join]  (t6)  {Process Smothly?};
\node [proc]        (p4)  {Submit Outsourcing Part};
\node [coord, yshift=2.5mm]       (c4)  {}; %\cmark{4}
\node [term, right=of m1] (m2)  {End with Error};

\node [coord, left=of t2] (c1) {}; %\cmark{1}
\node [coord, right=of t3](c2) {}; %\cmark{2}
\node [coord, right=of t4](c3) {}; %\cmark{3}
\node [coord, right=of t5, xshift=3mm](c6) {}; %\cmark{6}
\node [coord, left=of c4, xshift=-1mm](c8) {}; %\cmark{8}

\path (t1.east) to node [near start, yshift=1em] {$n$} (p3);
  \draw [o->, lcnorm] (t1.east) -- (p3);
  \draw [->, lcnorm] (p3) |- (c5);
\path (t1.south) to node [near start, xshift=1em] {$y$} (t2);
  \draw [*->, lcnorm] (t1.south) -- (t2);
\path (t2.west) to node [near start, yshift=1em]  {$n$} (p2.west);
  \draw [o->, lcnorm] (t2.west) -- (c1) |- (p2.west);
\path (t2.south) to node [near start, xshift=1em] {$y$} (t3);
  \draw [*->, lcnorm] (t2.south)  -- (t3);
\path (t3.east) to node [near start, yshift=1em] {$y$}  (c2);
  \draw [*->, lcnorm] (t3.east) -| (c2) |- (t5.west);
\path (t3.south) to node [near start, xshift=1em] {$n$} (p2);
  \draw [o->, lcnorm] (t3.south) -- (p2);
\path (t4.south) to node [near start, xshift=1em] {$y$} (m1);
  \draw [*->, lcnorm] (t4.south) -- (m1);
\path (t4.east) to node [near start, yshift=-1em] {$n$} (c3);
  \draw [o->, lcnorm] (t4.east) -- (c3) |- (c4) -- (m2);
\path (t5.east) to node [near start, yshift=1em] {whole} (c6);
\path (t5.south) to node [near start, xshift=2em] {partly} (p5);
  \draw [o->, lcnorm] (t5.east) -- (c6) |- (p1);
  \draw [*->, lcnorm] (t5.south) -- (p5);
\path (t6.south) to node [near start, xshift=1em] {$y$} (p4);
\path (t6.west) to node [near start, yshift=1em] {$n$} (c8);
  \draw [o->, lcnorm] (t6.west) -| (c8); 
  \draw [*->, lcnorm] (t6.south) -- (p4);
  \draw [->, lcnorm] (p4.east) -| (c6);
\end{tikzpicture}}
\caption{Principal Flow of Order}
\label{fig:orderflow}
\end{figure}
users will make many decisions for
\begin{inparaenum}[1)]
\item Order acceptance;
\item Task decomposition;
\item Task (re-)outsourscing;
\item Seller selection;
\item Manufacturing arrangement
\end{inparaenum}
and so forth.

Among these decisions, Order acceptance, Seller selection and Manufacturing arrangement can be descript as a classic order acceptance and scheduling problem, which has received increasing attention in recent years\cite{Og˘uz2010,Nobibon2011,Cesaret2012}. Task decomposition and (re-)outsourscing can be descript as the decision making problem for users. Within the permission users get, they can tune the proportion of outsourscing.

% subsection decision_making_support (end)

\subsection{Matching Regulation} % (fold)
\label{sub:matching_regulation}
There are two main parts, recommendation for resource matching and resource prediction, of resources recommendation module that we will discuss in \autoref{sub:recommand_system}, and the first one depicted the situation where user in buyer role will always encounter.

In order to recommand poper manufacturing service with the order user submitted, Cloud Manufacturing platfrom should provide best-match solution, thus service search engine need to adopt a series of well designed matching regulations. Platform will collect the hit or miss data that feeded back from user, to train the classifier.
% subsection matching_regulation (end)

\subsection{Platform Assistance} % (fold)
\label{sub:platform_assistance}
Platform-auxiliary part of services scheduling is about the priority setting for tasks to be deal or for services to select, both are merely recommendation in the global view from administrator. Users can get bonus if they take the priority setting, but they won't get punishment if they don't.
As a result, two main issues have came: \begin{inparaenum}[1)]
\item How to make priority setting recommendation and
\item How to make decisions with the recommendation?
\end{inparaenum}
% subsection platform_assistance (end)

\subsection{Dashboard} % (fold)
\label{sub:service_dashboard}
Platform provides the Dashbord module for users to check the services usage and order process, comprehensive rank information, credit status and so on.
% subsection service_dashboard (end)
% section model (end)