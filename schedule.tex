% !TEX root = Eco-Model.tex
\section{Services Scheduling} % (fold)
\label{sec:schedule}

In order to properly accomplish the proposed tasks with optimal allocation of the manufacturing resources, a series of methodologies were combined to assist the manufacturing services scheduling process in Cloud Manufacturing Ecosystem, since manufacturing serivices are contrusted from manufacturing resources. 

===================================

A scheduling methodology for production services is presented
by [59]. Order decomposition into tasks, the selection of one or several
service providers to perform them, and the scheduling of the whole
manufacturing process is considered. In [60], a resource scheduling
framework is proposed. On the basis of stochastic advanced Petri net,
queue balancing cutover (QBC) strategy is put forward as a solution to
the issue of request dispatching. The scheduling of collaborative
design tasks is described in [61], presenting a new immune genetic
algorithm. Improved searching diversity based on immune strategy, but
also adaptive adjustment for probabilities of crossover and mutation
with low time complexity are achieved with this concept.

==================================

Most of the related methodologies can be archived into \textbf{four} main modules, namely \textit{Service Standardization}, \textit{Decision Making Support}, \textit{Matching Regulation} and \textit{Platform Assistance} which are going to explained in detail respectively.

The scheduling effect can be estimated and observed through \textit{Service Dashboard}, which also provides the control function that will help the evolution in the Ecosystem.

\subsection{Service Standardization} % (fold)
\label{sub:service_standardization}
Service standardization based mainly on  resource virtualization and encapsulation, where many models and methods were build on software and hardware like \cite{Li2011,Wu2011,Li2011a,Liu2011}. However, even if these resources were contrcuted into services with these models or methods, the difficulty of services scheduling changed slightly, for the absence of service standardization and similar type of services always turn out in different apperances.

Standardization is a process of Principal Component Analysis, like Guo et al.\cite{Guo2011} mined the correlation relationship of service composition in Cloud Manufacturing, we can use Online Learning\cite{Shalev-Shwartz2011} to mine out service prototypes and to refine them as case arrives by, where the sequential study cases are the continuous transactions.
% subsection service_standardization (end)

\subsection{Decision Making Support} % (fold)
\label{sub:decision_making_support}

OAS  
\begin{figure}[!h]
%\centering\small\resizebox{0.6\textwidth}{!}{% !TEX root = flow_head.tex
\begin{tikzpicture}[%
    >=triangle 60,              % Nice arrows; your taste may be different
    start chain=going below,    % General flow is top-to-bottom
    node distance=6mm and 60mm, % Global setup of box spacing
    every join/.style={norm},   % Default linetype for connecting boxes
    ]
% ------------------------------------------------- 
% A few box styles 
% <on chain> *and* <on grid> reduce the need for manual relative
% positioning of nodes
\tikzset{
  base/.style={draw, on chain, on grid, align=center, minimum height=4ex},
  proc/.style={base, rectangle, text width=8em},
  test/.style={base, diamond, aspect=2, text width=5em},
  term/.style={proc, rounded corners},
  % coord node style is used for placing corners of connecting lines
  coord/.style={coordinate, on chain, on grid, node distance=6mm and 25mm},
  % nmark node style is used for coordinate debugging marks
  nmark/.style={draw, cyan, circle, font={\sffamily\bfseries}},
  % -------------------------------------------------
  % Connector line styles for different parts of the diagram
  norm/.style={->, draw, lcnorm},
  free/.style={->, draw, lcfree},
  cong/.style={->, draw, lccong},
  it/.style={font={\small\itshape}}
}
% -------------------------------------------------
% Start by placing the nodes
\node [term] {Be Submitted};
\end{tikzpicture}}
\caption{Primary Flow of Order}
\label{fig:orderflow}
\end{figure}

% subsection decision_making_support (end)

\subsection{Matching Regulation} % (fold)
\label{sub:matching_regulation}
There are two main parts, recommendation for resource matching and resource prediction, of resources recommendation module that we will discuss in \autoref{sub:recommand_system}, and the first one depicted the situation where user in buyer role will always encounter.

In order to recommand poper manufacturing service with the order user submitted, service search engine should adopt a series of well designed matching regulations. Platform will collect the hit or miss data that feeded back from user, to train the classifier.
% subsection matching_regulation (end)

\subsection{Platform Assistance} % (fold)
\label{sub:platform_assistance}
a guide for scheduling 
% subsection platform_assistance (end)

\subsection{Service Dashboard} % (fold)
\label{sub:service_dashboard}
a monitor
% subsection service_dashboard (end)
% section model (end)